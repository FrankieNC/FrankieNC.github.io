\documentclass[12pt, a4paper]{article}
\usepackage{francesco}
\usepackage[colorlinks=true,
            hidelinks,
            pdfauthor={Francesco N. Chotuck},
            pdftitle={Groups and Symmetries Notes}
            ]{hyperref}
\hypersetup{urlcolor=RubineRed,linktoc=all, linkcolor=black}

\usetikzlibrary{bending}

%\pgfplotsset{width=\textwidth}

\pagestyle{fancy}
\lhead{Francesco Chotuck}
\rhead{5CCM232A Groups and Symmetries}
\setlength{\headheight}{15pt}

\DeclareMathOperator{\aut}{Aut}
\DeclareMathOperator{\inn}{Inn}
\DeclareMathOperator{\out}{Out}
\DeclareMathOperator{\Tr}{Tr}
\DeclareMathOperator{\orb}{orb}

\title{Groups and Symmetries Notes}
\date{}
\author{Francesco Chotuck}
\begin{document}
\maketitle

\begin{abstract}
    This is KCL undergraduate module 5CCM232A, instructed by Dr Paul P. Cook. The formal name for this class is ``Groups and Symmetries''.
\end{abstract}
    
\tableofcontents

\pagebreak

%% COMMON USED ITEMS
%\langle \bm{x}, \bm{y} \rangle

%%WEEK 1

\section{Basics}

\begin{definition}
    A \textbf{group} is a set $G$ and a mapping from the Cartesian product $G \times G$ into $G$ which will we denote by juxtaposition (variables side by side) \[\begin{aligned}
        G \times G &\to G \\
        (g_1,g_2) &\mapsto g_1g_2,
    \end{aligned}\] with the following properties: 
    \begin{itemize}
        \item Associativity: \(g_1(g_2g_3)=(g_1g_2)g_3\) for all \(g_1,g_2,g_3 \in G\);
        \item Identity: there exists \(e\in G\), called an identity, such that \(ge=eg=g\) for all \(g\in G\);
        \item Inverse: for all \(g \in G\) there exists \(g \inv \in G\), called an inverse of \(g\), such that \(gg \inv=g \inv g=e\), where \(e\) is an identity in \(G\).
    \end{itemize}
\end{definition}

\begin{mdremark}
    When asked to define a group state this: ``a group is a set with a multiplication law \(G \times G \to G\) with \((g_1,g_2) \mapsto g_1g_2\) which satisfies etc\dots''.
\end{mdremark}

Elements in groups have some "nice" properties:

\begin{enumerate}
    \item The identity \(e\) is unique;
    \item Given any \(g\in G\) its inverse \(g \inv\) is unique;
    \item \((gh)\inv = h\inv g\inv\) for all \(g,h \in G\);
    \item \((g\inv)\inv=g\) for all \(g\in G\).
\end{enumerate}

\begin{definition}
    The order of a group \(G\) is the number of elements of \(G\), denoted \(\left|G\right|\).
\end{definition}

\begin{definition}
    If \(gh=hg\) for all \(g,h \in G\), then \(G\) is a \textbf{commutative}, or \textbf{abelian}, group.
\end{definition}

\begin{definition}
    A subset \(H \subset G\) is a subgroup of \(G\) if it is a group under the law of composition of \(G\). That is, \(H\) is a subgroup if 
    \begin{enumerate}
        \item \(h_1h_2 \in H\) for all \(h_1,h_2 \in H\) and
        \item \(h\inv \in H\) for all \(h \in H\).
    \end{enumerate}
\end{definition}

\begin{mdexample}
    Some interesting examples of groups are:
    \begin{itemize}
        \item \(\ZZ_p\) : the integers under addition modulo \(p \in \NN\);
        \item Matrices with determinant \(1\) under the matrix multiplication (determinant of \(1\) is necessary to guarantee the existence of inverses);
    \end{itemize}
\end{mdexample}

\begin{mdnote}
    In this course the set \(\RR^* = \RR \backslash {0}\).
\end{mdnote}

\pagebreak

\section{The Cyclic Groups}

Let \(g\in G\) be an element of a group \(G\) which has identity \(e\). The power notation for group elements \(g^n\) where \(n \in \ZZ\) is defined to be the \(n\)-times product of \(g\) with itself i.e. \(\underbrace{gg\ldots g}_{n \text{ times}}\), for \(n>0\); it is the \(\abs{n}\text{-times}\) product of \(g\inv\) with itself for \(n<0\), and it is \(e\) if \(n=0.\) Consequently, we define 
\[g^{m+n}=g^mg^n, \quad (g^m)^n = g^{mn} \quad \forall m,n \in \ZZ.\]

\begin{definition}
    A group \(G\) is called \textbf{cyclic} if there is exists a \(g \in G\) such that \(G =\langle g \rangle\). Such an element is called a \textbf{generating element} for \(G\); in general it is \textbf{not} unique.
\end{definition}

\begin{theorem}
    The set generated by \(g \in G\), denoted \(\langle g \rangle\), is the set of all powers if \(g\):
    \[\langle g \rangle =\{g^n : n \in \ZZ\}.\]
\end{theorem}

\begin{mdnote}
All cyclic groups are abelian.
\end{mdnote}

\begin{theorem}
    If \(p\) is prime then \(\ZZ_p= \langle n \rangle\) for any \(n\in\{1,2,3,\ldots, p-1\}\).
\end{theorem}

\begin{theorem}
    Let \(G\) be a cyclic group generated by \(g_0\). Then
    \begin{enumerate}
        \item \(g_0^n\) for \(n=0,1,2,\ldots, \abs{G}-1\) are all \textbf{distinct} elements and
        \item \(g^{\abs{G}} = e\) for all \(g\in G\).
    \end{enumerate}
\end{theorem}

\begin{theorem}
    Every subgroup of a cyclic group is cyclic.
\end{theorem}

\subsection{Symbols and relations}

The notion of a generating element can be extended to more than one element. Let \(\langle a,b \rangle\)  is the set of all powers of \(a\) and \(b\) and their products i.e. \[\langle a,b \rangle =\{e,a,a^2,\ldots,b,b^2,\ldots,ab,ab^2,\ldots,ba,ba^2,\ldots,abab, \ldots\}\] is called the span of \(a\) and \(b\), or the group generated by \(a\) and \(b\). By imposing restriction such as \(a^n =e, b^n =e\) and \(a^nb^n=b^na^n\) for any \(n\in \ZZ\) the span of \(a\) and \(b\) becomes a finite set which allows us to denote a group.

\pagebreak

\section{Maps and Permutation Groups}

Consider two sets \(X,Y\) and a map \(f:X \to Y\). We write \(y=f(x)\) for the value in \(Y\) that is mapped from \(x \in X\).

\begin{definition}
    We say \(y\) is the \textbf{image} of \(x\) under \(f\).
\end{definition}

\begin{definition}
    The set \(f(X) = \{f(x) : x \in X\} \subset Y\) is the image of \(X\) under \(f\).
\end{definition}

\begin{definition}
    The map \(f\) is \textbf{surjective} (or \textbf{onto}) if every \(y \in Y\) is the image of at least one \(x \in X\) i.e. if \(f(X)=Y\) (denoted \(f : X \twoheadrightarrow Y\)).
\end{definition}

\begin{definition}
    The map \(f\) is \textbf{injective} (or \textbf{one-to-one}) if for all \(y\in f(X)\) there exists a \textbf{unique} \(x \in X\) such that \(y=f(x)\) (denoted \(f:X \rightarrowtail Y\)).
\end{definition}

\begin{proposition}
    A function is said to be injective if \(f(x_1)=f(x_2)\) implies that \(x_1=x_2\).
\end{proposition}

\begin{definition}
    A map \(f\) that is both injective and surjective is said to be \textbf{bijective}.
\end{definition}

\begin{theorem}
    If \(f:X\to Y\) is bijective an \textbf{inverse} map \(f\inv:Y\to X\) can define such that 
    \[f(f\inv(y))=y, \quad f\inv(f(x))=x \quad \forall y\in Y, x\in X.\]
\end{theorem}

\begin{mdremark}
The inverse map itself is bijective.
\end{mdremark}

\begin{theorem}
    The span of a set of bijective maps of a finite set \(X\) to itself forms a group under composition of maps. This is called a \textbf{permutation group} of \(X\), \(\text{Perm}(X)\).
\end{theorem}

\begin{definition}
    The set of all permutations of a finite set containing \(n\) elements is called the \textbf{symmetric group} \(S_n\).
\end{definition}

\begin{mdnote}
    The set \(\text{Perm}(X) \in S_{\abs{X}}\).
\end{mdnote}

\begin{theorem}
The order of the symmetric group \(\abs{S_n}\) is \(n!\).
\end{theorem}

\begin{mdexample}
    \(S_3\) is a symmetric group of all permutations of three elements. It consists of the permutations: 
    \[\begin{aligned}
        \begin{pmatrix} 1 & 2& 3 \\ 1 & 2& 3 \end{pmatrix} &=e, 
        \begin{pmatrix} 1 & 2& 3 \\ 2 & 3 & 1 \end{pmatrix} =a,
        \begin{pmatrix} 1 & 2& 3 \\ 3 & 1& 2 \end{pmatrix} &= a^2, \\
        \begin{pmatrix} 1 & 2& 3 \\ 2 & 1& 3 \end{pmatrix} &=b, 
        \begin{pmatrix} 1 & 2& 3 \\ 1 & 3& 2 \end{pmatrix} =a^2b,
        \begin{pmatrix} 1 & 2& 3 \\ 3 & 2& 1 \end{pmatrix}&=ab,
    \end{aligned}\]
    Notice how the bottom row is cycled through by 'shifting to the right' i.e. bottom row for \(e = (123) , a = (231), a^2 = (312)\) and so on.
\end{mdexample}

\begin{mdremark}
    A \(2\)-cycle such as \((12)\) is called a \textbf{transposition} and every permutation can be rewritten as a product of a transposition. The converse is also true i.e. every cycle permutation can be rewritten as a transposition.
\end{mdremark}

\begin{definition}
    An \textbf{even permutation} is one which can be written as the product of an even number of transposition.
\end{definition}

\begin{definition}
    An \textbf{odd permutation} is one which can be written using an odd number of transposition.
\end{definition}

\begin{definition}
    The \textbf{alternating group} \(A_n\) is a subgroup of \(S_n\) consisting of all the even permutations in \(S_n\).
\end{definition}

\begin{theorem}
    The order of \(A_n\) is always half the permutations in \(S_n\), hence \(\abs{A_n} = \frac{n!}{2}\).
\end{theorem}

\subsection{Cycle composition}

\begin{definition}
    Two (or more) cycles are \textbf{disjoint} if they do not have any common elements.
\end{definition}

\begin{theorem}
    Every permutation \(\sigma \in S_n\) can be decomposed to a composition of \textbf{disjoint cycles}. 
\end{theorem}

\begin{mdexample}
    Let \(\sigma = (173)(15) \in S_7\). Write the decomposition of \(\sigma\) into disjoint cycles. \\
    \textbf{Solution:} The cycles \((173)\) and \((15)\) are \textbf{NOT} disjoint since they both share the element \(1\). \\
    Let \(\sigma = \underbrace{(173)}_{\alpha} \underbrace{(15)}_{\beta} = \alpha \beta.\) Then 
    \[\begin{aligned}
        \sigma(1) &= \alpha \beta(1) = \alpha(\beta(1)) =\alpha(5) =5 \\
        \sigma(2) &= \alpha \beta(2) = \alpha(\beta(2)) =\alpha(2) =2 \\
        \sigma(3) &= \alpha \beta(3) = \alpha(\beta(3)) =\alpha(3) =1 \\
        \sigma(4) &= \alpha \beta(4) = \alpha(\beta(4)) =\alpha(4) =4 \\
        \sigma(5) &= \alpha \beta(5) = \alpha(\beta(5)) =\alpha(1) =7 \\
        \sigma(6) &= \alpha \beta(6) = \alpha(\beta(6)) =\alpha(6) =6 \\
        \sigma(7) &= \alpha \beta(7) = \alpha(\beta(7)) =\alpha(7) =3.
    \end{aligned}\]
    So, \(\sigma =(1573)\).
\end{mdexample}

%%WEEK 2

\pagebreak

\section{Homomorphism and Isomorphism}

\begin{definition}
    Let \(G\) and \(H\) be groups. A map \(\phi : G \to H\) is a \textbf{homomorphism} (of groups) if \(\phi(gh)=\phi(g)\phi(h)\) for all \(g,h \in G\).
\end{definition}

\begin{mdremark}
    The operation on the LHS is the operation of the group \(G\), whereas the operation on the RHS is that of group \(H\); i.e.
    \[\phi(g \underbrace{\circ}_{\text{in }G} h) = \phi(g) \underbrace{\circ}_{\text{in } H} \phi(h).\]
\end{mdremark}

\begin{mdnote}
    For a group homomorphism given by \(\phi\) note that
    \begin{itemize}
        \item the notation \(\phi\inv(g)\) means the \textbf{inverse map} of \(\phi\) whereas,
        \item the notation \((\phi(g))\inv\) means the \textbf{inverse element} of \(\phi(g)\).
    \end{itemize}
\end{mdnote}

\begin{theorem}
    Let \(\phi\) be a homomorphism from \(G\) to \(H\) and \(e_G,e_H\) be the identity elements in \(G\) and \(H\) respectively, then
    \begin{enumerate}
        \item \(\phi(e_G)=e_H\) and 
        \item \(\phi(g\inv)=(\phi(g))\inv\) for all \(g \in G\)
    \end{enumerate}
\end{theorem}

\begin{proof}
    We prove each statement in turn.
    \begin{enumerate}
        \item Since \(e_G e_G =e_G\) then, \(\phi(e_G)=\phi(e_G e_G)=\phi(e_G)\phi(e_G)\). Multiplying by the inverse element \(\left( \phi(e_G) \right)\inv\) in \(H\), and we have \(\phi(e_G)\phi(e_G)\inv = \phi(e_G)\phi(e_G)\phi(e_G)\inv\) hence, \(e_H = \phi(e_G)\).
        \item Consider \(e_G = gg\inv\) for \(g \in G\) then, using the previous result we have \(e_H = \phi(e_G) = \phi(gg\inv)=\phi(g)\phi(g\inv)\). Multiplying by the inverse element \(\phi(g)\inv\) in \(H\) we have \(\phi(g)\inv=\phi(g)\inv\phi(g)\phi(g\inv)=\phi(g\inv)\).
    \end{enumerate}
\end{proof}

\begin{mdnote}
    If we want to check whether if two groups are homomorphic we need to check  whether Definition \(4.1\) and Theorem \(4.1\) hold.
\end{mdnote}

\begin{definition}
    An \textbf{isomorphism} is a homomorphism that is bijective.
\end{definition}

\begin{example}
    A trivial example of an isomorphism is the identity map \(\id :G \to H\).
\end{example}

\begin{definition}
    Two groups \(G\) and \(H\) are \textbf{isomorphic} if there exists an isomorphism \(\phi : G\to H\).
\end{definition}

\begin{mdthm}
    The 'isomorphic' relation is an equivalence relation, denoted by \(\cong\).
\end{mdthm}

\begin{mdremark}
    If two groups are isomorphic then they are structurally the same group; i.e. if \(G \cong H\) then: \(\abs{G}=\abs{H}; G \text{ abelian} \iff H \text{ abelian}\); etc.
\end{mdremark}

\begin{theorem}
    The \textbf{order of an element} \(g \in G\) is the smallest positive integer \(k\) such that \(g^k =e\).
\end{theorem}

\begin{corollary}
    Let \(\phi : G \to H\) be an isomorphism and let \(g \in G\) then \(\phi(g^n)=\left( \phi(g) \right)^n\).
\end{corollary}

\begin{mdthm}
    Two cyclic groups of the same order are isomorphic.
\end{mdthm}

\begin{proof}
    Let \(G = \langle g \rangle\) and \(H = \langle h \rangle\) with \(N = \abs{G}=\abs{H}\) where \(N\) is finite. Since, \(G\) and \(H\) are cyclic groups we can write 
    \[G = \{g^0,g^1,g^2,\ldots, g^{N-1}\} \quad \text{and} \quad G = \{h^0,h^1,h^2,\ldots, h^{N-1}\}\]
    as sets. Define the map 
    \[\begin{aligned}
        \phi : G &\to H \\
        \phi(g^n) &\mapsto h^n
    \end{aligned}\]
    fot all \(n \in \ZZ\). We will show that this is an isomorphism. By contruction it is bijective. To show it is a homomorphism, consider 
    \[\begin{aligned}
        \phi(g^m g^n) &= \phi(g^{m+n}) \\
        &= h^{m+n} \\
        &=h^{m} h^n \\
        &=\phi(g^m)\phi(g^n).
    \end{aligned}\]
\end{proof}

\pagebreak

\section{Cosets and Lagrange's theorem}

Let \(H \subset G\) be a subgroup of the group \(G\). We can define an equivalence relation between \(a,b \in G\) if they are related by \textbf{left multiplication} by an element of \(H\), i.e. if 
\[ha=b \quad \text{then} \quad ba\inv =h \in H \quad \text{for some } h\in H.\]

So we define an equivalence relation \(\sim\) by:
\[a\sim b \iff ba\inv \in H.\]

\begin{mdexample}
    Consider the group 
    \[G = \left\{ \begin{pmatrix} a & 0 \\ b & a\end{pmatrix} : a \in \RR,\; a\neq 0, \; b \in \RR \right\} .\]
    Let 
    \[H = \left\{ \begin{pmatrix} a & 0 \\ 0 & a\end{pmatrix} : a \in \RR,\; a\neq 0 \right\}\]
    be a subgroup of \(G\). Find matrices which, up to equivalence, form the left-coset space \(G/H\).
    \begin{solution}
        Recall, two elements \(M,N \in G\) lie in the same left coset with respect to \(H\) if \(M\inv N \in H\). Let 
        \[M = \begin{pmatrix} a_1 & 0 \\ b_1 & a_1\end{pmatrix} \quad \text{and} \quad \begin{pmatrix} a_2 & 0 \\ b_2 & a_2\end{pmatrix}\]
        then, 
        \[M\inv = \frac{1}{a_1} \begin{pmatrix} 1 & 0 \\ -\frac{b_1}{a_1} & 1\end{pmatrix}.\]
        We want 
        \[\begin{aligned}
            M\inv N &= \begin{pmatrix} 1 & 0 \\ -\frac{b_1}{a_1} & 1\end{pmatrix}\begin{pmatrix} a_2 & 0 \\ b_2 & a_2\end{pmatrix} \\
            &= \frac{1}{a_1}\begin{pmatrix} a_2 & 0 \\ -\frac{b_1a_2}{a_1}+b_2 & a_2\end{pmatrix} \in H.
        \end{aligned}\]
        Therefore, we must have \(-\frac{b_1a_2}{a_1}+b_2=0\) which implies 
        \[\frac{b_1}{a_1} =\frac{b_2}{a_2} \equiv \lambda \in \RR.\]
        Setting \(a_1 =1\) gives
        \[G/H = \left\{ \begin{pmatrix} 1 & 0 \\ \lambda & 1\end{pmatrix} : \lambda \in \RR \right\}.\]
    \end{solution}
\end{mdexample}

\begin{definition}
    The \textbf{equivalence class} of \(a\) is denoted by \([a]:= \{b \in G: b\sim a\}.\)
\end{definition}

Denote the set \(Ha:=\{ha: h \in H\}\) then, this will be shown as \([a]=Ha = \{e,h_1,h_2,\ldots, h_{n-1}\}a =\{a,h_1a,h_2a,\ldots,h_{n-1}a\}\), (supposing the set has \(n\) elements).

\begin{definition}
    The set of equivalence classes \(Ha = [a]=\{Ha: a\in G\}\) is the set of \textbf{right cosets} of \(G\) with respect to \(H\), where \(G\) is a group and \(H \subset G\) is a subgroup of \(G\).
\end{definition}

\begin{definition}
    The set of equivalence classes \(\{aH : a \in G\}\) is the set of \textbf{left cosets} of \(G\) with respect to \(H\).
\end{definition}

\begin{mdremark}
    For left cosets we define the equivalence relation as \(a\sim b\) if and only \(ah=b\) for some \(h\in H\). Or equivalently \(a\sim b\) if and only if \(a\inv b \in H\).
\end{mdremark}

\begin{theorem}
    Two right cosets of \(G\) with respect to \(H\) are either \textbf{disjoint} or \textbf{identical}.
\end{theorem}

\begin{proof}
    Let \(a,b \in G\). If \([a]\) and \([b]\) have no elements in common, then they are disjoint. If \(c \in [a]\) and \(c \in [b]\), then \(a\sim c\) and \(b\sim c\), hence \(a\sim b\) by transitivity and symmetry therefore, \([a]=[b]\).
\end{proof}

\begin{mdnote}
    When listings cosets if one the cosets has \textbf{at least} on element in common with another coset then stop listing, as such coset is identical to one listed before.
\end{mdnote}

\begin{theorem}
    All right cosets of \(G\) with respect to \(H\) have the same number of elements.
\end{theorem}

\begin{proof}
    It is enough to show there is a bijection between two right cosets. \\
    Fix \(g \in G\) and consider its right coset \(Hg\). Consider the map
    \[\begin{aligned}
        M : H &\to Hg \\
        h &\mapsto hg.
    \end{aligned}\]
    We now show it's a bijective map. Given \(a \in Hg\) we have \(a =hg\) for some \(h \in H\) hence, \(a= M(g)\) so, \(M\) is surjective. Suppose \(M(h)=M(h')\) then \(hg=h'g\) hence, \(hgg\inv=h'gg\inv \then h=h'\) so, \(M\) is injective.
\end{proof}

\begin{mdnote}
    In mathematics when wanting to prove two sets have the same cardinality it is enough to show that there is a bijection between the two sets.
\end{mdnote}

\begin{definition}
    The number of cosets of \(G\) with respect to \(H\) is called the \textbf{index} of \(H\) in \(G\), which is denoted by \(i(H,G)\).
\end{definition}

\begin{mdthm}[Lagrange's Theorem]
    Let \(H\) be a subgroup of \(G\). The order of \(H\) divides the order of \(G\) i.e. \[\abs{G}=\abs{H} i(H,G).\]
\end{mdthm}

\begin{definition}
    A \textbf{proper subgroup} of \(G\) is a subgroup \(H \subset G\) that is different from the trivial group \(\{e\}\) and from \(G\) itself.
\end{definition}

\begin{corollary}
    If \(\abs{G}\) is prime, then the group \(G\) has no proper subgroup.
\end{corollary}

\begin{proof}
    For the sake of contradiction, suppose \(H\) is a proper subgroup of \(G\) then, \(\abs{H}\) divides \(\abs{G}\) however, \(\abs{H}\) is a number not equal to \(1\) or \(\abs{G}\).
\end{proof}

\begin{corollary}
    Let \(g \in G\) and let \(k\) be the order of \(g\). Then \(k\) divides \(\abs{G}\).
\end{corollary}

\begin{proof}
    The order of \(g\) is the same as the order of \(\langle g \rangle\), the subgroup of \(G\) generated by \(g\). By applying Lagrange's theorem on \(\langle g \rangle\) we conclude that the order of \(g\) divides the order of \(G\).
\end{proof}

\begin{corollary}
    If \(\abs{G}\) is prime then \(G\) is a cyclic group.
\end{corollary}

\begin{proof}
    Given any \(g \in G\) where \(g \neq e\), consider the subgroup \(\langle g \rangle\). Since \(\abs{G}\) is prime it has no proper subgroups. Since the order of \(\langle g \rangle\) is greater than \(1\) then \(\langle g \rangle\) must be \(G\) itself.
\end{proof}

%%WEEK 3

\pagebreak

\section{Groups of Low order}

We will identify or construct all the groups of low order.

\begin{itemize}
    \item \(\abs{G} =1 \then G=\{e\}.\) This is the only possibility as \(e\) must always be contained in the group.
    \item \(\abs{G}=2 \then G=\{e,a\}.\) With \(a\neq e\), as \(G\) is a group then \(a\inv\) exists and is \(a\inv =a \then a^2=e\). Since the order of \(G\) is \(2\), that is prime then we have \(G\cong \ZZ_2\).
    \item \(\abs{G}=3 \then G=\{e,a,b\}.\) As \(\abs{G}\) is prime then \(G\cong \ZZ_3\). 
    \item \(\abs{G}=4 \then G=\{e,a,b,c\}\)
\end{itemize}

\begin{theorem}
    Every group of order \(4\) is either cyclic or has the rules:
    \begin{figure}[H]
        \begin{center}
            \noindent\begin{tabular}{c | c c c c }
                & $e$ & $a$ & $b$ & $c$   \\
                \cline{1-5}
               $e$ & $e$ & $a$ & $b$ & $c$  \\
               $a$ & $a$ & $e$ & $c$ & $b$  \\
               $b$ & $b$ & $c$ & $e$ & $a$  \\
               $c$ & $c$ & $b$ & $a$ & $e$ \\
            \end{tabular}
        \end{center}
    \end{figure}
\end{theorem}

\begin{mdnote}
    In a Cayley table there are no repeat entries in a row and column.
\end{mdnote}

\begin{mdremark}
    If Cayley table is symmetric i.e. switch the rows with column then the group is abelian.
\end{mdremark}

\subsection{The Klein's Four Group}

\begin{definition}
    The group with the rules above is denoted \(V_4\) and called the \textbf{Klein four-group} (Vieregruppe).
\end{definition}

\begin{mdthm}
    The group \(V_4\) can be described using symbols and relations. It is generated by the symbols \(a,b\) with the relations \(a^2=b^2=e\) and \(ab=ba\). So we have 
    \[V_4=\langle a,b\rangle =\{e,a,b,ab\}.\]
\end{mdthm}

\begin{proposition}
    All groups of order \(5\) or less are abelian.
\end{proposition}

\begin{proposition}
    Properties of \(V_4\):
    \begin{enumerate}
        \item it is the smallest non-cyclic group;
        \item All the elements have order \(2\) (for exception to the identity);
        \item it has \(5\) subgroups: the trivial ones as well as the \(3\) proper subgroups, \(\langle a \rangle,\langle b \rangle\) and \(\langle ab \rangle\).
    \end{enumerate}
\end{proposition}

\begin{theorem}
    The group \(V_4\) can be seen as a subgroup of \(S_4\):
    \[\left\{e,\begin{pmatrix} 1 & 2 & 3 & 4 \\ 2 & 1 & 4 & 3\end{pmatrix}, \begin{pmatrix} 1 & 2 & 3 & 4 \\ 3 & 4 & 1 & 2\end{pmatrix},\begin{pmatrix} 1 & 2 & 3 & 4 \\ 4 & 3 & 2 & 1\end{pmatrix}\right\}.\]
\end{theorem}

\begin{proposition}
    \(V_4 \cong \ZZ_2 \times \ZZ_2\).
\end{proposition}

\pagebreak

\section{Direct Products}

\begin{definition}
    Let \(G\) and \(H\) be groups. Then \(G\times H = \{(g,h):g\in G,h\in H\}\) is a group, with the multiplication law \((g,h)(g',h') = (gg',hh')\). We call \(G\times H\) the \textbf{direct product} of \(G\) and \(H\).
\end{definition}

\begin{proposition}
    If \(G\) and \(H\) are abelian then so is \(G\times H\).
\end{proposition}

\begin{proposition}
    The order of \(G\times H\) is \(\abs{G\times H} =\abs{G}\abs{H}\).
\end{proposition}

\begin{mdthm}
    A group of order \(6\) is isomorphic to \(\ZZ_6\) (the cyclic group of order \(6\)) or to \(S_3\).
\end{mdthm}

\begin{example}
    Consider the group \(\ZZ_2\times \ZZ_3\); this group has order \(6\). Is this group isomorphic to \(\ZZ_6\) or \(S_3\)? We note that \(\ZZ_2\times \ZZ_3\) is abelian and so is \(\ZZ_6\) however \(S_3\) is not. Hence, we conclude that 
    \[\ZZ_2\times \ZZ_3 \cong \ZZ_6.\]
\end{example}

\begin{mdthm}
    The group \(\ZZ_p \times \ZZ_q\) is isomorphic to \(\ZZ_{pq}\) if and only if \(p\) and \(q\) are coprime.
\end{mdthm}

\begin{lemma}
    All groups of even order contain at least one non-identity element whose order is \(2\).
\end{lemma}

\begin{proof}
    Consider a finite group of even order where \(G = \{e,g_1,g_2,\ldots, g_{2n-1}\}\) for \(n \in \ZZ\). For the sake of contradiction, suppose that \(G\) does not contain any non-identity element of order \(2\) and note that such an element is its own inverse element. We can pair up each element with its unique inverse element. But, \(e\) is its own inverse element which means that we have an odd number of remaining elements to be ordered into distinct pairs -- which cannot be done. Hence, the assumption is contradicted.
\end{proof}

\begin{example}
    \hphantom{This is to make it look nice}
    \begin{itemize}
        \item Consider the group \(\ZZ_2 = \{e,a\}\), it has \(a^2=e\);
        \item the group \(\ZZ_4 = \{e,a,a^2,a^3\}\) has \((a^2)^2=e\).
    \end{itemize}
\end{example}

\pagebreak

\section{Symmetry transformations}

\subsection{Symmetries and groups}

\begin{definition}
    A \textbf{symmetry transformation} is an action on a set that leaves the set as a whole \textbf{unaltered}.
\end{definition}

% We will investigate the isometries (distance-preserving symmetries) of regular polygons embedded in \(\RR^2\). 

% We will consider the circle: \(x^2+y^2=1\). Its isometries include:

% \begin{itemize}
%     \item rotations about the origin and
%     \item reflections in any straight line passing through the origin.
% \end{itemize}

\begin{mdnote}
    An even function \(f(x)\) has the property \(f(x)=f(-x)\), it is symmetric under the map \(x \mapsto -x\). In \(\RR^2\) this is a symmetric reflection in the \(y\)-axis. An odd function satisfies \(g(x)=-g(-x)\), and it is symmetric reflection about the origin.
\end{mdnote}

\subsection{Isometries of the Euclidean Plane}

\begin{definition}
    Let \(X\) and \(Y\) be two vector spaces equipped with distance functions \(D_X\) and \(D_Y\). An isometry between \(X\) and \(Y\) is a distance preserving map \(f:X\to Y\) i.e.
    \[D_X(x_1,x_2)=D_Y(y_1,y_2)\]
    where \(f(x_1)=y_1\) and \(f(x_2)=y_2\).
\end{definition}

The transformations of \(\RR^2\) can be described with matrices:
\begin{itemize}
    \item an anti-clockwise rotation of \(\theta\) about the origin is given by 
    \[
        A(\theta) = \begin{pmatrix}
                        \cos\theta & -\sin\theta \\
                        \sin\theta & \cos\theta
                    \end{pmatrix}.
    \]
    Its inverse is given by 
    \[A(-\theta)= \begin{pmatrix}
        \cos\theta & \sin\theta \\
        -\sin\theta & \cos\theta
    \end{pmatrix};\]
    \item a reflection in the straight line through the origin at angle \(\theta\) to the \(x\)-axis is 
    \[B(\theta) = A(\theta)BA(-\theta)\]
    where \(B\) is a reflection in the \(x\)-axis, so \(B=\begin{pmatrix}
        1&0 \\
        0&-1
    \end{pmatrix}\) i.e. \(B\begin{pmatrix}
        x\\y
    \end{pmatrix} = \begin{pmatrix}
        x \\ -y
    \end{pmatrix}.\)
\end{itemize}

Geometrically the operation \(B(\theta)\) is a reflection on the straight line inclined by \(\theta\) as indicated in the figure below.
\begin{figure}[H]
    \begin{center}
        \includegraphics[scale=0.5]{Resources/Reflection matrix.png}
    \end{center}
\end{figure}

Similarly consider the operation \(B(\theta) = A(\theta) B A(-\theta)\), we do the operation from right to left. We will explain the geometric interpretation with the aid of the diagram below.

\begin{figure}[H]
    \begin{center}
        \includegraphics[scale=0.4]{Resources/Reflection matrix as rotation.png}
    \end{center}
\end{figure}

Consider the straight line through the origin inclined at \(\theta\) and a triangle which we want to apply \(B(\theta)\):
\begin{enumerate}
    \item we rotate the line down to the \(x\)-axis with \(A(-\theta)\);
    \item then apply a reflection on the \(x\)-axis with \(B\);
    \item restore the line by moving back by \(\theta\) with \(A(\theta)\).
\end{enumerate}

\begin{mdprop}
    The properties of \(A(\theta)\) and \(B(\theta)\):
    \begin{itemize}
        \item \(A(\theta+\pi)=-A(\theta)\);
        \item \(B(\theta+\pi)=B(\theta)\);
        \item \(A(\theta)BA(\theta) = B \then A(\theta)B=BA(-\theta)\);
        \item \((A(\theta))^\top=A(-\theta)\);
        \item \(B^\top=B\);
        \item \((B(\theta))^\top = B(\theta)\).
    \end{itemize}
\end{mdprop}

\begin{definition}
    Let \(n\geq 2\) be an integer. The set of rotations and reflections that preserve the regular \(n\)-sided polygon \(P_n\) is called \textbf{dihedral group}, \(D_n\).
\end{definition}

\begin{mdthm}
    Let \(n \geq 2\) be an integer. The set of rotations and reflections that preserve the regular polygon \(P_n\), formed by successively joining the points
    \[\begin{pmatrix}
        \cos\left( \frac{2\pi k}{n} \right) \\
        \sin \left( \frac{2\pi k}{n} \right)
    \end{pmatrix}, \: k=0,1,\ldots, n-1\] 
    by straight lines, is called the \textbf{dihedral group} \(D_n\).
\end{mdthm}

\begin{mdremark}
    The case \(n=2\) is a special case: \(P_2\) is a line segment and not a polygon. We have that \(D_2 \cong V_4\).
\end{mdremark}

\begin{proposition}
    The order of the groups \(D_n\) is given by \(\abs{D_n}=2n\).
\end{proposition}

\begin{mdthm}
    The full set of the dihedral group elements is:
    \[D_n =\langle a,b \rangle = \{\underbrace{e,a,a^2,\ldots, a^{n-1}}_{n \text{ rotations}}, \underbrace{b,ab,a^2b,\ldots, a^{n-1}b}_{n \text{ reflections}}\}\]
    equipped with \(a^n =e, b^2=e\) and \(a^kb =ba^{-k}\) as 
    \[
    \underbrace{\left( A\left( \frac{2\pi}{n} \right) \right)^k B}_{a^k b} = A\left( \frac{2\pi k}{n} \right)B = B A\left(  - \frac{2\pi k}{n} \right) = \underbrace{B \left( A\left( -\frac{2\pi}{n} \right) \right)^k}_{ba^{-k}}.
    \]
\end{mdthm}

\begin{mdremark}
    We have that \(D_3 \cong S_3\).
\end{mdremark}

\begin{corollary}
    A combination of rotations and reflection can always be represented by a reflection.
\end{corollary}

\begin{mdcor}
    The order of every reflection is \(2\).
\end{mdcor}

\pagebreak

\section{Conjugation, Normal Subgroups and Quotient Groups}

\subsection{Conjugation}

\begin{definition}
    Given a group \(G\), we sat that \(a\) is \textbf{conjugate} to \(b\) if there exists a \(g \in G\) such that \(a=gbg\inv\) for \(a,b\in G\).
\end{definition}

\begin{mdremark}
    Every finite group \(G\) is made up of some set of permutation in the symmetric group, if \(\abs{G}=n\) then we can embed \(G\) into \(S_n\) i.e. \(G \subset S_{\abs{G}}\). This is known as \textbf{Cayley's theorem} which can be restated as: every group \(G\) is isomorphic to a subgroup of a symmetric group.
\end{mdremark}

\begin{mdnote}
    We can interpret conjugation as changing labels of the elements.
\end{mdnote}

\begin{example}
    Conjugation in the symmetric group relates the permutations formed out of the same length cycles where the only difference is that the labels in the cycles are change. Thus, conjugation is a way of relating similar transformations. Consider the permutation \((123) \in S_3\). If we swapped the labels of the elements \(2 \leftrightarrow 3\)  we would find another \(3\)-cycle permutation, namely \((132)\). We can carry out the swap \(2 \leftrightarrow 3\) using elements of \(S_3\) as follows:
    \[
        (23)(123)(23)\inv=(23)(123)(23)=(132).
    \]
    The operation \((23)(123)(23)\) is a conjugation of \((123)\) with \(g=(23)\).
\end{example}

\begin{theorem}
    The conjugacy relation is an equivalence relation.
\end{theorem}

% \begin{proof}
%     We need to check the properties of an equivalence relation for conjugation.
%     \begin{itemize}
%         \item Reflexivity: \(a\) is conjugate to itself as \(a=eae\inv\) and \(e \in G\).
%         \item Symmetry: if \(a\) is conjugate to \(b\) then \(a=gbg\inv\) for some \(g\in G\), therefore \(b=g\inv a g=g\inv a \left( g\inv \right)\inv\) which means \(b\) is conjugate to \(a\).
%         \item Transitivity: if \(a\) is conjugate to \(b\) and \(b\) is conjugate to \(c\) then, \(a=gbg\inv\) and 
%     \end{itemize}
% \end{proof}

\begin{definition}
    The set \([a]_C = \{gag\inv : g\in G\}\) is called the \textbf{conjugacy class} of \(a\).
\end{definition}

\begin{mdprop}
    Two or more conjugacy classes are either disjoint or identical.
\end{mdprop}

\begin{mdremark}
    Conjugation is an equivalence relation.
\end{mdremark}

\begin{theorem}
    Conjugacy classes form a partition of \(G\) however, it is \textbf{not} an equipartition.
\end{theorem}

\begin{proposition}\label{prop: conjugacy properties}
    Properties of conjugacy classes:
    \begin{itemize}
        \item \([e]_C = \{e\}\) hence, no other class is a subgroup as the identity element is not contained in all classes.
        \item All elements of a conjugacy class have the same order.
        \item If \(G\) is abelian then \([a]_C\ = \{a\}\) for all \(a\in G\).
    \end{itemize}
\end{proposition}

\begin{corollary}
    Suppose \(G\) is a group and \(a,b,g \in G\). Let \(b=gag\inv\) then \(b^k=ga^k g\inv\); i.e. the order is preserved.
\end{corollary}

\begin{mdcor}
    Suppose \(G\) is a group then all elements of \(G\) in the same conjugacy class are elements of the same order.
\end{mdcor}

\begin{proof}
    Suppose that the order of \(a \in G\) is \(k\) i.e. \(a^k=e\). Consider an element \(b \in [a]_C\), i.e. \(b \sim a \then b=gag\inv\). We have that \(b^n = ga^n g\inv\), the lowest value of \(n\) for which \(b^n=e\) is when \(n=k\), the order of \(a\), i.e. \(b^k = ga^k g\inv = geg\inv = gg\inv =e\). Hence, all elements in the same conjucacy class have the same order.
\end{proof}

\begin{proposition}
    If \(G\) is an abelian group then \([a]_C =\{a\}\) for all \(a\in G\).
\end{proposition}

\begin{proof}
    As \([a]_C =\{gag\inv \mid \forall a \in G\} =\{gg\inv a \mid \forall g \in G\} = \{a\}\).
\end{proof}

\begin{theorem}
    On any subset \(H \subset G\), the conjugation map 
    \[\begin{aligned}
        M:H &\mapsto gHg\inv \\
        h &\mapsto ghg\inv
    \end{aligned}\]
    associated to \(g\in G\) is bijective.
\end{theorem}

\begin{example}
    In this example we illustrate how the group \(D_3\) is partitioned in respect to its conjugacy classes. Notice how it is not an equipartition.
    \begin{figure}[H]
        \begin{center}
            \includegraphics[width=\textwidth]{Resources/D3 conjucacy class partition.png}
        \end{center}
    \end{figure}
\end{example}

\begin{mdexample}
    Consider \(D_3 \cong S_3\). We know \(D_3 = \langle a,b \rangle\) with \(a^3=e, b^2 =e\) and \(ab=ba\inv=ba^2\) so, \(D_3 = \{e,a,a^2,b,ab,a^2b\}\). Let us look at the order of each element: 
    \[D_3 = \{\underbrace{e}_{1},\underbrace{a,a^2}_{3},\underbrace{b,ab,a^2b}_{2}\}.\]
    Firstly, by Proposition \ref{prop: conjugacy properties} we have our first conjugacy class,
    \[[e]_C =\{e\};\]
    secondly we have that all elements in a conjugacy class have the same order thus, we can guess that the conjugacy classes could be 
    \[[a]_C = \{a,a^2\} \quad \text{and} \quad [b]_C =\{b,ab,a^2b\}.\]
    Now we need to check if they are conjugacy classes:
    \begin{itemize}
        \item \([a]_C\) can only contain \(a\) and \(a^2\); by definition \(a\) is contained in its conjugacy class, so we need to check if \(a^2 \in [a]_C\). We can split the elements of \(D_3\) (in general \(D_n\)) into rotations and reflections i.e. \(a^n\) and \(a^n b\). Now check if \(a^2\) can be conjugated by rotations and reflections:
        \[\begin{aligned}
            a^n a^2 a^{-n} &=a^2 \checkmark \\
            (a^n b) a^2 (a^n b)\inv &= (a^n b) a^2 (ba^{-n}) = a^2. \checkmark
        \end{aligned}\]
        We conclude that \(a^2 \in [a]_C\).
        \item Similarly, we need to check if \(ab,a^2b\in [b]_C\); as before \(b \in [b]_C\) by definition of a conjugacy class. So, check if \(b\) can be conjugated by rotations and reflections:
        \[
            a^n b a^{-n} = a^{2n}b =\{\underbrace{b}_{n=0},\underbrace{a^2 b}_{n=1} ,\underbrace{ab}_{n=2}\}.\checkmark
        \]
        This is enough as all remaining elements of \(D_3\) are listed in the conjugacy class of \([b]_C\), thus we do not need to check for rotations in this case.
        We conclude that the conjugacy classes of \(D_3\) are as follows:
        \[\begin{aligned}
            [e]_C &=\{e\} \\
            [a]_C &= \{a,a^2\} \\
            [b]_C &=\{b,ab,a^2 b\}.
        \end{aligned}\]
    \end{itemize}
\end{mdexample}

\subsection{Normal subgroups}

\begin{definition}
    A subgroup \(H\) is called \textbf{normal} or \textbf{invariant} if \(gHg\inv \subset H\) for all \(g\in G\) (equivalently \(ghg\inv \in H\) \(\forall h \in H\) and \(\forall g\in G\)). \\
    If \(H\) is a \textbf{finite group} then \(gHg\inv =H\). \\
    If \(H\) is normal in \(G\) we write \(H \triangleleft G\).
\end{definition}

\begin{mdnote}
    We can interpret a normal subgroup to be a group which under conjugation it remains in itself.
\end{mdnote}

\begin{mdremark}
    The notation \(gHg\inv = \{ghg\inv : h\in H\}\). That is, \(H\) is a normal if for every \(h\in H\) and every \(g \in G\), we have \(ghg\inv \in H\).
\end{mdremark}

\begin{mdprop}
    Let \(H \subset G\). Properties of normal subgroups:
    \begin{itemize}
        \item If \(H\) is normal then \(gHg\inv = H\) for all \(g \in G\).
        \item The previous property implies \(gH=Hg\).
        \item The group \(H\) is normal to \(G\) if and only if \(H\) is the union of the entire conjugacy classes.
        \item \(\{e\}\) is a (trivial) normal subgroup.
        \item Every subgroup of an abelian group is normal.
        \item \(G\) and \(H\) are normal subgroups of \(G\times H\).
    \end{itemize}
\end{mdprop}

\begin{mdremark}
    The last property does not imply that every normal group is abelian. 
\end{mdremark}

\begin{example}
    Some examples of normal subgroups:
    \begin{itemize}
        \item every subgroup of \(\ZZ_n\) is normal as \(\ZZ_n\) is abelian;
        \item \(V_4 \cong \ZZ_2 \times \ZZ_2\) is a normal subgroup of \(A_4\), as \(V_4 =[e]_C \cup [(12)(34)]_C \subset A_4\);
    \end{itemize}
\end{example}

\begin{definition}
    A group is \textbf{simple} if it has no proper normal subgroup(s).
\end{definition}

\begin{mdnote}
    We can think of simple groups as atomic groups i.e. they cannot be formed by other groups.
\end{mdnote}

\begin{example}
    Some examples of groups which are simple or not:
    \begin{itemize}
        \item We have \(\ZZ_p\) for \(p\) is a prime are simple groups, while \(\ZZ_{pq} \cong \ZZ_p \times \ZZ_q\) are not.
        \item The dihedral groups are not simple groups as \(\langle a \rangle \triangleleft D_n\).
        \item \(A_5\) is a simple group.
        \item In general the groups \(A_n\) with \(n>4\) are all simple groups.
    \end{itemize}
\end{example}

\begin{definition}
    A group is \textbf{semi-simple} if it has no proper abelian normal subgroup.
\end{definition}

\begin{definition}
    The \textbf{centre} \(Z(G)\) of a group \(G\) is the set of elements which commute with all elements of \(G\):
    \[
        Z(G) = \{a\in G : ag=ga \hphantom{h} \forall g \in G\}.
    \]
\end{definition}

\begin{example}
    If \(G\) is abelian then \(Z(G)=G\).
\end{example}

\begin{mdthm}
    The centre \(Z(G)\) of a group is a normal subgroup.
\end{mdthm}

\begin{proof}
    We first show that \(Z(G)\) is indeed a subgroup.
    \begin{itemize}
        \item Closure: let \(a,b \in Z(G)\) and \(g \in G\) then \(abg=agb=gab\) hence, \(ab \in Z(G)\).
        \item Identity: \(e \in Z(G)\).
        \item Inverse: let \(a \in Z(G)\) and \(g \in G\) then, \(ag\inv =g\inv a\) hence, \(ga\inv=a\inv g\) which implies \(a\inv \in Z(G)\).
        \item Associativity: \(Z(G)\) has the same group multiplication rule as \(G\), which is associative as \(G\) is a group by construction.
    \end{itemize}
    We now prove \(Z(G)\) is a normal subgroup. For \(a \in Z(G)\) and \(g \in G\) we have that \(gag\inv=gg\inv a=a\in Z(G)\).
\end{proof}

\begin{corollary}
    Consequently, if \(G\) is a finite simple group then \(Z(G)=\{e\}\) or \(Z(G)=G\).
\end{corollary}

\begin{mdprop}
    The centre of the group \(D_n\):
    \begin{itemize}
        \item \(Z(D_n)= \{e\}\) for odd \(n\);
        \item \(Z(D_n) = \{e,a^{\frac{n}{2}}\}\) for even \(n\).
    \end{itemize}
\end{mdprop}

\begin{proof}
    Consider \(D_n = \langle a,b \rangle\) with \(a^n=e,b^2 =e\) and \(a^k b= ba^{-k}\). Elements of \(Z(D_n)\) satisfy \(zg=gz\) for all \(g \in D_n\) and \(z \in Z(D_n)\), but we know that 
    \[a^k b=ba^{-k}.\]
    Hence, if any element of the form \(a^m \in D_n\) is in \(Z(D_n)\) we would require \(a^m b =ba^m\); by imposing the condition of \(D_n\) we need \(a^{-m}=a^m\), for it to commute with \(b\). Now, \(a^{-m}=a^m \then e=a^{2m}\) but we know that \(a^n=e\) thus, the condition becomes \(a^n = a^{2m}\). Therefore, if \(n\) is odd then \(a^m\) is not in the centre whereas, if \(n\) is even then \(m = \frac{n}{2}\) and \(a^{\frac{n}{2}}\) is in the centre. We check the remaining elements of \(D_n\). 
    \begin{itemize}
        \item For odd \(n\): \\
        \[(a^m b)b = ba^{-m}b=b(a^{-m}b)\]
        so, \(a^m b\) is not in the centre as \(a^m b\neq a^{-m} b\) for odd \(n\).
        \item For even \(n\): \\
        \[(a^mb)a = a^m a\inv b =a\inv(a^m b) = a^{n-1}(a^m b)\]
        and for \(n >2\) we have \(a\inv \neq a^{n-1}\).
    \end{itemize}
\end{proof}

% \begin{example}
%     Consider \(D_n=\{e,a,a^2,\ldots, a^{n-1}, b, ab, a^2b, \ldots, a^{n-1}b\}\) with \(a^n =e, b^2=e, a^k b =ba^{-k}\). Elements of \(Z(D_n)\) satisfy \(zg=gz\) for all \(g \in D_n\) but we know that 
%     \[a^kb =ba^{-k}.\]
%     Hence, if any element of the form \(a^m \in D_n\) is in \(Z(D_n)\) we would require \(a^{-m}=a^m\), now
% \end{example}

\subsection{Quotients}

\begin{definition}
    Let \(G\) be a group and \(H\) a subgroup of \(G\).
    \begin{itemize}
        \item The \textbf{quotient} \(G/H = \{gH : g \in G\}\) is the set of all \textbf{left-cosets.}
        \item The \textbf{quotient} \(H\backslash G = \{Hg : g\in G\}\) is the set of all \textbf{right-cosets.}
    \end{itemize}
\end{definition}

\begin{definition}
    Given two subsets \(A\) and \(B\) of \(G\), the multiplication of a set \(A\) by \(B\) is defined by element wise multiplication, \(AB:=\{ab : a\in A, b\in B\}.\)
\end{definition}

\begin{mdthm}
    If \(H\) is normal to \(G\) then, the quotient \(G/H\) with the above multiplication law on subsets is a group.
\end{mdthm}

\begin{mdnote}
    The theorem above also applies to right cosets, however we will now focus on left cosets.
\end{mdnote}

\begin{proof}
    We need to check the axioms of a group.
    \begin{itemize}
        \item Closure: We have \((g_1H)(g_2 H)=g_1Hg_2H = g_1(g_2Hg_2\inv)g_2H=g_1g_2HH=g_1g_2H\), where we have used the fact that \(H\) is a normal subgroup of \(G\).
        \item Associativity: inherent from the associativity of \(G\) and the relation above.
        \item Identity: \(e \in H\) is the identity as \(eH=H\). 
        \item Inverse: the inverse of \(gH\) is \(g\inv H\) under the multiplication law.
    \end{itemize}
\end{proof}

\section{Kernel and Image of a group}

\begin{definition}
    Let \(\phi\) be a homomorphism of \(G\) onto \(H\) (i.e. \(\phi\) is a surjective map to \(H\)). Then the \textbf{kernel} of \(\phi\) is 
    \[\ker \phi =\{g\in G : \phi(g)=e_H\}\]
    where \(e_H\) is the identity element of \(H\).
\end{definition}

\begin{mdremark}
    Observe that \(e_G \in \ker \phi\).
\end{mdremark}

\begin{example}
    Consider the map \(\phi : D_3 \to \ZZ_2\) given by 
    \[\phi(a^n b^m) =b^m\]
    (i.e. \(b^m \in \langle b \rangle \cong \ZZ_2\)). This map is indeed a homomorphism as:
    \[\phi(a^{n_1} b^{m_1})\phi(a^{n_2} b^{m_2}) = b^{m_1} b^{m_2}\]
    while
    \[\begin{aligned}
        \phi(a^{n_1} b^{m_1} a^{n_2} b^{m_2}) &= \phi(a^{n_1} a^{(-1)^{m_1}n_2} b^{m_1} b^{m_2}) \\
        &= b^{m_1} b^{m_2} \\
        &=\phi(a^{n_1} b^{m_1})\phi(a^{n_2} b^{m_2}).
    \end{aligned}\]
    Then, given \(b^m =e\)
    \[\begin{aligned}
        \ker \phi &= \{a^n b^m \in D_3 : \phi(a^n b^m) = e \in \ZZ_2\} \\
        &= \{a^n b^m : \forall n \in \ZZ \text{ and } m \in 2\ZZ\} \\
        &= \{a^n \in D_3 : \forall n \in \ZZ\}\\
        &= \langle a \rangle \\
        &\cong \ZZ_3.
    \end{aligned}\]
\end{example}

\begin{mdthm}
    A group homomorphism \(\phi : G\to H\) is an isomorphism if and only if it is surjective and \(\ker \phi = \{e_G\}\).
\end{mdthm}

\begin{mdremark}
    We can reformulate the theorem above as follows:
    A group homomorphism \(\phi : G\to \Img \phi\) is an isomorphism if and only if \(\ker \phi = \{e_G\}\).
    By the way \(\phi\) has been defined, the surjectivity of \(\phi\) is guaranteed.
\end{mdremark}

\begin{mdnote}
    In other words the theorem is saying that a map (of groups) is injective if and only if \(\ker\phi =\{e_G\}\).
\end{mdnote}

\begin{mdthm}
    The kernel is a normal subgroup.
\end{mdthm}

\begin{mdnote}
    This is a powerful theorem as it allows us to find a normal subgroup rather simply.
\end{mdnote}

\begin{example}
    Let \(\phi : D_3 \to \ZZ_2\) given by \(\phi(a^n b^m) = b^m\). We have already shows that this is a homomorphism where we found that \(\ker \phi = \langle a\rangle \cong \ZZ_3\). Therefore, \(\ZZ_3 \cong \langle a \rangle \triangleleft D_3\).
\end{example}

\begin{definition}
    Let \(\phi\) be a homomorphism of \(G\) on \(H\). Then the \textbf{image} of \(\phi\) is
    \[\Img \phi = \phi(G) = \{\phi(g) : g\in G\}\]
\end{definition}

\begin{mdremark}
    Notice that for \(\phi : G\to H\) we have that \(\ker \phi \subset G\) and \(\Img \phi \subset H\).
\end{mdremark}

\begin{mdthm}
    The image \(\Img \phi \) of a group homomorphism \(\phi :G\to H\) is a subgroup of \(H\).
\end{mdthm}

\subsection{The homomorphism theorem}

We know that \(G / \ker \phi\) is a quotient group since \(\ker \phi \triangleleft G\), so the question becomes which group it is.

% \begin{example} 
%     \hphantom{space} \\
%     Let \(\RR^* = \RR \backslash \{0\}\) be a group under multiplication. \\
%     Let \(\RR^+ = \{x \in \RR : x>0\}\) be a group under multiplication. \\
%     Consider the map \(\phi : \RR^* \to \RR^+\) given by \(\phi(x)=\abs{x}\). This is a homomorphism onto \(\RR^+\) as for \(x,x' \in \RR^+\) we have 
%     \[\begin{aligned}
%         \phi(xx')&=\abs{xx'} \\
%                 &=\abs{x}\abs{x'}\\
%                 &=\phi(x)\phi(x').
%     \end{aligned}\]
%     Its kernel is 
%     \[\begin{aligned}
%         \ker \phi &= \{\phi(x) =1 : x \in \RR^*\} \\
%                 &=\{-1,1\} \\
%                 &\cong \ZZ_2 \text{ (under multiplication)}.
%     \end{aligned}\]
%     Therefore \(\ZZ_2 \triangleleft \RR^*\) since it is the kernel of \(\phi\). We can form the quotient group 
%     \[\begin{aligned}
%         R^* / \ZZ_2 &=\{x \ZZ_2 : x\in \RR^*\}\\
%                     &=\{\{-x,x\} : x\in \RR^*\}.
%     \end{aligned}\]
% \end{example}

\begin{mdthm}[The Homomorphism Theorem]
    Let \(G\) and \(H\) be groups and \(\phi: G \to H\) be a homomorphism then, \(G/\ker \phi \cong \Img \phi\).
\end{mdthm}

\begin{example}
    Let \(\phi:D_3 \to \ZZ_2\) given by \(\phi(a^n b^m)=b^m\). This is a homomorphism so, by the homomorphism theorem 
    \[D_3 / \ZZ_3 \cong \ZZ_2,\]
    as \(\ker \phi =\langle a \rangle \cong \ZZ_3\) and \(\Img \phi =\langle b \rangle \cong \ZZ_3\).
\end{example}

\begin{mdthm}
    Given a group \(G\) and a normal subgroup \(H \subset G\) then there exists a homomorphism \(\phi: G \to G/H\) which is surjective such that \(H = \ker \phi\).
\end{mdthm}

\begin{proof}
    If \(g \in G\) define \(\phi(g)=gH\). This is a homomorphism: 
    \[\begin{aligned}
        \phi(gg')&=gg'H \\
        &=gHg'H \\
        &=\phi(g)\phi(g').
    \end{aligned}\]
    Its kernel is 
    \[\begin{aligned}
        \ker\phi&=\{g \in G :gH=H\} \\
        &= \{g \in G : g \sim e\} \\
        &= eH \\
        &= H.
    \end{aligned}\]
\end{proof}

\begin{corollary}
    Simple group, having no non-trivial normal subgroup, admit only trivial homomorphism i.e. if \(\ker \phi = \{e\}\) then 
    \(\Img \phi \cong G\) or, if \(\ker \phi = G\) then \(\Img \phi =\{e\}\).
\end{corollary}

\pagebreak

\section{Automorphism}

\begin{definition}
    An \textbf{automorphism} is an isomorphism of \(G\) onto itself.
\end{definition}

\begin{mdnote}
    An automorphism is a homomorphism \(\phi : G\to G\) such that \(\phi\) is surjective.
\end{mdnote}

\begin{definition}
    For each element \(a\in G\), define the map \(\phi_a :G\to G\) by \(\phi_a(g)=aga\inv\) for all \(g\in G\).
    \begin{itemize}
        \item An \textbf{inner automorphism} is an automorphism \(\phi\) such that \(\phi=\phi_a\) for some \(a\in G\).
        \item If an automorphism \(\phi\) is not inner, then it is called an \textbf{outer automorphism}.
    \end{itemize}
\end{definition}

\begin{mdnote}
    An inner automorphism is an automorphism which admits to a \textbf{conjugation} automorphism.
\end{mdnote}

\begin{definition}
    \hphantom{space}\\
    Notation:
    \begin{itemize}
        \item The \textbf{set of all automorphism} of a group \(G\) is denoted \(\aut(G)\).
        \item The set of all \textbf{inner} automorphism of a group \(G\) is denoted \(\inn(G)\).
        \item The set of all \textbf{outer} automorphism of a group \(G\) is denoted \(\out(G)\).
    \end{itemize}
\end{definition}

\begin{example}
    Let \(a\in G\) and define \(\phi_a : G \to G\) by \(\phi_a(g) = aga\inv\) for all \(g\in G\) (this is the conjugation of \(g\) by \(a\)). Then \(\phi_a\) is an inner automorphism.
    \begin{itemize}
        \item Homomorphism: \(\phi_a(g_1g_2) = ag_1g_2a\inv = ag_1 \underbrace{a\inv a}_{e} g_2a\inv = \phi_a(g_1)\phi_a(g_2).\)
        \item Bijection:
        \begin{itemize}
            \item Surjective: given \(g\in G\) there exists \(g'\in G\) such that \(\phi_a(g')=g \then ag'a\inv =g\) therefore, \(g'=a\inv ga \in G\).
            \item Injective: Suppose \(\exists g_1\neq g_2 \in G\) such that \(\phi_a(g_1)=\phi_a(g_2)\) then \(ag_1a\inv=ag_2a\inv \then g_1=g_2\) which is a contradiction.
        \end{itemize}
    \end{itemize}
\end{example}

\begin{corollary}
    If \(G\) is an abelian group then \(\inn(G)=\{\id\}\).
\end{corollary}

\begin{mdnote}
    The identity map being \(\phi_a(g) =g\).
\end{mdnote}

\begin{mdthm}
    The set of all automorphism, \(\aut(G)\), is a group under composition. Furthermore, subset \(\inn(G)\) is a normal subgroup to \(\aut(G)\) i.e. \(\inn(G) \triangleleft \aut(G)\).
\end{mdthm}

\begin{mdremark}
    \(\aut(G) / \inn(G)\cong \out(G)\).
\end{mdremark}

\begin{mdthm}
    Let \(G\) be a group then \(G / Z(G) \cong \inn(G)\).
\end{mdthm}

\pagebreak

\section{Matrix groups}

\subsection{Basics}

\begin{definition}
    The set of all \(N\times N\) matrices with elements in \(\RR\) and \(\CC\) are denoted by \(M_N(\RR)\) and \(M_N(\CC)\) respectively.
\end{definition}

\begin{theorem}[Matrix operation]
    We can combine matrices \(A\) and \(B\) to find another matrix by
    \begin{itemize}
        \item Addition: \(A+B=C\) where in components \(C_{ij} =A_{ij}+B_{ij}\).
        \item Matrix multiplication: \(AB=C\) where, in components, \(C_{ij}=\sum_{k=1}^N A_{ik} B_{kj}\).
    \end{itemize}
\end{theorem}

\begin{mdnote}
    The notation for components form, \(C_{ij}\), means the entry on the \(i^{\text{th}}\) \textbf{row} and the \(j^{\text{th}}\) \textbf{column}. So, in matrix multiplication to get the element in the \(i^{\text{th}}\) row and \(j^{\text{th}}\) column of the product \(BA\), take the scalar product of the \(i^{\text{th}}\) row-vector of \(B\) with the \(j^{\text{th}}\) column vector of \(A\). 
\end{mdnote}

\begin{definition}
    A matrix \(A\) is \textbf{invertible} if there exists a matrix \(A\inv\) such that \(AA\inv =A\inv A=I\).
\end{definition}

\begin{mdthm}
    Given any matrix \(A\), we may
    \begin{itemize}
        \item Take its \textbf{complex conjugate}, \(A^*: (A^*)_{jk}=(A_{jk})^*\).
        \item Take its \textbf{transpose}, \(A^{\top}:(A^{\top})_{jk}=A_{kj}\).
        \item Take its \textbf{adjoint}, \(A^{\dagger}=(A^{\top})^*=(A^*)^{\top}\).
    \end{itemize}
\end{mdthm}

\begin{mdnote}
    We can interpret the \textit{adjoint} as the \textbf{complex-conjugate transpose}.
\end{mdnote}

\begin{definition}
    A matrix \(A\) is
    \begin{itemize}
        \item \textbf{self-adjoint} if \(A^{\dagger}=A\).
        \item \textbf{symmetric} if \(A^{\top}=A\).
        \item \textbf{unitary} if \(A^{\dagger} =A\inv\).
        \item \textbf{diagonal} if \(A_{jk} =0\) for all \(j\neq k\).
    \end{itemize}
\end{definition}

\begin{mdthm}
    Properties of the determinant:
    \begin{itemize}
        \item \(\det(I)=1\).
        \item \(\det(AB)=\det(A)\det(B)\)
        \item If \(A\inv\) exists then \(\det(A\inv)=(\det(A))\inv = \frac{1}{\det(A)}\).
        \item \(\det(A)\neq 0\) if and only if \(A\) is invertible.
        \item \(\det(A^*)=(\det(A))^*\).
        \item \(\det(A^{\top})=\det(A)\).
        \item \(\det(\lambda A)=\lambda^N \det(A)\), where \(A\) is an \(N\times N\) matrix.
        \item If \(A\) is a diagonal \(N\times N\) matrix then \(\det(A)=\prod_{j=1}^N A_{jj}\), i.e. \(\det(A)\) is the product of the diagonal entries.
        \item \(\det(SAS\inv)=\det(A)\).
        \item For an \(N \times N\) matrix \(\det(\det(A)\mathbb{I})=(\det(A))^N\).
    \end{itemize}
\end{mdthm}

\begin{proof}
    We provide a proof of the last property. The matrix \(\det(A)\mathbb{I}\) is a diagonal matrix with each diagonal entry being \(\det(A)\) therefore, \(\det(\det(A)\mathbb{I})= \det(A)^N\) i.e. the product of the diagonal entries.
\end{proof}

\subsection{The classical groups as matrix groups}

\subsubsection{The General Linear Group}

\begin{definition}
    The \textbf{general linear group} is defined as all \(N\times N\) matrices with complex entries and non-zero determinant. It is denoted by
    \[\GL(N,\CC) =\{A \in M_N(\CC) : \det(A) \neq 0\}.\]
\end{definition}

\begin{corollary}
    The set of \(N \times N\) matrices with real entries is a subgroup of \(\GL(N,\CC)\) i.e.
    \[\begin{aligned}
        \GL(N,\RR) &=\{A \in M_N(\RR) : \det(A) \neq 0\}\\ 
                    &\subset \GL(N,\CC).
    \end{aligned}\]
\end{corollary}

\begin{mdthm}
    Consider the maps
    \[\det : \GL(N,\CC) \to \CC^* = \CC \backslash \{0\} \quad \text{and} \quad \det : \GL(N,\RR) \to \RR^* = \RR \backslash \{0\}\]
    are both surjective homomorphisms.
\end{mdthm}

\begin{proof}
    To show the map is surjective take a \ul{diagonal} matrix with \(\lambda\) for one of the diagonal entries and the rest of the diagonal to be \(1\). To show it is a homomorphism use the property of \(\det(AB)=\det(A)\det(B)\).
\end{proof}

\begin{mdthm}
    We have 
    \[Z(\GL(N,\CC)) = \{\lambda \mathbb{I} : \lambda \in \CC^*\}\cong \CC^*\]
    and
    \[Z(\GL(N,\RR))= \{\lambda \mathbb{I} : \lambda \in \RR^*\} \cong \RR^*.\]
\end{mdthm}

\begin{mdremark}
    That is the centre of the general linear groups is the set of all diagonal matrices.
\end{mdremark}

\subsubsection{The Special Linear Group}

\begin{definition}
    The \textbf{special linear group} is defined as all the \(N \times N\) matrices with complex entries and a determinant of \(1\). It is denoted by 
    \[\SL(N,\CC) = \{A \in M_N(\CC) : \det(A) =1\}.\]
\end{definition}

\begin{mdremark}
    The special linear group with real entries is similarly defined and is denoted by \(\SL(N,\RR)\).
\end{mdremark}

\begin{theorem}
    The group \(\SL(N,\RR)\) is a normal subgroup of \(\GL(N,\CC)\).
\end{theorem}

\begin{proof}
    By definition \(\SL(N,\CC)=\ker(\det)\), where by \("\det"\) we mean the map
    \[\det: \GL(N,\CC) \to \CC^*.\]
\end{proof}

\begin{theorem}
    \(\GL(N,\CC) / \SL(N,\CC) \cong \CC^*\).
\end{theorem}

\begin{mdremark}
    Similar theorems are equivalent with real entries.
\end{mdremark}

\begin{mdthm}
    We have
    \[Z(\SL(N,\CC)) \cong \ZZ_N\]
    and
    \[\begin{aligned}
        Z(\SL(N,\RR)) \cong \begin{cases}
            \ZZ_2 &\text{if } N \text{ is even} \\
            \{\mathbb{I}\} &\text{if } N \text{ is odd}.
        \end{cases}
    \end{aligned}\]
\end{mdthm}

\begin{mdremark}
    The symbol \(\mathbb{I}\) denotes the \(N \times N\) identity matrix.
\end{mdremark}

\begin{proof}
    We outline a proof for each case.
    \begin{itemize}
        \item As before the centres must be matrices proportional to the identity matric i.e. the centre is of the form \(\{\lambda \mathbb{I} : \lambda \in \CC^*\}\). However, for \(A \in \SL(N,\CC)\) we require \(\det(A) = \lambda^N =1\) which implies that \(\lambda\) must be an \(N\)-th root of unity which are isomorphic to \(\ZZ_N\). 
        \item Similarly, we require \(\det(A) = \lambda^N =1\) hence, if \(N\) is even we have \(\lambda \in \{-1,1\} \cong \ZZ_2\) and if \(N\) is odd we have \(\lambda = 1 \cong \{\mathbb{I}\}\).
    \end{itemize}
\end{proof}

\subsubsection{The Unitary Group}

\begin{definition}
    The \textbf{unitary group} is the group
    \[U(N) = \{A \in \GL(N,\CC) : A^{\dagger}=A\inv\}\]
    where \(A^{\dagger} = (A^*)^\top=(A^\top)^*\).
\end{definition}

\begin{mdnote}
    We can think of \(\dagger\) as the "complex-conjugate transpose".
\end{mdnote}

\begin{mdprop}
    Some properties of \textbf{unitary matrices}, we take \(A \in U(N)\):
    \begin{enumerate}
        \item \(A^{\dagger}A = \mathbb{I}\).
        \item \(\abs{\det(A)}=1\).
        \item \(\det(A) = e^{i\theta}\) for some \(\theta\).
    \end{enumerate}
\end{mdprop}

\begin{corollary}
    The group \(U(1)\) can be expressed as
    \[\begin{aligned}
        U(1) &= \{z\in \CC : zz^*=1\} \\
            &= \{z \in \CC : \abs{z}=1\}.
    \end{aligned}\]
\end{corollary}

\begin{mdnote}
    The group \(U(1)\) parametrises the unit circle, \(S^1\), as \(z=e^{i\theta}\).
\end{mdnote}

\begin{mdthm}
    The map \(\det : U(N) \to U(1)\) is a surjective homomorphism.
\end{mdthm}

\begin{proof}
    To prove surjective let \(z \in \CC\) with \(\abs{z}=1\) and choose a \ul{diagonal} matrix with an entry \(z\) and the remaining diagonal entries to be \(1\).
\end{proof}

\subsubsection{The Special Unitary Group}

\begin{definition}
    The \textbf{special unitary group} is the group 
    \[SU(N) = \{A\in U(N) : \det(A) =1\}.\]
\end{definition}

\begin{theorem}
    The group \(SU(N)\) is a normal subgroup to \(U(N)\).
\end{theorem}

\begin{corollary}
    We have 
    \[U(N) / SU(N) \cong U(1).\]
\end{corollary}

\begin{proof}
    Consider the map \(\det: U(N) \to U(1)\) with \(\ker(\det)=SU(N)\). By the \\ homomorphism theorem we get the isomorphism.
\end{proof}

\subsubsection{The Orthogonal Group}

\begin{definition}
    The \textbf{orthogonal group} is the group 
    \[O(N)=\{A \in M_N(\RR) : A^{\top} = A\inv\}.\]
\end{definition}

\begin{corollary}
    The orthogonal group is a subgroup of the unitary group as \(O(N) \subset U(N)\).
\end{corollary}

\begin{proposition}
    If \(A \in O(N)\) then \((\det(A))^2 =1\). If \(\det(A) \in \RR\) then \(\det(A) \in \{-1,1\} \cong \ZZ_2\).
\end{proposition}

\begin{mdthm}
    The map \(\det : O(N) \to \ZZ_2\) is a surjective homomorphism.
\end{mdthm}

\subsubsection{The Special Orthogonal Group}

\begin{definition}
    The \textbf{special orthogonal group} is the group 
    \[SO(N) = \{A \in O(N) : \det(A) =1\}.\]
\end{definition}

\begin{mdthm}
    We have that \(O(N) / SO(N) \cong \ZZ_2\).
\end{mdthm}

\begin{proof}
    Recall \(\det:O(N) \to \ZZ_2\) is surjective also, \(\ker(\det)=SO(N)\) hence, by the \\ homomorphism theorem we have that 
    \[O(N) / SO(N) \cong \ZZ_2.\]
\end{proof}

\begin{mdthm}
    If \(N\) is odd then \(O(N) \cong \ZZ_2 \times SO(N)\).
\end{mdthm}

\begin{proof}
    Consider the map \(\phi : O(N)\ \to \ZZ_2 \times SO(N)\) given by
    \[\phi(A)= \left( \det(A), \frac{A}{\det(A)} \right)\]
    for some \(A \in O(N)\). \\
    We prove the map is well-defined.
    \begin{enumerate}
        \item \(\det(A) \in \ZZ_2\) as 
        \[\begin{aligned}
            A^{\top}A &= \mathbb{I} \\
            \then \det(A^{\top}A) &=1 \\
            \then (\det(A))^2 &=1.
        \end{aligned}\]
        As \(\det(A) \in \RR\) we must have \(\det(A) \in \{-1,1\} \cong \ZZ_2\).
        \item \(\frac{A}{\det(A)} \in O(N)\) as 
        \[\begin{aligned}
            \left( \frac{A}{\det(A)} \right)^{\top} \left(\frac{A}{\det(A)}\right) &= \frac{A^{\top}A}{(\det(A))^2} \\
            &= A^{\top}A \\
            &= \mathbb{I}.
        \end{aligned}\]
        \item \(\frac{A}{\det(A)} \in SO(N)\) as 
        \[\begin{aligned}
            \det\left( \frac{A}{\det(A)} \right) &= \det(\lambda A) \\
            &= \lambda^N \det(A),
        \end{aligned}\]
        where \(\lambda = \frac{1}{\det(A)}\). Therefore, we have 
        \[\begin{aligned}
            \det\left( \frac{A}{\det(A)} \right) &= \left( \frac{1}{\det(A)} \right)^N \det(A) \\
            &= \frac{1}{(\det(A))^{N-1}} \\
            &= 1
        \end{aligned}\]
        since \(N\) is odd.
    \end{enumerate}
    We prove surjective: \\
    We can always find a matrix \(A \in O(N)\) such that \(\phi(A)=(a,B)\). Indeed, just take \(A =aB\). This is in \(O(N)\) because \(A^{\top}=(aB)^{\top}=aB^{\top}\) and \(A\inv=(aB)\inv=a\inv B\inv =aB^{\top}\) so, they are both equal (here we have used that \(a\inv = a\) for \(a \in \ZZ_2\) and \(B\inv=B^{\top}\) for \(B \in SO(N)\)). Also, \(A\) has determinant \(\det(A)=\det(aB)=a^N\det(B)=a\det(B)\) (since \(N\) is odd and \(a = \pm 1\)), so that \(\det(A)=a\) (since \(B \in SO(N)\)). In conclusion, \(\phi(A)=(a,A/a)=(a,B)\), which shows surjectivity.
\end{proof}

\subsection{Groups of infinite order}

For finite group the order of the group is finite and is defined to be the number of elements in the group.

\begin{definition}
    The \textbf{real dimension} of a matrix is the number of real numbers needed to specify a particular element in the matrix group.
\end{definition}

\begin{mdremark}
    This idea can also be extended to define the \textbf{complex dimension}.
\end{mdremark}

\begin{mdthm}
    \(\dim(G/H)=\dim(G)-\dim(H)\).
\end{mdthm}

\begin{theorem}
    The real dimension of \(\GL(N,\RR)\) is \(N^2\).
\end{theorem}

\begin{proof}
    Each element \(A \in \GL(N,\RR)\) has \(N^2\) real entries subject to the condition that \(\det(A) \neq 0\).
\end{proof}

\begin{theorem}
    The real dimension of \(\SL(N,\RR)\) is \(N^2-1\).
\end{theorem}

\begin{proof}
    Recall that \(\frac{\GL(N,\RR)}{\SL(N,\RR)} \cong \RR^*\) therefore,
    \[\begin{aligned}
        \dim[\GL(N,\RR)]-\dim[\SL(N,\RR)]&=\dim(\RR^*) \\
        N^2 - \dim[\SL(N,\RR)] &=1.
    \end{aligned}\]
    This implies that \(\dim[\SL(N,\RR)]=N^2-1\).
\end{proof}

\begin{theorem}
    The real dimension of \(O(N)\) is \(\frac{N(N-1)}{2}\).
\end{theorem}

\begin{proof}
    If \(A \in O(N)\) then \(A \in \GL(N,\RR)\) so, \(A\) has at most \(N^2\) entries. As \(A \in O(N)\) then \(A^{\top}A = \mathbb{I}\) but also \(\left( A^{\top}A \right)^{\top}=\mathbb{I}^{\top}=\mathbb{I}\). Therefore, we must only consider either upper triangular or lower triangular matrices; each of these matrices has \(1+2+\ldots+N=\frac{N(N+1)}{2}\) entries i.e. constraints. The dimension of \(O(N)\) is then 
    \[N^2-\frac{N(N+1)}{2}=\frac{N(N-1)}{2}.\]
\end{proof}

\begin{mdthm}
    The real dimension of \(SO(N)\) is \(\frac{N(N-1)}{2}\).
\end{mdthm}

\begin{proof}
    Recall that \(\frac{O(N)}{SO(N)} \cong \ZZ_2\) therefore,
    \[\begin{aligned}
        \dim[O(N)]-\dim[SO(N)]&=\dim(\ZZ_2) \\
        \frac{N(N-1)}{2} - \dim[SO(N)] &=0.
    \end{aligned}\]
    Which implies that \(\dim[SO(N)]=\dim[O(N)]=\frac{N(N-1)}{2}\).
\end{proof}

\pagebreak

\section{The structure of matrix groups}

\subsection{\texorpdfstring{\(SO(2)\)}{TEXT}}

Let \(A \in SO(2)\) be given by \(A=\begin{pmatrix} a & b \\ c & d  \end{pmatrix}\) for \(a,b,c,d \in \RR\) such that \(A^{\top}A =\mathbb{I}\) and \(\det(A)=1\). Now \(A^{\top}=A\inv\) we have that
\[\begin{aligned}
    \begin{pmatrix} a & b \\ c & d  \end{pmatrix} &= \frac{1}{\det(A)}\begin{pmatrix} d & -b \\ -c & a  \end{pmatrix} \\
    &= \begin{pmatrix} d & -b \\ -c & a  \end{pmatrix}.
\end{aligned}\]
Therefore, \(a=d\) and \(b=-c\). We can rewrite \(A\) as
\[A= \begin{pmatrix} a & b \\ -b & a  \end{pmatrix} \text{ with } \det(A)=a^2+b^2=1.\]
A consistent choice is \(a=\cos\theta\) and \(b=-\sin\theta\), such that
\[A(\theta)=\begin{pmatrix} \cos\theta & -\sin\theta \\ \sin\theta & \cos\theta  \end{pmatrix}  \text{ for } \theta \in [0,2\pi).\]
By this definition of \(A(\theta) \in SO(2)\), we conclude that \(SO(2)\) has real dimension \(1\), as only one parameter is needed to construct a matrix in \(SO(2)\). Furthermore, we notice that \(SO(2)\) is abelian as by explicit matrix multiplication we have
\[A(\theta)A(\theta')=A(\theta+\theta') \text{ and } A(\theta)\inv =A(-\theta).\]

\begin{mdthm}
    \(SO(2) \cong S^1 \cong U(1)\).
\end{mdthm}

\begin{proof}
    Consider \(\phi : SO(2) \to S^1\) given by 
    \[\phi(A(\theta))= e^{i\theta} \in S^1.\]
    We can also rewrite \(U(1)= \{e^{i\theta} : \theta \in [0,2\pi)\}\).
\end{proof}

\begin{definition}
    \(S^N\) is the \(N\)-dimensional sphere of unit radius i.e.
    \[S^N=\left\{\bm{x} = \begin{pmatrix} x_1 \\x_2 \\ \vdots \\x_{N+1} \end{pmatrix} \in \RR^{N+1} : \| \bm{x} \|=1\right\}\]
\end{definition}

\pagebreak

\begin{mdnote}
    Therefore, \(S^1 = \{\bm{x} \in \RR^2 : x^2+y^2=1\}\), which are the set of all points which make up the unit circle. Similarly, 
    \[S^2= \left\{ \begin{pmatrix} x\\y\\z\end{pmatrix} \in \RR^3 : x^2+y^2+z^2 =1\right\}.\]
\end{mdnote}

\subsection{\texorpdfstring{\(SU(2)\)}{TEXT}}

Let \(A \in SU(2)\) hence \(A^{\dagger}=A\inv\) and \(\det(A)=1\). By applying these defining relations to
\[A=\begin{pmatrix}
   \alpha  & \beta \\ \gamma & \delta
\end{pmatrix}\]
for \(\alpha, \beta, \gamma, \delta \in \CC\), we find that 
\[\begin{aligned}
    A^{\dagger}=\begin{pmatrix}
        \alpha^*  & \gamma^*  \\ \beta^*  & \delta^*
    \end{pmatrix} &= \frac{1}{\det(A)} \begin{pmatrix}
        \delta & -\beta \\ -\gamma & \alpha
    \end{pmatrix} \\
    &= A\inv \\
    &=\begin{pmatrix}
        \delta & -\beta \\ -\gamma & \alpha
    \end{pmatrix}
\end{aligned}.\]
By comparing the entries we find that \(\alpha^*=\delta\) and \(\gamma^*=-\beta\). Therefore, any matrix \(A \in SU(2)\) takes the form
\[A=\begin{pmatrix}
    \alpha & \beta \\ -\beta^* & \alpha^*
\end{pmatrix} \text{ with } \abs{\alpha}^2+\abs{\beta}^2=1.\]
By writing the complex numbers as
\[\begin{aligned}
    \alpha &= a+ib_z &\quad \alpha^*&=a-ib_z \\
    \beta&=b_y+ib_x &\quad -\beta^*&=-b_y+ib_x,
\end{aligned}\]
we find that
\[\begin{aligned}
    A &=\begin{pmatrix}
        a+ib_z & b_y+ib_x \\
        -b_y+ib_x, & a-ib_z
    \end{pmatrix} \\
    &=a\begin{pmatrix} 1 & 0 \\ 0 & 1\end{pmatrix} +ib_x \begin{pmatrix} 0 & 1 \\ 1 &0 \end{pmatrix}+ib_y \begin{pmatrix} 0 & -i \\i & 0 \end{pmatrix}+ ib_z\begin{pmatrix} 1 & 0 \\ 0 & -1 \end{pmatrix}\\
        &= a\mathbb{I}+ib_x \sigma_x +ib_y \sigma_y +ib_z \sigma_z \\
        &= a \mathbb{I}+ i \bm{b}\cdot \bm{\sigma}.
\end{aligned}\]
Where \(\bm{b}=\begin{pmatrix} b_x \\ b_y \\ b_z \end{pmatrix}\) and \(\bm{\sigma} = \begin{pmatrix} \sigma_x \\ \sigma_y \\ \sigma_z \end{pmatrix}\), so that \(\bm{b}\cdot \bm{\sigma} = b_x\sigma_x +b_y\sigma_y+b_z\sigma_z\) and also where
\[\sigma_x \equiv \begin{pmatrix} 0 & 1 \\ 1 &0 \end{pmatrix} \quad \sigma_y \equiv \begin{pmatrix} 0 & -i \\i & 0 \end{pmatrix} \quad \sigma_z \equiv \begin{pmatrix} 1 & 0 \\ 0 & -1 \end{pmatrix}.\]

\begin{definition}
    The matrices \(\sigma_x,\sigma_y\) and \(\sigma_z\) are called the \textbf{Pauli matrices}.
\end{definition}

\begin{theorem}
    Each \(A\in SU(2)\) is given by a point on \(S^3 \subset \RR^4\).
\end{theorem}

\begin{mdthm}
    The properties of Pauli matrices; this is for any \(i \in \{x,y,z\}\).
    \begin{itemize}
        \item Traceless, \(\Tr(\sigma_i)=0\).
        \item Self-adjoint, \(\sigma_i^{\dagger}=\sigma_i\)
        \item \(\sigma_i^2 = \mathbb{I}\).
        \item \(\sigma_i \sigma_j = -\sigma_j\sigma_i\) for \(i \neq j\).
        \item \((i\sigma_x)^2=(i\sigma_y)^2=(i\sigma_z)^2=-\mathbb{I}\).
    \end{itemize}
\end{mdthm}

\subsubsection{Quaternions}

\begin{definition}
    The division algebra \(\mathbb{H}\) of \textbf{quaternions} is the non-commutative algebra of all real linear combinations of \(i,j,k\) and \(\mathbb{I}\) i.e. \(z\in \mathbb{H}\) then \(z=a+ib_z+jb_y+kb_x\) where \(a,b_x,b_y,b_z \in \RR\) with the relations \(i^2=j^2=k^2 =-1\), \(ijk=-1\) and all other cyclic combinations.
\end{definition}

\begin{mdremark}
    A \textbf{division algebra} is an algebra over which division, except division by zero, is always possible.
\end{mdremark}

\begin{mdnote}
    The cyclic combinations of \(i,j,k\) can be obtained from the cyclic diagram below:
    \begin{center}
        \include{./Resources/Q8 diagrams.tex}
    \end{center}
    For example, \(ij =k\) as shown by the clockwise orientation of the arrows. Following the anti-clockwise orientation leads to a change in sign, for example \(ji=-k\).
\end{mdnote}

\begin{definition}
    A \textbf{quaternionic conjugate} is defined by:
    \[\overline{z}=a-ib_z-jb_y-kb_x.\]
    As matrices the conjugate is defined as
    \[\overline{z}=z^{\dagger}.\]
    So, \(\abs{z}=\sqrt{z\overline{z}}\).
\end{definition}

\subsubsection{Quaternion group}

\begin{definition}
    The \textbf{Quaternion group} is the group
    \[Q_8 = \{1,i,j,k,-i,-j,-k\},\]
    where \(i^2=j^2=k^2=-1\) and \(ijk=-1\).
\end{definition}

\begin{mdnote}
    \(Q_8\) is equivalent to \(\langle a,b \rangle\) with
    \[\begin{aligned}
        a^4&=e \\
        b^2 &= a^2 \\
        ab &= -ba.
    \end{aligned}\]
\end{mdnote}

\subsection{Invariant inner product: \texorpdfstring{\(O(N)\)}{TEXT} and \texorpdfstring{\(U(N)\)}{TEXT}}

\subsubsection{The inner product}

\begin{definition}
    Let \(V\) be a finite-dimensional vector space over \(\CC\). A map \(V \times V \to \CC\) given by \((\bm{x},\bm{y}) \mapsto \langle \bm{x}, \bm{y} \rangle\) is an \textbf{inner product} if it satisfies:
    \begin{itemize}
        \item \(\langle \bm{x}, \bm{y} \rangle^* =\langle \bm{y}, \bm{x} \rangle\).
        \item \(\langle \bm{x}, a\bm{y}+b\bm{z} \rangle = a \langle \bm{x}, \bm{y} \rangle +b\langle \bm{x}, \bm{z} \rangle\).
        \item \(\langle \bm{x}, \bm{x} \rangle \geq 0\) and if \(\langle \bm{x}, \bm{x} \rangle =0\) then \(\bm{x}=0\).
    \end{itemize}
\end{definition}

\begin{corollary}
    The first and second property imply
    \[\begin{aligned}
        \langle a\bm{y}+b\bm{z}, \bm{x} \rangle &= \langle \bm{x}, a\bm{y}+b\bm{z} \rangle^* \\
        &= \left( a\langle \bm{x},\bm{y} \rangle +b \langle \bm{x}, \bm{z} \rangle \right)^* \\
        &= a^* \langle \bm{y}, \bm{x} \rangle +b^* \langle \bm{z},\bm{x} \rangle.
    \end{aligned}\]
\end{corollary}

\begin{definition}
    An inner product equipped with all the properties above is called \textbf{sesquilinear}.
\end{definition}

From now on the only inner product we will use is given by
\[\langle \bm{x}, \bm{y} \rangle = \sum_i x_i^* y_i =\bm{x}^{\dagger}\bm{y}.\]

\begin{mdcor}
    For a matrix \(A\) we have that 
    \[\langle \bm{x}, A\bm{y} \rangle = \langle A^{\dagger}\bm{x}, \bm{y} \rangle.\]
\end{mdcor}

\subsubsection{\texorpdfstring{\(O(N)\)}{TEXT} and \texorpdfstring{\(U(N)\)}{TEXT} preserve vector length}

\begin{definition}
    The \textbf{norm} of a vector is defined by 
    \[\| \bm{x} \|= \sqrt{\langle \bm{x}, \bm{x} \rangle}.\]
\end{definition}

\begin{mdnote}
    On a Euclidean vector space it is common place to use \(\| \bm{x} \|\) to denote the length of a vector.
\end{mdnote}

\begin{mdthm}
    A real, linear transformation \(A\) of \(\RR^N\) is such that \(\|A\bm{x}\| = \|\bm{x} \|\) for all \(\bm{x} \in \RR^N\) if and only if \(A \in O(N)\).
\end{mdthm}

\begin{mdnote}
    The theorem implies that orthogonal transformation preserve vector length.
\end{mdnote}

\begin{mdthm}
    A complex, linear transformation \(A\) of \(\CC^N\) preserves the norm, \(\| A\bm{x} \|=\| \bm{x} \|\) for all \(\bm{x} \in \CC^N\) if and only if \(A\in U(N)\).
\end{mdthm}

\subsection{\texorpdfstring{\(SO(3)\)}{TEXT}}

\begin{mdthm}
    If \(\bm{x} \in \CC^n\) is an eigenvector of \(A \in O(N)\) with eigenvalue \(\lambda\), then \(\abs{\lambda}=1\).
\end{mdthm}

\begin{theorem}
    If \(A \in SO(3)\) then there exists a vector \(\bm{n} \in \RR^3\) such that \(A\bm{n}=\bm{n}\).
\end{theorem}

\begin{proof}
    Consider \(A\bm{x}=\lambda\bm{x}\) where \(\bm{x} \in \CC^3\), as \(A\in SO(3) \subset O(3)\) then \(\abs{\lambda}=1\) by the previous theorem. So, if \((A-\lambda \mathbbm{1})\bm{x}=\bm{0} \) is non-trivial then \(P(\lambda)=\det(A-\lambda \mathbbm{1})=0\).
    \[P(\lambda)=\det\begin{pmatrix}
        A_{11} - \lambda & A_{12} & A_{13} \\
        A_{21} & A_{22}-\lambda & A_{23} \\
        A_{31} & A_{32} & A_{33} -\lambda
    \end{pmatrix},\]
    so \(P(\lambda)\) is a cubic equation in \(\lambda\). \\
    Denote the roots of \(P(\lambda) =0\) with \(\lambda_1,\lambda_2,\lambda_3 \in \CC\). Hence, 
    \[P(\lambda) = -(\lambda-\lambda_3)(\lambda-\lambda_2)(\lambda-\lambda_1).\]
    As \(P(0)=1\) then \(P(0)=\lambda_1\lambda_2\lambda_3 =1\). We need to show that at least one of the \(\lambda_1,\lambda_2\) and \(\lambda_3\) is equal to \(+1\).
    There are two possibilities:
    \begin{enumerate}
        \item If \(\lambda_1\) is complex but not real, i.e. \(\Img(\lambda_1) \neq 0\) then as \(\abs{\lambda_1}=1\) we can write 
        \[\lambda_1=e^{i\alpha}\]
        for \(\alpha \in (0,2\pi) \backslash \{\pi\}\) (so that \(\Img(\lambda_1)\neq 0\)).  \\ 
        In this case we have,
        \[A\bm{x}=\lambda_1 \bm{x}= e^{i\alpha} \bm{x}.\]
        Since \(A\in SO(3)\) and \(\lambda_1 \in \CC\) we deduce that \(\bm{x}\) is a complex vector. Therefore,
        \[\begin{aligned}
            (A\bm{x})^* &= (\lambda_1 \bm{x})^* \\
            \then A\bm{x}^* =\lambda_1^* \bm{x}^* &= e^{-i\alpha} \bm{x}^*.
        \end{aligned}\]
        Hence, we have found a second complex eigenvalue, \(\lambda_2 = e^{-i\alpha}\). Then as 
        \[\begin{aligned}
            \lambda_1\lambda_2\lambda_3 &=1 \\
            &=\left( e^{i\alpha} \right) \left( e^{-i\alpha} \right) \lambda_3 \\
            &= \lambda_3.
        \end{aligned}\]
        So, \(\lambda_3 =1\).
    \item All three eigenvalues are real: as \(\abs{\lambda_1}=\abs{\lambda_2}=\abs{\lambda_3}=1\) then 
    \[\lambda_1,\lambda_2,\lambda_3 \in \{-1,1\}.\]
    As \(\lambda_1\lambda_2\lambda_3 =1\) we can rule out the possibility that \(\lambda_1=\lambda_2=\lambda_3=-1\) which implies that at least one of \(\lambda_1,\lambda_2,\lambda_3\) must equal to \(+1\).
    \end{enumerate}
    Finally, we show that if \(\bm{x}\) is the \textbf{complex eigenvector} with eigenvalue \(1\) then there exists a \textbf{real eigenvector} \(\bm{n}\) with eigenvalue \(1\). We have \(A\bm{x}=\bm{x}\) if \(\bm{x} \neq \bm{x}^*\) then we also have \(A\bm{x}^*=\bm{x}^*\) so,
    \[A(\bm{x}+\bm{x}^*) = (\bm{x}+\bm{x}^*)\]
    then \(\bm{n}=\bm{x}+\bm{x}^*\), which is real and has eigenvalue \(1\).
\end{proof}

\begin{mdthm}
    Each matrix \(A \in SO(3)\) encodes a rotation about an axis spanned by \(\hat{\bm{n}} \in \RR^3\).
\end{mdthm}

\begin{mdnote}
    \[SO(3) =\{\text{all rotations about all possible axes in } \RR^3\}.\]
\end{mdnote}

Choose a basis of \(\RR^3\) such that \(\hat{\bm{n}} =\frac{\bm{n}}{\abs{\bm{n}}}=\hat{\bm{e}}_x\) i.e. the unit vector along the \(x\)-axis, where \(A\bm{n}=\bm{n}\) for \(A \in SO(3)\). That is 
\[\begin{pmatrix}
     A_{11} & A_{12} & A_{13} \\
    A_{21} & A_{22} & A_{23} \\
    A_{31} & A_{31} & A_{33} 
\end{pmatrix} 
\begin{pmatrix}
    1 \\ 0\\ 0
\end{pmatrix}
= 
\begin{pmatrix}
    1 \\ 0\\ 0
\end{pmatrix},\]
which implies 
\[\begin{aligned}
    A_{11} &=1 \\
    A_{21} &= 0 \\
    A_{31} &= 0.
\end{aligned}\]
Further, imposing the properties that \(A\) must satisfy to be in \(SO(3)\), i.e. \(A^{\top}A =\mathbb{I}\) then we have that \(A\) is of the form 
\[A = \begin{pmatrix}
    1 & 0 &0 \\
    0& A_{22} & A_{23} \\
    0 & A_{32} & A_{33} 
\end{pmatrix}.\]
Denote the \(2 \times 2\) bottom right matrix by \(a = \begin{pmatrix} A_{22} & A_{23} \\ A_{32} & A_{33}  \end{pmatrix}\).
Then \(\det(A)=1 \then \det(a)=1\) and \(A^{\top}A =\mathbb{I} \then a^{\top} a = \mathbb{I}\) therefore, \(a \in SO(2)\).
We previously saw the structure matrices in \(SO(2)\): rotation matrices,
\[a=\begin{pmatrix} 
    \cos\theta & -\sin\theta \\ 
    \sin\theta & \cos\theta  
\end{pmatrix}.\]
Finally, we have that 
\[A = \begin{pmatrix}
    1 & 0 & 0 \\
    0 & \cos\theta & -\sin\theta \\
    0 & \sin\theta & \cos\theta
\end{pmatrix},\]
hence \(A\) is a rotation by \(\theta\) about the axis spanned by \(\hat{\bm{n}}\).


\subsubsection{Geometry of \texorpdfstring{\(SO(3)\)}{TEXT}}

Each matrix \(A \in SO(3)\) encodes a rotation about an axis spanned by \(\hat{\bm{n}} \in \RR^3\). Let \(\abs{\bm{n}}=\theta\) then all the rotations in \(\RR^3\) (the elements of \(SO(3)\)) are represented by a ball (of vectors) of radius \(\pi\) where diametrically opposing points of the ball encode the same rotation matrix in \(SO(3)\).


\subsection{Relating \texorpdfstring{\(SU(2)\)}{TEXT} to \texorpdfstring{\(SO(3)\)}{TEXT}}

\begin{proposition}
    There exists a linear bijective map \(\Theta :\Sigma \to \RR^3\) where \(\Sigma\) is the real, linear space of self-adjoint, traceless \(2\times 2\) matrices.
\end{proposition}

\begin{mdnote}
    The Pauli matrices form a basis for \(\Sigma\).
\end{mdnote}

\begin{proof}
    Since the Pauli matrices \(\sigma_i\) are a basis for \(\Sigma\), a general \(2 \times 2\) matrix can be written in the form \(A=a\mathbb{I}+\bm{b} \cdot \bm{\sigma}\) for \(a,b_x,b_y,b_z \in \CC\). The condition of tracelessness imposes \(a=0\) and the condition of self-adjointness imposes \(\bm{b}^*=\bm{b}\) hence \(\bm{b}\in \RR^3.\) The space \(\Sigma\) is the space of real linear combinations \(A=\bm{b} \cdot \bm{\sigma}\). Consider the map \(\Theta\) given by
    \[\Theta(\bm{b} \cdot \bm{\sigma})=\bm{b},\]
    this map is both linear and bijective.
\end{proof}

\begin{mdthm}
    \(SU(2) / \ZZ_2 \cong SO(3)\).
\end{mdthm}

\begin{proof}
    Use homomorphism theorem on the surjective map \(\varphi:SU(2) \to SO(3)\) with kernel \(\ker \varphi \cong \ZZ_2\).
\end{proof}

\pagebreak

\section{The Semi-Direct Product}

\begin{definition}
    A group \(J\) is a \textbf{semi-direct product} of a subgroup \(H\) by a subgroup \(G\) if the following conditions are satisfied:
    \begin{enumerate}
        \item[(i)] \(J=HG\),
        \item[(ii)] \(H \triangleleft J\),
        \item[(iii)] \(H \cap G =\{e\}\) where \(e\) is the identity element in \(J\).
    \end{enumerate}
    The semi-direct product is denoted by \(J=G \ltimes H\) \\(or \(J=G \ltimes_{\psi} H\) where \(\psi: G \to \aut(H)\) given by \(\psi(g)=\phi_g\)).
\end{definition}

\begin{mdnote}
    Elements in \(J\), \(j\in J\), take the form of \(j=hg\) for \(h \in H\) and \(g \in G\).
\end{mdnote}

\begin{mdremark}
    On the semi-direct product.
    \begin{enumerate}
        \item The \textbf{direct product} is a special case of the semi-direct product where both \(G\) and \(H\) are normal subgroups of \(J\).
        \item By the notation \(J = G \ltimes H\) we mean that \(J \triangleright H\) i.e. \(H\) is a normal subgroup of \(J\).
        \item Inner product notation: elements of the semi-direct product group are expressed as \(hg\).
        \item Outer product notation: elements of the semi-direct product group are expressed as \((g,k)\).
    \end{enumerate}
\end{mdremark}

\begin{definition}
    The semi-direct product \(G \ltimes H\) is the group whose elements are those of the set \(G \times H\) and whose multiplication law is 
    \[(g,h)(g',h')=(gg',h\phi_g(h'))\]
    where \(\phi_g \in \aut(H)\).
\end{definition}

\begin{theorem}
    The multiplication law in the previous equation gives rise to a group structure on the set \(G \times H\).
\end{theorem}

\begin{proof}
    We check the axioms of a group:
    \begin{itemize}
        \item \textbf{Closure}: We have that \(gg' \in G\) by closure of \(G\) and \(\phi_g(h) \in H\) since \(\phi_g\) is an \\ automorphism of \(H\), so that \(h\phi_g(h') \in H\) by closure of \(H\).
        \item \textbf{Associativity}: 
        \[\begin{aligned}
            (g_1,h_1)((g_2,h_2)(g_3,h_3)) &= (g_1,h_1)(g_2g_3,h_2\phi_{g_2}(h_3)) \\
            &= (g_1g_2g_3, h_1\phi_{g_1}(h_2\phi_{g_2}(h_3))) \\
            &= (g_1g_2g_3, h_1\phi_{g_1}(h_2)\phi_{g_1}(\phi_{g_2}(h_3))) \\
            &= (g_1g_2g_3, h_1\phi_{g_1}(h_2)\phi_{g_1g_2}(h_3)) \\
            \vspace{5em}
            ((g_1,h_1)(g_2,h_2))(g_3,h_3) &= (g_1g_2,h_1\phi_{g_1}(h_2)) (g_3,h_3) \\
            &= (g_1g_2g_3 , h_1\phi_{g_1}(h_2)\phi_{g_1g_2}(h_3)).
        \end{aligned}\]
        \item \textbf{Identity}: The identity element is \((e_G, e_H)\) as:
        \[\begin{aligned}
            (e_G, e_H)(g,h) &= (e_G g, e_H \phi_{e_G}(h)) \\
                            &= (g,h).
        \end{aligned}\]
        The identity map of \(\psi : G \to \aut(H)\) given by \(\psi(g)=\phi_g\) is a homomorphism therefore, \(\psi_{e_G}=\phi_{e_G} = \id \in \aut(H)\).
        \[\begin{aligned}
            (g,h)(e_G, e_H) &= (ge_G, h\phi_g(e_h)) \\
                            &= (g,h).
        \end{aligned}\]
        Since, \(\phi_g : H \to H\) is a homomorphism and so \(\phi_g(e_H)=e_H\).
        \item \textbf{Inverse}: The inverse to \((g,h)\) is \((g\inv, \phi_{g\inv}(h\inv))\) as
        \[\begin{aligned}
            (g,h)(g\inv,\phi_{g\inv}(h\inv)) &= (gg\inv,h\phi_g(\phi_g\inv)(h\inv)) \\
            &=(gg\inv, h\phi_{gg\inv}(h\inv)) \\
            &= (e_G, h\phi_{e_G}(h\inv)) \\
            &= (e_G, e_H) \\
            \vspace{5em}
            (g\inv,\phi_{g\inv}(h\inv))(g,h) &= (g\inv g, \phi_{g\inv}(h\inv)\phi_{g\inv}(h)) \\
            &= (e_G, \phi_{g\inv} (h\inv h)) \\
            &= (e_G,e_H).
        \end{aligned}\]
    \end{itemize}
\end{proof}

\subsection{\texorpdfstring{\(O(N) \cong \ZZ_2 \ltimes_{\psi} SO(N)\)}{TEXT}}

\begin{mdthm}
    \[O(N) \cong \ZZ_2 \ltimes_{\psi} SO(N)\]
    where the map \(\psi:\ZZ_2 \to \aut(SO(N))\) is defined by \(\psi(s)=\phi_s\) with \(\phi_s \in \aut(SO(N))\) given by \(\phi_s(g)=sgs\inv = sgs\) for all \(s \in \ZZ_2\) and \(g\in SO(N)\).
\end{mdthm}

\begin{proof}
    We express \(\ZZ_2\) as \(N \times N\) matrices such that \(\ZZ_2=\{\mathbb{I},R\}\) where \(\mathbb{I}\) is the \(N \times N\) identity matrix and \(R\) is the (reflection) matrix 
    \[R = \begin{pmatrix}
        -1 & 0 & 0 & \ldots & 0 \\
        0 & 1 & 0 & \ldots & 0 \\
        0 & 0 & 1 & \ldots & 0 \\
        0 & 0 & 0 & \ddots & \vdots \\
        0 & 0 & 0 & \ldots & 1 \\
    \end{pmatrix}.\]
    Note \(R^2=\mathbb{I}\) and that \(\ZZ_2=\{\mathbb{I}, R\}\) is a subgroup of \(O(N)\).
    We show that \(\psi : \ZZ_2 \to \aut(SO(N))\) given by \(\psi(s) = \phi_s\) for \(s \in \{\mathbb{I},R\} \cong \ZZ_2\) is well-defined. So, we need \(\psi \in \aut(SO(N))\) i.e. \(\psi\) is 
    \begin{enumerate}
        \item First we show that \(\psi : \ZZ_2 \to \aut(SO(N))\) given by \(\psi(s) = \phi_s\) for \(s \in \{\mathbb{I},R\}\cong \ZZ_2\) is a well-defined homomorphism i.e. \(\psi \in \aut(SO(N))\).
        \begin{itemize}
            \item \(\phi_s\) is well-defined i.e. \(\phi_s(A) \in SO(N)\): \\
                    Consider \(\phi_s(A)=sAs\) as \(A \in SO(N)\) and \(s\in O(N)\) then \(sAs \in O(N)\). Next we need \(\det(sAs)=1\). We have,
                    \[\begin{aligned}
                        \det(sAs) &= \det(s)\det(A)\det(s) \\
                                &= (\det(s))^2 \\
                                &=1.
                    \end{aligned}\]
                    Therefore, \(\det(s) \in \{-1,1\}\) which implies that \(sAs \in SO(N)\).
            \item \(\phi_s\) is a homomorphism:
                    \[\begin{aligned}
                        \phi_s(AB) &= sABs \\
                                    &= sAssBs \\
                                    &= \phi_s(A)\phi_s(B).
                    \end{aligned}\]
            \item \(\phi_s\) is a bijection:
                \begin{itemize}
                    \item Injective: If \(\phi_s(A)=\phi_(B)\) then \(sAs=sBs\) which implies that \(A=B\).
                    \item Surjective: For any \(B \in SO(N)\) we have 
                    \[\phi_s(sBs)=s^2Bs^2 =B,\]
                    where \(sBs \in SO(N)\).
                \end{itemize}
        \end{itemize}
        Therefore, \(\phi_s \in \aut(SO(N))\).
        \item Next, we construct an isomorphism, \(\Phi\), that maps \(O(N)\) to \(\ZZ_2 \ltimes_{\psi} SO(N)\). \\
        We will need a map 
        \[\begin{aligned}
            \omega: O(N) &\to \{\mathbb{I},R\} \cong \ZZ_2 \quad \text{ given by} \\
            \omega(A) &=\begin{cases}
                \mathbb{I} &\text{ if } \det(A)=1 \\
                R &\text{ if } \det(A)=-1
            \end{cases}
        \end{aligned}\]
        This map is a homomorphism as for \(A,B \in O(N)\) we have 
        \[\begin{aligned}
            \omega(AB) &= \begin{cases}
            \mathbb{I}  &\text{ if } \det(AB)=1 \then \det(A)=\det(B)\\
            R &\text{ if } \det(A) \neq \det(B)
            \end{cases} \\
            \omega(A)\omega(B) &=\begin{cases}
                \mathbb{I} &\text{ if } \det(A)=\det(B) \\
                R &\text{ if } \det(A) \neq \det(B)
            \end{cases} \\
        \end{aligned}\]
        Hence,
        \[\omega(AB)=\omega(A)\omega(B).\]
        Using \(\omega\) we define \(\Phi :O(N) \to \ZZ_2 \ltimes_{\psi} SO(N)\) given by
        \[\Phi(A) =(\omega(A),A\omega(A)) \quad \forall A\in O(N).\]
        We check:
        \begin{itemize}
            \item \(\Phi\) is well-defined:
                    Notice \(\omega(A) \in \{\mathbb{I},R\} \cong \ZZ_2\) and \(A,\omega(A) \in O(N)\) so, \(A\omega(A) \in O(N)\). Furthermore, 
                    \[\begin{aligned}
                        \det(A\omega(A)) &= \det(A)\det(\omega(A)) \\
                        &= \det(A)\det(\det(A)) \\
                        &= (\det(A))^2 \\
                        &= 1.
                    \end{aligned}\]
                    Therefore, \(A\omega(A) \in SO(N)\).
            \item \(\Phi\) is a homomorphism:
                    \[\begin{aligned}
                        \Phi(A)\Phi(B) &=(\omega(A),A\omega(A))(\omega(B),B\omega(B)) \\
                        &= (\omega(A),\omega(B),A\omega(A)\phi_{\omega(A)}[B\omega(B)]) \\
                        &= (\omega(AB),A \omega(A)\omega(A)B\omega(B),\omega(A)) \\
                        &= (\omega(AB), AB \omega(A)\omega(B)) \\
                        &= (\omega(AB), AB\omega(AB)) \\
                        &= \Phi(AB).
                    \end{aligned}\]
                    We have used that \(\omega(A)\omega(A) = \mathbb{I}\) as \(\omega(A) \in \ZZ_2\) and that \(\omega(B)\omega(A)=\omega(A)\omega(B)\) since they are diagonal matrices in \(\{\mathbb{I},R\}\).
            \item \(\Phi\) is bijective:
                    \begin{itemize}
                        \item Injective: 
                        Suppose there exists \(A \neq B\) such that \(\Phi(A) = \Phi(B)\) then 
                        \[(\omega(A),A\omega(A))=(\omega(B),B\omega(B)),\]
                        which implies \(\omega(A)=\omega(B)\) and \(A\omega(A)=B\omega(B)\) thus, \(A=B\).
                        \item Surjective: Consider \((s,B) \in \ZZ_2 \ltimes_{\psi} SO(N)\) for \(s \in \ZZ_2\) and \(B \in SO(N)\) then,
                        \[\begin{aligned}
                            \Phi(Bs) &= (\omega(Bs),Bs\omega(Bs)) \\
                            &= (\omega(B)\omega(s),Bs\omega(B)\omega(s))\\
                            &= (s,Bs^2) \\
                            &= (s,B).
                        \end{aligned}\]
                        Where we have used the fact that \(B \in SO(N)\) so, \(\det(B)=1\) which implies \(\omega(B)=\mathbb{I}\), \(\omega(s)=s\) and \(s \in \ZZ_2\) so \(s^2 = \mathbb{I}\).
                    \end{itemize}
        \end{itemize}
    \end{enumerate}
\end{proof}

\begin{mdthm}
    We have that 
    \begin{itemize}
        \item The subset \(\{(e,h) : h \in H\} \subset G \times H\) is a subgroup of \(G \ltimes_{\psi} H\) where 
        \[\{(e,h) : h \in H\} \cong H.\]
        Since, the subgroup is isomorphic to \(H\) we have that it is also normal in \(G \ltimes_{\psi} H\).
        \item The subset \(\{(g,e) : g\in G\}\) is a subgroup of \(G \ltimes_{\psi} H\) where 
        \[\{(g,e) : g \in G\} \cong G.\]
    \end{itemize}
\end{mdthm}

\begin{proof}
    We prove each statement in turn.
    \begin{itemize}
        \item We show the subset is indeed a subgroup of \(G \ltimes H\).
        \begin{itemize}
            \item Identity: \((e,e)\).
            \item Closure: \((e,h)(e,h')=(e,h\phi_e(h'))=(e,hh')\).
            \item Inverse: \((e,h)\inv = (e,h\inv)\).
        \end{itemize}
        It is indeed isomorphic to \(H\) with \((e,h) \mapsto h\). Furthermore, it is normal as 
        \[(g,h)\inv(e,h')(g,h)=(g\inv,\phi_{g\inv}(h\inv h'))(g,h)=(e,\phi_{g\inv}(h\inv h'h)).\]
        \item We prove only closure as the rest is similar to above. Closure: \((g,e)(g,e')=(gg',\phi_g (e))=(gg',e)\).
    \end{itemize}
\end{proof}

\begin{mdthm}
    The left cosets of \(G \ltimes H\) with respect to the normal subgroup \(H\) are the subsets \(\{(g,h) : h \in H\}\). Furthermore,
    \[\frac{G \ltimes H}{H} \cong G.\]
\end{mdthm}

\begin{proof}
    We prove each statement in turn.
    \begin{itemize}
        \item The left cosets are given by \((g,h)(e,H)=(g,h\phi_g(H))=(g,hH)=(g,H)\).
        \item The isomorphism is given by \((g,H) \mapsto g\) which is trivially bijective. It is a homomorphism as \((g,H)(g',H)=(gg',H\phi_g(H))=(gg',H)\).
    \end{itemize}
\end{proof}

\pagebreak

\section{The Euclidean Group}

The Euclidean group is the group of all transformation of \(\RR^N\) that leave the distance
\[D(\bm{x},\bm{y})=\|\bm{x}-\bm{y}\| =\sqrt{\langle \bm{x}-\bm{y}, \bm{x}-\bm{y} \rangle}\]
between any two points \(\bm{x},\bm{y} \in \RR^N\) invariant. 

\begin{proposition}
    \((\RR^N,+)\) is an abelian group.
\end{proposition}

\begin{mdthm}
    The set of all translations of \(\RR^N\) forms a group which is isomorphic to \(\RR^N\).
\end{mdthm}

\begin{proof}
    Let \(T_{\bm{b}} : \RR^N \to \RR^N\) be given by \(T_{\bm{b}} =\bm{x}+\bm{b}\). So, \(T_{\bm{b}}\) is a translation by \(\bm{b} \in \RR^N\). We check the axioms of groups:
    \begin{itemize}
        \item Closure: \(T_{\bm{b}} \circ T_{\bm{c}} =T_{\bm{b}}(T_{\bm{c}}(\bm{x})) = T_{\bm{b}+\bm{c}}\) for all \(\bm{x} \in \RR^N\).
        \item Associativity: inherent from the composition of maps.
        \item Identity: Translation by \(\bm{0}\) is the identity i.e. \(T_{\bm{0}}\).
        \item Inverse: \(T_{-\bm{b}}\) is the inverse to \(T_{\bm{b}}\) as \(T_{-\bm{b}} \circ T_{\bm{b}} =T_{\bm{0}}= T_{\bm{b}}\circ T_{-\bm{b}}\).
    \end{itemize}
    Hence, the set of all translations is a group. \\
    Consider the map \(\phi(T_{\bm{b}})=\bm{b}\), a map from the translation group to \(\RR^N\). This is a homomorphism as
    \[\begin{aligned}
        \phi(T_{\bm{b}} \circ T_{\bm{c}}) &= \phi(T_{\bm{b+c}}) \\
        &= \bm{b}+\bm{c} \\
        &= \phi(\bm{b}) + \phi(\bm{c}).
    \end{aligned}\]
    It is also bijective by construction therefore, \(\RR^N\) under addition is isomorphic to the group of translations.
\end{proof}

\begin{mdthm}
    The set of transformation \(Q : \RR^N \to \RR^N\) such that \(D(Q(\bm{x}),Q(\bm{y})) =D(\bm{x},\bm{y})\) consists of combinations of translations and orthogonal transformations.
\end{mdthm}

\begin{mdnote}
    The theorem is saying that all length preserving transformations on \(\RR^N\) consists \textbf{only} of translations and orthogonal transformations.
\end{mdnote}

\begin{definition}
    The \textbf{Euclidean group} is
    \[E_N=O(N) \ltimes_{\psi} \RR^N\]
    where \(\psi : O(N) \to \aut(\RR^N)\) given by \(\psi(A)=\varphi_A\) is a homomorphism, with \(\varphi_A\) defined by 
    \[\varphi_A(\bm{b})=A\bm{b}\]
    where \(A \in O(N)\) and \(\bm{b} \in \RR^N\).
\end{definition}

\begin{proof}
    We need to check: the Euclidean group is well-defined, \(\psi\) being a homomorphism and \(\varphi\) is an automorphism.
    \begin{enumerate}
        \item \(\varphi\) is an automorphism. \\
        If \(\varphi_A\) is bijective then it has an inverse. Since \(A\inv \in O(N)\) we have that \(\varphi_{A\inv}\) is the inverse to the map \(\varphi_A\) as 
        \[\varphi_{A\inv} \circ \varphi_A(\bm{b}) = A\inv A \bm{b} =\bm{b}\]
        and \(\varphi_{A\inv} \circ \varphi_A(\bm{b})=\bm{b}\). \\
        \(\varphi_A\) is a homomorphism.
        \[\begin{aligned}
            \varphi_A(\bm{b}+\bm{x})&=A(\bm{b}+\bm{x}) \\
            &= A\bm{b}+A\bm{c} \\
            &= \varphi_A(\bm{b})+\varphi_A(\bm{c}).
        \end{aligned}\]
        \item \(\psi\) is a homomorphism.
        \[\begin{aligned}
            \psi(AB)(\bm{b}) &= \varphi_{AB}(\bm{b}) \\
            &= AB(\bm{b}) \\
            &= \varphi_A \circ \varphi_B (\bm{b}) \\
            &= (\psi(A)\psi(B))(\bm{b})
        \end{aligned}\]
        for all \(\bm{b} \in \RR^N\).
        \item The Euclidean group is well-defined. \\
        By checking the properties above we have shown that the Euclidean group is well-defined.
    \end{enumerate}
\end{proof}

\noindent A general element of the Euclidean group will consist of both an orthogonal \\ transformation and translation which we will denote as \((A,T_{\bm{b}})=T_{\bm{b}} \circ A\) (the LHS is in outer product notation and the RHS is in the inner product notation). 
\[(A,T_{\bm{b}})(\bm{x}) =T_{\bm{b}}(A(\bm{x})) = A\bm{x}+\bm{b}.\]

\begin{theorem}
    The multiplication rule of the semi-direct product of the Euclidean group is defined as
    \[(A,\bm{b})(A',\bm{b}')=(AA',\bm{b} \cdot (\phi_A(\bm{b}')))=(AA',A\bm{b}'+\bm{b}),\]
    where \(\phi_A = A\bm{b}\).
\end{theorem}

\begin{mdthm}
    The Euclidean group, \(E_N\), is the group generated by translations and orthogonal transformations of \(\RR^N\).
\end{mdthm}

\pagebreak

\section{G-sets, stabilisers and orbits}

\begin{definition}
    For a group \(G\) a \textbf{G-set} is a set \(X\) equipped with a rule assigning to each element \(g\in G\) and each element \(x\in X\) an element \(g\cdot x \in X\) satisfying:
    \begin{itemize}
    \item[(i)] \(e \cdot x =x\) for all \(x \in X\) where \(e \in G\) is the identity element of \(G\);
    \item[(ii)] \((g_1g_2)\cdot x =g_1\cdot(g_2 \cdot x)\) for all \(g_1,g_2 \in G\) and \(x\in X\).
    \end{itemize}
\end{definition}

\begin{definition}
    Given a G-set, \(X\), the \textbf{stabiliser}, \(G_x\), of \(x \in X\) is the set of elements \(g \in G\) such that \(g \cdot x =x\) i.e.
    \[G_x = \{ g \in G : g\cdot x =x\}.\]
\end{definition}

\begin{mdthm}
    \(G_x\) is a subgroup of \(G\).
\end{mdthm}

\begin{definition}
    When \(X=G\) and the G-set action is \(g \cdot x=gxg\inv\) then \(G_x\) is called the \textbf{centraliser} of an element \(x\) and denoted:
    \[\begin{aligned}
        C_{G}(x)&=\{g \in G : gxg\inv=x\} \\
        &=\{g\in G : gx=xg\}.
    \end{aligned}\]
\end{definition}

\begin{mdremark}
    The centre, \(Z(G)\), consists of all those elements in \(G\) which commute with \textbf{all} other of \(G\). Whereas, \(C_G(x)\) consists of all elements of \(G\) which commute with a \textbf{single element} \(x \in G\).
\end{mdremark}

\begin{mdexample}
    The centraliser of \(b \in D_3\):
    \[C_{D_3}(b)=\{g \in D_3 : gbg\inv =b\}.\]
    As the centraliser is a subgroup of \(D_3\) it has order \(1,2,3\) or \(6\). Since, \(D_3\) is not abelian then \(\abs{C_{D_3}} \neq 6\). The subgroup \(\langle b \rangle \subset C_{D_3}(b)\); since \(\langle b \rangle\) is a subgroup of \( C_{D_3}(b)\) by Lagrange's theorem \(\abs{\langle b \rangle}\) divides \(\abs{C_{D_3}(b)}\). So, as \(\abs{\langle b \rangle} =2\) then \(\abs{C_{D_3}(b)}=2\). Hence, \(C_{D_3}= \langle b \rangle =\{e,b\}\).
\end{mdexample}

\begin{definition}
    When \(X\) is the set of subgroups in \(G\) with action \(g \cdot H = gHg\inv\) where \(H \in X\) is a subgroup of \(G\) then the stabiliser, \(G_H\), is called the \textbf{normaliser}. It is denoted by
    \[N_G(H)=\{g \in G : gHg\inv=H\}.\]
\end{definition}

\begin{theorem}
    The normaliser, \(N_G(H)\) is a subgroup of \(G\) that always contains \(H\).
\end{theorem}

\begin{example}
    The normaliser of \(\langle b \rangle \subset D_3\) is:
    \[N_{D_3}(\langle b \rangle) =\{g \in D_3 : g \langle b \rangle g\inv = \langle b\rangle\}.\]
    Now \(\langle b \rangle \subset N_{D_3}(\langle b \rangle)\) and \(\abs{\langle b \rangle}=2\) so \(\abs{N_{D_3}(\langle b \rangle)}=2\) or \(6\) (by Lagrange's theorem). If the normaliser has order \(6\) then it must contain all of \(D_3\), but this is not the case as if \(g=a\) we find that 
    \[\begin{aligned}
        a \langle b \rangle a\inv &=a\{e,b\}a\inv \\
        &= \{aea\inv, aba\inv\} \\
        &=\{e,a^2b\} \\
        &\neq \langle b \rangle.
    \end{aligned}\]
    So, \(a \notin N_{D_3}(\langle b \rangle)\) hence, \(\abs{N_{D_3}(\langle b \rangle)}=2\). Therefore, \(N_{D_3}(\langle b \rangle)=\langle b \rangle=\{e,b\}\).
\end{example}

\begin{mdremark}
    This is the same as observing that \(\langle b \rangle\) is \textbf{not} a normal subgroup. On the other hand \(a \triangleleft D_3\) so, \(N_{D_3}(\langle a \rangle)=D_3\).
\end{mdremark}

\begin{definition}
    The \textbf{orbit} of \(x\) in a G-set, \(X\), is given by
    \[\orb(x)=\{g\cdot x : \forall g \in G\}.\]
\end{definition}

\begin{mdthm}
    The orbit of an element, \(x\in X\), is an equivalence class with equivalence relation \(y \sim x\) if there exists a \(g\in G\) such that \(y=g\cdot x\).
\end{mdthm}

\begin{corollary}
    Orbits partition the G-set i.e.
    \[\abs{X}=\abs{\orb(x_1)}+\abs{\orb(x_2)}+\cdots+\abs{\orb(x_n)},\]
    where \(x_1,x_2,\ldots, x_n\) are representative elements from each of the distinct orbits that cover \(X\).
\end{corollary}

\begin{proof}
    As equivalence classes are either or identical or disjoint then the orbits partition \(X\).
\end{proof}

\begin{example}
    If \(X=G\) and \(g \cdot x =gx\) then \(\orb(x)=G\) (while the stabiliser \(G_x=\{e\}\)).
\end{example}

\begin{mdexample}
    If \(X\) is the set of all left cosets of \(G\) with respect to a subgroup \(H\subset G\) with group action \(g \cdot (g_1 H)=gg_1 H\). Then \(\orb(H)\) is the set of all left cosets of \(H\) in \(G\). (While the stabiliser \(G_H=H\)).
\end{mdexample}

\begin{example}
    For \(X=G\) with \(g \cdot x=gxg\inv\) then \(\orb(x)\) is the conjugacy class of \(x\).
\end{example}

\begin{mdthm}[Orbit-stabiliser theorem]
    Let \(G\) be a finite group and \(X\) be a G-set. For each \(x\in X\)
    \[\abs{\orb(x)}=\frac{\abs{G}}{\abs{G_x}}.\]
\end{mdthm}

\begin{proof}
    We will construct a bijection between the elements of \(\orb(x)\) and the cosets \(gG_x \in G/G_x\) given by
    \[M(g\cdot x)=gG_x.\]
    \begin{itemize}
        \item Injective: suppose \(\exists g,h \in G\) such that \(g \cdot x \neq h \cdot x\) but \(M(g\cdot x) = M(h\cdot x)\). So,
        \[\begin{aligned}
            gG_x &= hG_x \\
            \then h\inv gG_x &= G_x \\
            \then h\inv g &\in G_x.
        \end{aligned}\]
        Therefore, \((h\inv g) \cdot x =x\) which implies \(g \cdot x =h \cdot x\) which contradicts the assumption that \(g \cdot x \neq h \cdot x\).
        \item Surjective: by construction, i.e. the pre-image \(gG_x \in G / G_x\) is \(g \cdot x \in \orb(x)\).
    \end{itemize}
    Hence, \(M\) is bijective, which means \(\abs{\orb(x)}=\abs{G/G_x}=\frac{\abs{G}}{\abs{G_x}}\).
\end{proof}

\begin{mdexample}
    Consider \(G=D_3\):
    \begin{itemize}
        \item When \(x=e\), \(C_{D_3}(e)=\{g \in D_3 : geg\inv =e\}=D_3\) so,
        \[\abs{\orb(e)} =\frac{\abs{D_3}}{\abs{C_{D_3}(e)}}=\frac{\abs{D_3}}{\abs{D_3}}=1.\]
        \item When \(x=a\), we know that \(\langle a \rangle \subset C_{D_3}(a)\) then as \(\abs{\langle a \rangle}=3\) we have that \(\abs{C_{D_3}}(a)=3\) or \(6\). It cannot be \(6\) as \(a \notin Z(D_3)\). Therefore, \(\abs{C_{D_3}}=3\) which implies \(\abs{\orb(a)}=\frac{6}{3}=2\) so, \(\orb(a)=\{a,a^2\}\).
        \item When \(x=b\) then \(\langle b \rangle C_{D_3}(b)\) so \(\abs{C_{D_3}(b)}=2\) hence, \(\abs{\orb(b)}=\frac{6}{2}=3\) which implies \(\orb(b)=\{b,ab,a^2b\}\).
    \end{itemize}
\end{mdexample}

\begin{example}
    Let \(X\) be the set of any permutation of all the letters of the word BANANAS. How many distinct `words' are there in \(X\). \\
    \textbf{Solution:} The symmetric group \(S_7\) acts on the letter to give all permutations, \(\abs{S_7}=7!\). So, 
    \[\abs{\orb(\text{BANANAS})}=\frac{\abs{S_7}}{3! \cdot 2!}.\]
    Where, \(3!\) represents the permutation of the \(3\) A's and \(2!\) the permutation of the \(2\) N's.
\end{example}

\begin{mdthm}
    Let \(G\) be a finite group of order \(p^n\) where \(p\) is prime and \(n\in \ZZ^+\). Then \(Z(G)\) contains more than one element.
\end{mdthm}

\begin{proof}
    Let \(X=G\) with \(g \cdot x =gxg\inv\). As the conjugacy classes cover \(G\) then
    \[\abs{G}=\abs{\orb(g_1)}+\abs{\orb(g_2)}+\ldots +\abs{\orb(g_k)}.\]
    At least one of the conjugacy classes is that of the identity element thus, it contains just a single element, i.e. \(\abs{\orb{e}}=1\) so,
    \[\abs{G}=1+\abs{\orb(g_1)}+\abs{\orb(g_2)}+\ldots +\abs{\orb(g_{k-1})}.\]
    As \(\abs{G}=p^n\) then by the orbit-stabiliser theorem we have that 
    \[\abs{\orb{g_i}}=\frac{\abs{G}}{\abs{C_G(g_i)}} =\frac{p^n}{p^{m_i}}=p^{n-m_i}\]
    for some integer \(m_i \leq n\) and for each \(\orb(g_i)\).
    Therefore,
    \[\abs{G}=1+p^{m_1}+p^{m_2}+\ldots p^{m_{k-1}}.\]
    Since, \(\abs{G}=p^n\) we must have that \(\abs{G}\equiv 0 \Mod{p}\) whereas, the RHS \(\equiv 1 \Mod{p}\). Therefore, at least one or more conjugacy class must contain only one element. Suppose \(\orb(g_1)=\{g_1\}\) then \(m_1=0\) and \(gg_1g\inv =g\) for all \(g \in G\), this implies \(gg_1=g_1g\) for all \(g \in G\) hence, \(g_1 \in Z(G)\) with \(e\) so \(\abs{Z(G)}>1\).
\end{proof}

\begin{theorem}
    Let \(G\) be a group such that \(G / Z(G)\) is a cyclic group. Then \(G\) is abelian so, \(Z(G)=G\).
\end{theorem}

\begin{proof}
    Suppose that \(\frac{G}{Z(G)}\) is a cylic group generated by \(gZ(G)\) hence, every element of \(G\) lies in one of the cosets \(g^n Z(G)\) for \(n \in \ZZ\). Therefore, any pair of elements \(g_1,g_2 \in G\) may be written as \(g_1 = g^{n_1}z_1\) and \(g_2 =g^{n_2}z_2\). Now,
    \[g_1g_2=g^{n_1}z_1g^{n_2}z_2=g^{n_1}g^{n_2}z_1z_2=g^{n_1+n_2}z_2z_1=g^{n_2}z_2g^{n_1}z_1=g_2g_1.\]
    Therefore, \(G\) is abelian.
\end{proof}

\begin{mdthm}
    Any finite group with \(\abs{G}=p^2\) elements, where \(p\) is prime, is abelian.
\end{mdthm}

\begin{proof}
    As \(\abs{G}=p^2\) then \(\abs{Z(G)}>1\). Since \(Z(G)\) is a subgroup then \(\abs{Z(G)}\) is \(p\) or \(p^2\). If \(\abs{Z(G)}=p^2\) then \(Z(G)=G\) and \(G\) is abelian. If \(\abs{Z(G)}=p\) then, \(\abs{\frac{G}{Z(G)}} =p\) which implies \(\frac{G}{Z(G)}\) is isomorphic to a cyclic group hence, \(G\) is abelian and this implies \(Z(G)=G\), which is a contradiction. So, \(\abs{Z(G)}=p\) is not allowed.
\end{proof}

\begin{mdlemma}
    Let \(G\) and \(H\) be two subgroups of a finite group \(J\). Then
    \[\abs{J}=\abs{GH}=\frac{\abs{G}\abs{H}}{\abs{G \cap H}}.\]
\end{mdlemma}

\begin{mdthm}
    A group of order \(p^2\) is isomorphic to either \(\ZZ_{p^2}\) or \(\ZZ_p \times \ZZ_p\), where \(p\) is prime.
\end{mdthm}

\begin{proof}
    The order of each element in \(G\) must divide \(\abs{G}=p^2\). Hence, each element \(g \in G\) has order \(1,p\) or \(p^2\). If \(G\) contains an element of order \(p^2\) then \(G \cong \ZZ_{p^2}\). If \(G\) has only (non-identity) elements of order \(p\), then let \(g\) and \(h\) be two such elements such that \(\langle g \rangle \cap \langle h \rangle =\{e\}\). Then using the lemma above 
    \[\abs{\langle g \rangle \langle h \rangle} = \frac{ \abs{\langle g \rangle} \abs{\langle h \rangle} }{ \abs{\langle g \rangle \cap \langle h \rangle} } =\frac{p^2}{1} =p^2.\]
    Hence, \(\langle g \rangle \langle h \rangle \) covers \(G\) and the distinct elements of \(G\) take the form \(g^n h^m\) for \\ \(0 \leq n,m <p-1\). Using \(\phi(g^n h^m) = (g^n, h^m) \in \ZZ_p \times \ZZ_p\) we can show that 
    \[\langle g \rangle \langle h \rangle \cong \ZZ_p \times \ZZ_p.\]
\end{proof}

\begin{mdremark}
    Notice \(\ZZ_{p^2}\) is cyclic whereas \(\ZZ_p \times \ZZ_p\) is not cyclic.
\end{mdremark}

\begin{mdremark}
    Hence, there are only two groups (up to isomorphism) of order \\ \(4,9,15,25, \ldots \) both of which are abelian.
\end{mdremark}

\pagebreak

\section{The Sylow theorems}

\begin{definition}
    Let \(p\) be a positive, prime integer. A \(\mathbf{p}\)\textbf{-group} is a group in which every element has order of a power of \(p\).
\end{definition}

\begin{mdremark}
    If \(G\) is a finite \(p\)-group, then \(\abs{G}=p^k\) for some \(k\).
\end{mdremark}

\begin{example}
    \(\ZZ_p\) for a prime \(p\) is a \(p\)-group. Whereas \(\abs{D_3}= 6= 2 \cdot 3\) is not a \(p\)-group as its order is not some power of a prime \(p\).
\end{example}

\begin{definition}
    A \textbf{p-subgroup} is one in which every element is a power of \(p\).
\end{definition}

\begin{mdremark}
    A \(p\)-subgroup does not necessarily need to be a subgroup of a \(p\)-group.
\end{mdremark}

\begin{definition}
    Let \(G\) be a finite group with \(\abs{G}=mp^k\) where \(p\) is a prime which \textit{does not} divide \(m \in \ZZ\) i.e. \(p \nmid m\). A subgroup of order \(p^k\) is called a \textbf{Sylow p-subgroup}.
\end{definition}

\begin{mdremark}
    A Sylow \(p\)-subgroup is the maximal \(p\)-subgroup in \(G\).
\end{mdremark}

\begin{mdexample}
    Let \(\abs{G} = 60 = 2^2 \cdot 3 \cdot 5\) then, potentially \(G\) may have:
    \begin{itemize}
        \item Sylow \(2\)-subgroup of order \(2^2 =4\),
        \item Sylow \(3\)-subgroup of order \(3\) and
        \item Sylow \(5\)-subgroup of order \(5\).
    \end{itemize}
    The Sylow theorems tells us that these Sylow \(p\)-subgroup exists and how many there are.
\end{mdexample}

\begin{mdthm}[Sylow theorems]
    Let \(G\) be a (finite) group of order \(mp^k\) where \(p\) is prime and \(p \nmid m\), then:
    \begin{itemize}
        \item[I.] a Sylow \(p\)-subgroup exists,
        \item[II.] for each prime \(p\), the Sylow \(p\)-subgroup are conjugate to each other and
        \item[III.]  Let \(n_p\) be the number of Sylow \(p\)-subgroups then 
        \begin{enumerate}
            \item[(i).] \(n_p \equiv 1 \Mod{p}\),
            \item[(ii).] \(n_p = \frac{\abs{G}}{\abs{N_G(P)}}\) where \(N_G(P)\) is the normaliser of the Sylow \(p\)-subgroup \(P \subset G\) and 
            \item[(iii)] \(n_p \mid m\) (\(n_p\) is the index of the Sylow \(p\)-subgroup in \(G\)).
        \end{enumerate}
    \end{itemize}
\end{mdthm}

\begin{mdremark}
    Sylow II: suppose there exists multiple Sylow $5$-subgroup then, each of the subgroups are conjuagte to each other.
\end{mdremark}

\begin{mdexample}
    For \(G=D_3\), as \(\abs{D_3}=6=2\cdot 3\) then:
    \begin{itemize}
        \item for \(p=2\) then \(m=3\) hence, Sylow (I) implies a \(2\)-subgroup of order \(2\) exists i.e. \(\langle b \rangle \cong \ZZ_2\).
        \begin{itemize}
            \item Sylow (III) implies by
                \begin{enumerate}
                    \item[(i).] \(n_2 \equiv 1 \Mod{2}\), so \(n_2 = 1,3,5 \ldots\)
                    \item[(ii).] \(n_2 \leq \abs{D_3}=6\)
                    \item[(iii).] \(n_2\) divides \(3\) which implies \(n_2 =1\) or \(3\). \\
                    For example, let \(P=\langle b \rangle\) then \(P \subset N_{D_3}(P)\) so, \(\abs{N_{D_3}(P)}=2\) or \(6\). \(P\) is not a normal subgroup so \(n_2 = \frac{\abs{D_3}}{\abs{N_{D_3}(P)}} = \frac{6}{2}= 3\).
                \end{enumerate}
            \item Sylow (II) implies these subgroups are \(\langle b \rangle , \langle ab \rangle\) and \(\langle a^2 b \rangle\) obtained by \\ conjugation as \(a\langle b \rangle a\inv = \langle a^2b \rangle\) and \(a \langle a^2 b \rangle a\inv = \langle ab \rangle\).
        \end{itemize}
        \item For \(p=3\) then \(m=2\):
        \begin{itemize}
            \item Sylow (I) implies a \(3\)-subgroup of order \(3\) exists.
            \item Sylow (III)
            \begin{itemize}
                \item[(i).] \(n_3 \equiv 1 \Mod{3}\) so, \(n_3 =1\) or \(4\).
                \item[(ii).] \(n_3 \leq 6\).
                \item[(iii).] \(n_3\) divides \(2\) so \(n_3=1\).
            \end{itemize}
            The Sylow \(3\)-subgroup is \(\langle a \rangle\).
            \item Sylow (II) implies \(a \langle a \rangle a\inv =\langle a \rangle\) and \(b \langle a \rangle b\inv =\langle a^2 \rangle =\langle a \rangle\). Conjugation does not produce any new \(3\)-subgroups. Hence, \(\langle a \rangle\) is a normal subgroup.
        \end{itemize}
    \end{itemize}
\end{mdexample}

\begin{mdthm}
    A Sylow \(p\)-subgroup is a normal subgroup if and only if it is the only Sylow \(p\)-subgroup i.e. \(n_p=1\).
\end{mdthm}

\begin{proof}
    We will prove it in two parts.
    \begin{itemize}
        \item Proof of \((\then)\). If \(P \triangleleft G\) then \(gPg\inv =P\) hence there are no new conjugate subgroups which implies \(n_p=1\).
        \item Proof of \((\Leftarrow)\). If \(n_p=1\) then \(gPg\inv =P\) therefore, \(P \triangleleft G\).
    \end{itemize}
\end{proof}

\subsection{Example use of the Sylow theorems}

\includepdf[pages=-]{./Resources/Sylow Theorems Lecture example.pdf}

\includepdf[pages=-]{./Resources/An example emphasising the features used in the proof of the Sylow Theorems.pdf}








\pagebreak

\appendix

\addcontentsline{toc}{section}{Appendix}
\section*{Appendix}

\section{Equivalence relations}

\begin{definition}
    A binary operation on a set \(X\) is said to be an \textbf{equivalence relation}, if and only if it is reflexive, symmetric and transitive. That is for all \(a,b,c \in X\):
    \begin{itemize}
        \item Reflexivity: \(a \sim a\);
        \item Symmetry: \(a \sim b\) if and only if \(b\sim a\);
        \item Transitivity: if \(a\sim b\) and \(b\sim c\) then \(a\sim c\).
    \end{itemize}
\end{definition}

\subsection{Equivalence classes}

\begin{theorem}
    If \(\sim\) is an equivalence relation on a set \(X\) and \(x,y\in X\) then, these statements are equivalent:
    \begin{itemize}
        \item \(x\sim y\);
        \item \([x]=[y]\);
        \item \([x]\cap [y] =\emptyset\)
    \end{itemize}
\end{theorem}

\section{Functions}

\subsection{Well-defined maps}

\begin{definition}
A map is said to be \textbf{well-defined} if and only if the map is \textbf{not} a \textbf{many-to-one} map.
\end{definition}

\begin{mdnote}
    In a well-define map an element from the domain \textbf{cannot} be mapped to two or more elements in the range.
\end{mdnote}

\subsection{Injectivity}

\begin{definition}
    A function $f : X \to Y$ is \textbf{injective} (or \textbf{one-to-one}) if every element $f(x) \in B$ is mapped to by at most one element in the domain $A.$ An injective function is called an \textbf{injection}.
    
    With symbols, a function \(f : A \to B\) is called \textbf{injective} if, for all \(x_1,x_2 \in X, f(x_1)=f(x_2)\)  implies that \(x_1=x_2\).
\end{definition}
    
\begin{mdnote}
    To show that a function is injective, first suppose that \(f(x_1)=f(x_2)\) then show by direct implication that \(x_1=x_2\).
\end{mdnote}

\begin{example}
    Show that the function \(f:\RR \to \RR\) given by \(f(x)=5x-3\) is injective. \\
    \textbf{Solution:} Suppose that \(f(x_1)=f(x_2)\); then we have
    \[\begin{aligned}
        f(x_1)&=f(x_2) \\
        \then 5x_1-3&=5x_2-3 \\
        \then x_1 &=x_2.
    \end{aligned}\]
    Thus, \(f\) is injective.
\end{example}
    
\begin{theorem}
    A function $f$ is injective if and only if every horizontal line intersects the graph of $f$ no more than once.
\end{theorem}

\subsection{Surjectivity}

\begin{definition}
    A function $f : X \to Y$ is \textbf{surjective} (or \textbf{onto}) if every element in $B$ is mapped to by at least one element of $A.$ A surjective function is called a \textbf{surjection}. 
    
    In symbols, a function \(f:X \to Y\) is called \textbf{surjective} if for all \(y \in Y\) there exists \(x \in X\) such that \(f(x)=y\).
\end{definition}

\begin{example}
    Show that the map \(f : \RR \to \RR\) given by \(f(x)=x^3\) is surjective. \\
    \textbf{Solution:} Suppose \(y \in \RR\). We require that \(f(x)=y\), i.e. \(x^3=y\) thus, we can take \(x=\sqrt[3]{y}\). We rewrite the proof. Suppose that \(y \in \RR\). Then let \(x=\sqrt[3]{y}\) we have 
    \[\begin{aligned}
        f(x)&=f(\sqrt[3]{y}) \\
        &= (\sqrt[3]{y})^3 \\
        &=y.
    \end{aligned}\]
    Thus, \(f\) is surjective.
\end{example}

\subsection{Bijections}

\begin{definition}
    A function $f : A \to B$ is \textbf{bijective} if it is both surjective and injective. A bijective function is called a \textbf{bijection}.
\end{definition}

\end{document}
