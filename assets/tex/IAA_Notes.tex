\documentclass[12pt, a4paper]{article}   	
\usepackage{geometry}
\usepackage{amssymb}
\usepackage{mathtools}
\usepackage{amsmath}
\usepackage{amsthm}
\usepackage[utf8]{inputenc}
\usepackage{color}   
\usepackage{tikz}
\usepackage{tcolorbox}
\usepackage{multicol}
\usepackage[thinc]{esdiff}
\usepackage{physics}
\usepackage{bm}
\usepackage{pdfpages}
\usepackage{pdflscape}
\usepackage{listings}
\usepackage{float}

\usepackage{hyperref}

\hypersetup{colorlinks=true, linktoc=all, linkcolor=black,}

\newcommand{\bb}[1]{\mathbb{#1}}
\newcommand{\f}[2]{\frac{#1}{#2}}
\newcommand{\imply}{\Rightarrow}
\newcommand{\Cal}[1]{\mathcal{#1}}
\newcommand{\mb}[1]{\mathbf{#1}}

\newcommand{\Mod}[1]{\ (\mathrm{mod}\ #1)}

\DeclareMathOperator{\lcm}{lcm}
\DeclareMathOperator{\id}{id}

\newtheorem*{remark}{Remark}
\newtheorem*{note}{Note}

\theoremstyle{definition}
\newtheorem{definition}{Definition}[section]
\newtheorem{theorem}{Theorem}[section]
\newtheorem*{example}{Example}
\newtheorem{proposition}{Proposition}

\theoremstyle{plain}
\newtheorem{corollary}{Corollary}[theorem]
\newtheorem{lemma}[theorem]{Lemma}

\title{Introduction to Abstract Algebra Notes}
\date{}
\author{Francesco Chotuck}
\begin{document} 
\maketitle 
]
\tableofcontents

\pagebreak

\section{The Integers}

\subsection{The Division Algorithm}

\begin{definition}
We say that $m$ is \textbf{divisible} by $n$ or that $n$ \textbf{divides} $m$ if $m=nk$ for some $k \in \bb{Z}.$ It is written as $n|m$ and read as ``$n$ divides $m$.'' 
\end{definition}

\begin{proposition}
Suppose that $a,b,c,d \in \bb{Z}$ and $n\in \bb{N}.$
\begin{enumerate}
	
	\item If $a|b$ and $b|c,$ then $a|c.$

	\item If $a|n,$ then $a\leq n.$

	\item If $n|a$ and $n|b,$ then $n|(ac+bd).$

\end{enumerate}
\end{proposition}

\begin{theorem}\textbf{(The Division Algorithm)}
Suppose that $m\in \bb{Z}$ and $n\in\bb{N}.$ Then there exists $q,r \in \bb{Z}$ such that both of the following hold:
\begin{enumerate}
	
	\item $m=qn+r,$ and 

	\item $0\leq r<n.$

\end{enumerate}

Note that $r$ is called the \textbf{remainder}. Moreover, $q$ and $r$ are the unique pair of integers such that these hold.
\end{theorem}

\begin{remark}
In general $m$ and $m+kn,$ for $k\in \bb{Z},$ have the same remainder on division by $n.$
\end{remark}

\subsection{The Euclidean Algorithm}

\begin{definition}
Suppose $a$ and $b$ are integers, not both zero. Then the \textbf{greatest common divisor} of $a$ and $b,$ denoted $\gcd(a,b),$ is the greatest integer which divides both $a$ and $b.$ Thus $g = \gcd(a, b)$ if the following hold:
\begin{enumerate}
	
	\item $g|a$ and $g|b,$

	\item If $d|a$ and $d|b,$ then $d\leq g.$

\end{enumerate}
\end{definition}

\begin{theorem}
Suppose that $a,b \in \bb{Z}$ and that $a$ and $b$ are not both zero, and let $g=\gcd(a,b).$ Then 
\begin{enumerate}
	
	\item $g$ is the least positive integer of the form $ax+by$ with $x,y \in \bb{Z};$

	\item if $d|a$ and $d|b,$ then $d|g.$

\end{enumerate}
\end{theorem}

\begin{proposition}
If $a,b,c,k \in \bb{Z}$ with $b\neq 0$ and $a=kb+c,$ then $\gcd(a,b)=\gcd(b,c).$
\end{proposition}

\begin{example}\label{example:Euclidean Algorithm gcd}
Work out the $\gcd(114,42).$ \\
\textbf{Solution:} By using the Euclidean Algorithm.
$$\begin{aligned}
114 &= 2\cdot 42 +30 \\
42 &=1\cdot 30 +12 \\
30 &=2\cdot 12 +6 \\
12 &=2\cdot 6 +0. \\
\end{aligned}$$ 
The last non-zero remainder is $6$ therefore, $\gcd(114,42)=6.$
\end{example}

\begin{example}
We can ``unwind'' the Euclidean Algorithm to obtain an the equation $ax+by=g=\gcd(a,b)$ for $a,b,x,y,g \in \bb{Z}.$ Using the equations from Example \ref{example:Euclidean Algorithm gcd} gives: 
$$\begin{aligned}
6&=30-2\cdot12 \\
12 &=42-30 \\
30&=114-2\cdot 42.
\end{aligned}$$ 
Substituting each equation into the previous one gives: 
$$\begin{aligned}
6	&=30-2\cdot 12 \\
	&=30-2(42-30) \\
	&=3\cdot 30 - 2\cdot 42\\
	&=3(114-2\cdot 42)-2\cdot 42 \\
	&=3\cdot114 -8\cdot 42.
\end{aligned}$$
Therefore we have $6=\gcd(114,42)=114x+42y$ by taking $x=3$ and $y=-8.$
\end{example}

\subsection{Relatively prime integers}

\begin{definition}
Suppose that $a,b \in \bb{Z}$ and that $a$ and $b$ are not both zero. We say that $a$ and $b$ are \textbf{relatively prime} if $\gcd(a, b) = 1.$
\end{definition}

\begin{corollary}
Suppose that $a, b \in \bb{Z},$ not both $0.$ Then $a$ and $b$ are relatively prime if and only if $ax + by = 1$ for some $x, y \in \bb{Z}.$
\end{corollary}

\begin{corollary}
Suppose that $a, b \in \bb{Z},$ not both $0.$ Let $g = \gcd(a, b).$ Then $a/g$ and $b/g$ are relatively prime.
\end{corollary}

\begin{corollary}\label{corollary:coprime divisibility}
Suppose that $a, b, c \in \bb{Z}$ with $a$ and $b$ relatively prime. If $a|bc,$ then $a|c.$
\end{corollary}

\begin{remark}
We can think of Corollary \ref{corollary:coprime divisibility} in the following formulation: if $a$ and $b$ are relatively prime and $a|bc,$ then $a$ cannot divide into $b$ hence, it must divide $c.$
\end{remark}

\begin{corollary}
Suppose that $a, b, c \in \bb{Z}$ with $a$ and $b$ relatively prime. If $a|c$ and $b|c,$ then $ab|c.$
\end{corollary}

\subsection{Linear Diophantine equations}

\begin{definition}
A \textbf{linear Diophantine equation} (in two variables) is an equation of the form $$ax + by = c,$$ where $a,b,c \in \bb{Z}$ and we regard $x$ and $y$ as variables taking only integer values. 
\end{definition}

\begin{theorem}
Suppose that $a, b, c \in \bb{Z}$ and that $a$ and $b$ are not both zero. Then the equation $ax+by = c$ has solutions $x,y \in \bb{Z}$ if and only if $\gcd(a,b)|c.$
\end{theorem}

\begin{theorem}
Suppose that $a, b, c \in \bb{Z}$ and that $g = \gcd(a, b).$ If $x = x_0, y = y_0$ is an integer solution to $ax + by = c,$ then all the integer solutions of $ax + by = c$ are given by 
$$\begin{aligned}
x&=x_0 +k\left(\frac{b}{g}\right), \\
y&=y_0-k\left(\frac{a}{g}\right) \quad \text{for } k\in \bb{Z}.
\end{aligned}$$
\end{theorem}

\begin{example}
You’re at a shop in the airport about to leave New York. You want to spend the last $\$15$ of your US money on chocolate by buying a combination of Mercury bars, which cost $39$ cents apiece, and Andromeda bars, which cost $47$ cents apiece. How many of each do you need to buy so you won’t have any change? \\
\textbf{Solution:} Let $x$ be the number of Mercury bars and $y$ the number of Andromeda bars. We have to solve the equation 
$$39x+47y=1500$$ where $x$ and $y$ are non-negative integers. By the Euclidean Algorithm we have that $g=\gcd(39,47)=1$ therefore, 
$$39x_0+47y_0=1500$$ with $x_0=-9000$ and $y_0=7500.$ Hence, all the solutions are 
$$\begin{aligned}
x&=-9000+47k,\\
y&=7500-39k
\end{aligned}$$ for $k\in\bb{Z}.$ But we require $x$ and $y$ to be non-negative integers so, the solutions must satisfy 
$$\begin{aligned}
x&=-9000+47k\geq0 \\ 
y&=7500-39k\geq0.
\end{aligned}$$ 
This is equivalent to $k \geq 9000/47 \approx 191.5,$ and $k \leq 7500/39 \approx 192.3.$ Therefore the only possible value of $k$ is $192,$ giving $x=24$ and $y=12.$
\end{example}

\subsection{Prime factorisation}

\begin{definition}
Suppose that $n$ is an integer and $n > 1.$ Then $n$ is \textbf{prime} if its only positive divisors are $1$ and $n;$ otherwise $n$ is \textbf{composite.}
\end{definition}

\begin{proposition}
Suppose that $a, b \in \bb{Z}$ and $p$ is a prime number. If $p|ab$ then $p|a$ or $p|b.$
\end{proposition}

\begin{corollary}
Suppose that $a_1, a_2, \ldots, a_k \in \bb{Z}$ and $p$ is a prime number. If $p|a_1a_2\ldots a_k$ then $p|a_i$ for some $i \in \{1,2,\ldots,k\}.$
\end{corollary}

\begin{theorem}\textbf{(The Fundamental theorem of Arithmetic).} Suppose that $n$ is an integer greater than $1.$ Then there is a positive integer $k$ and prime numbers $p_1,p_2,\ldots,p_k$ such that $n = p_1p_2\ldots p_k.$ Moreover the factorization is unique, up to changing the order of the prime factors $p_1,p_2,\ldots,p_k$.
\end{theorem}

\begin{corollary}
Suppose that $m$ and $n$ are integers, not both $0.$ Then $m$ and $n$ are relatively prime if and only if they have no common prime divisors.
\end{corollary}

\begin{corollary}
Suppose that $p_1, p_2,\ldots, p_k$ are distinct prime numbers and $r_1,r_2 ,\ldots,r_k$ and $s_1,s_2 ,\ldots,s_k$ are non-negative integers. Let $m = p_1^{r_1}p_2^{r_2}\ldots p_k^{r_k}$ and $n = p_1^{s_1}p_2^{s_2}\ldots p_k^{s_k}.$
\begin{enumerate}
	
	\item Then $n|m$ if and only if $s_1 \leq r_1,s_2 \leq r_2,\ldots,s_{k-1} \leq r_{k-1}$ and $s_k \leq r_k.$

	\item $\gcd(m,n) = p_1^{t_1}p_2^{t_2}\ldots p_k^{t_k}$ where $t_1=\min(r_1,s_1), t_2=\min(r_2,s_2),\ldots,t_{k-1}=\min(r_{k-1},s_{k-1})$ and $t_k=\min(r_k,s_k).$

	\item $\lcm(m,n)=p_1^{t_1}p_2^{t_2}\ldots p_k^{t_k}$ where $t_1=\max(r_1,s_1), t_2=\max(r_2,s_2),\ldots,t_{k-1}=\max(r_{k-1},s_{k-1})$ and $t_k=\max(r_k,s_k).$

\end{enumerate}
\end{corollary}

\pagebreak

\section{Binary operations}

\subsection{Arithmetic modulo \texorpdfstring{$n$}{TEXT}}

\begin{definition}
Suppose that $a$ and $b$ are integers. We say that $a$ is \textbf{congruent} to $b$ \textbf{modulo} $n$ if $a - b$ is divisible by $n.$ The notation for this is $a \equiv b \Mod{n}.$
\end{definition}

\begin{proposition}
Suppose that $a, b$ and $n$ are integers and $n > 0.$ Then
the following are equivalent: 
\begin{enumerate}
	
	\item[(a)] $a \equiv b \Mod{n};$

	\item[(b)] $a=b+kn$ for some $k\in \bb{Z};$

	\item[(c)] $a$ and $b$ have the same remainder on division by $n$.

\end{enumerate}
\end{proposition}

\begin{corollary}
If $a \equiv b \Mod{n}$ and $b \equiv c \Mod{n},$ then $a \equiv c \Mod{n}.$
\end{corollary}

\begin{proposition}
Suppose that $a, b, c, d$ and $n$ are integers and $n > 0.$ If $a \equiv b \Mod{n}$ and $c \equiv d \Mod{n},$ then
\begin{enumerate}
	
	\item $a\pm c \equiv b\pm d \Mod{n},$ and 

	\item $ac \equiv bd \Mod{n}.$

\end{enumerate}
\end{proposition}

\begin{proposition}
Suppose that $a,b,c,d \in \bb{Z}$ and $m,n \in \bb{N}.$ If $a\equiv b \Mod{n}$,then $a^m \equiv b^m \Mod{n}.$
\end{proposition}

\begin{definition}\label{definition:congruence class}
If $n\in \bb{N}$ and $a\in \bb{Z},$ then we call the set $$\{b\in \bb{Z} : b \equiv a \Mod{n}\}$$ the \textbf{congruence} (or \textbf{residue}) \textbf{class} of $a$ \textbf{modulo} $n,$ and denote it by $[a]_n.$ 
\end{definition}

\begin{remark}
Definition \ref{definition:congruence class} is equivalent as saying $$[a]_n=\{a+kn : k \in \bb{Z}\}.$$
\end{remark}

\begin{example}
$$[5]_{12}=\{b\in\bb{Z} : b \equiv 5 \Mod{n}\}=\{\ldots,-19,-7,5,17,\ldots\}.$$
\end{example}

\begin{remark}
Since $a\equiv a \Mod{n}$ to get $[a]_n$ it suffices to add integer constants to $a.$
\end{remark}

\begin{proposition}
Suppose $a, b \in \bb{Z}, n \in \bb{N}.$ Then 
$$a\equiv b \Mod{n} \iff [a]_n=[b]_n.$$
\end{proposition}

\begin{proposition}
Suppose that $n \in \bb{N}$ and $a,b,a',b' \in \bb{Z}.$ If $[a]_n = [a']_n$ and $[b]_n = [b']_n,$ then $[a + b]_n = [a' + b']_n$ and $[ab]_n = [a'b']_n.$
\end{proposition}

\begin{definition}
Let $\bb{Z}_n$ denote the set of congruence classes modulo $n,$ so: $$\bb{Z}_n=\{[0]_n,[1]_n,\ldots,[n-1]_n\}$$ is a set with $n$ elements; each element of $\bb{Z}_n$ is itself a set of infinitely many integers.
\end{definition}

\begin{proposition}
The addition and multiplication operations on $\bb{Z}_n$ are commutative and associative.
\end{proposition}

\pagebreak

\section{Groups}

\subsection{Definition of a group}

\begin{definition}
A \textbf{group} is a set $G$ with a binary operation $*$ satisfying the following properties:
\begin{enumerate}
	
	\item $*$ is associative;

	\item there is an element $e\in G$ such that $e*g=g*e=g$ for all $g\in G;$

	\item if $g\in G,$ then there is an element $h\in G$ such that $g*h=h*g=e.$

\end{enumerate}
\end{definition}

\begin{remark}
By the definition of a binary operation the operation $*$ already satisfies the closure axiom.
\end{remark}

\subsection{Example of groups}

We write $(G,*)$ to express a group $G$ with operation $*.$

\begin{example} \hphantom{This is to make it nice}
\begin{itemize}
	\item $(\bb{Z},+);$
	\item $(\bb{R}^{\times},\cdot)$ where $\bb{R}^{\times}=\bb{R}\backslash\{0\};$
	\item $(M_2(\bb{R}),+);$
	\item $(\text{GL}_2(\bb{R}),\cdot)$ where $\text{GL}_2(\bb{R})=\{A\in M_2(\bb{R}):\det(A)\neq 0\}.$ This is called the ``General linear'' group;
	\item $D_3$ the set of symmetries of an equilateral triangle.
\end{itemize}
\end{example}

\subsubsection{The \texorpdfstring{$D_3$}{TEXT} group}

Let $D_3$ denote the set of symmetries of an equilateral triangle. There are $6$ of these:

\begin{itemize}
	\item The identity, $e;$

	\item  Two $120^{\circ}$ rotations; call these $\rho_1$ (clockwise) and $\rho_2$ (anticlockwise);

	\item Three reflections (one axis through each vertex), call these $\sigma_A,\sigma_B$ and $\sigma_C$ (ordering the vertices clockwise.)
\end{itemize}

\begin{figure}[H]
\centering
\includegraphics[width=1\textwidth]{./Resources/Group D3.png}
\caption{Cayley table of $D_3$}
\label{}
\end{figure}

\begin{proposition}
The $D_3$ group is not cyclic.
\end{proposition}

\begin{proposition}
The $D_3$ group is not abelian.
\end{proposition}

\begin{proposition}
The group $D_3$ is isomorphic to $S_3.$
\end{proposition}

\begin{remark}
The order of the Dihedral group $n$ is given by $2n.$
\end{remark}

\subsubsection{The \texorpdfstring{$\bb{Z}_n$}{TEXT} groups}

The $(\bb{Z}_n,+)$ is a group.

\begin{proposition}
Suppose $a, n \in \bb{Z}$ with $n \geq 1.$ Then $[a]_n$ has a multiplicative inverse in $\bb{Z}_n$ if and only if $a$ and $n$ are relatively prime.
\end{proposition}

\begin{proposition}
Suppose that $a,b,n \in \bb{Z}$ with $n > 0.$ If $a \equiv b \Mod{n},$ then $\gcd(a, n) = \gcd(b, n).$
\end{proposition}

\begin{definition}
We define the set $\bb{Z}_n^{\times}= \{[a]_n :\gcd(a,n)=1\}.$
\end{definition}

\begin{proposition}
$(\bb{Z}_n^{\times},\cdot)$ is a group.
\end{proposition}

\subsection{Permutation groups}

\begin{definition}
For any set $A,$ we define the \textbf{identity function} on $A$ as the function
$$\id_A: A\to A, \quad \text{where } \id_A(a)=a \text{ for all } a\in A.$$
\end{definition}

\begin{definition}
Suppose that $f$ is a function from $A$ to $B.$ We say that a function $g : B \to A$ is an \textbf{inverse function} of $f$ if 
$$g\circ f =\id_A \quad \text{and} \quad f \circ g =\id_B.$$
\end{definition}

\begin{definition}
Suppose that $f$ is a function from $A$ to $B.$ We say $f$ is \textbf{injective} if it has the following property: $$x,y\in A, \quad f(x)=f(y) \imply x=y.$$
\end{definition}

\begin{definition}
Suppose that $f$ is a function from $A$ to $B.$ We say $f$ is \textbf{surjective} if it has the following property: $$b\in B \imply b=f(a) \quad \text{for some } a\in A.$$ 
\end{definition}

\begin{definition}
A function $f : A \to B$ is bijective if it is both injective and surjective.
\end{definition}

\begin{proposition}
Suppose that $f : A \to B$ and $g : B \to C$ are functions.
\begin{enumerate}

	\item If $f$ and $g$ are injective, then so is $g\circ f.$

	\item If $f$ and $g$ are surjective, then so is $g\circ f.$

	\item If $f$ and $g$ are bijective, then so is $g\circ f.$

\end{enumerate}
\end{proposition}

\begin{theorem}
Suppose that $f : A \to B$ is a function. Then $f$ has an inverse function if and only if $f$ is bijective.
\end{theorem}

\begin{definition}
We define $S_A$ to be the set of bijective functions from $A$ to $A.$
\end{definition}

\begin{proposition}
If $A$ is a set, then $S_A$ is a group under $\circ.$
\end{proposition}

\begin{remark}
The group $S_A$ under $\circ$ is called the \textbf{symmetric group}, or \textbf{permutation group}, on $A,$ and its elements are called \textbf{permutations} of $A.$
\end{remark}

\begin{proposition}
There are $n!$ elements in $S_n.$
\end{proposition}

\begin{proposition}
The order of $\sigma$ when it is made up of disjoin cycles, is the $\lcm$ of the cycle length of each disjoint cycle.
\end{proposition}

There are two standard ways of denoting elements of $S_n.$ One of these is to write 
$$\begin{pmatrix} 1&2&\ldots&n \\ a_1 & a_2 &\ldots & a_n  \end{pmatrix}$$ for the function (or permutation) $\sigma \in S_n$ such that $\sigma(1) = a_1, \sigma(2) = a_2,\ldots, \sigma(n) = a_n.$

The other standard notation is \textbf{cycle notation}. If $a_1, a_2,\ldots, a_k$ are distinct elements of $\{1,2,\ldots,n\}$ (so $k \leq n),$ we write 
$$(a_1a_2\ldots a_k).$$

An element of $S_n$ of the form $(a_1a_2\ldots a_k)$ is called a $k$-\textbf{cycle} (or just a \textbf{cycle}).

\subsection{Cycle composition}

TO DO!!

\subsubsection{Cycle decomposition}

\begin{definition}
\textbf{Disjoint cycles} have no cycle elements in common.
\end{definition}

\begin{example}
For example $(1,2,3)$ and $(4,5,6,7)$ are disjoint cycles.
By contrast, $(1,2,3)$ and $(3,4,5,6)$ are not disjoint because they have the 3 in common.
\end{example}

\begin{example}
\includepdf[pages=-]{./Resources/Cycle decomposition example.pdf}
\end{example}

\subsection{Basic properties of groups}

\begin{proposition}
Suppose $(G,*)$ is a group and $a \in G.$ Then there is a unique $b\in G$ such that $a*b=b*a=e.$
\end{proposition}

\begin{definition}
Suppose that $(G,*)$ is a group and $a \in G.$ Then the inverse of $a$ is the unique element $b \in G$ such that $a * b = b * a = e;$ we denote this element by $a^{-1}.$
\end{definition}

\begin{definition}
We say a group $(G,*)$ is an \textbf{abelian} group if the operation $*$ is commutative; i.e., $a * b = b * a$ for all $a, b \in G.$
\end{definition}

\begin{proposition}\textbf{(Cancellation Law).}
Suppose that $G$ is a group and $a, b, c \in G.$ If $ab = ac$ or $ba=ca,$ then $b=c.$
\end{proposition}

\begin{corollary}
If $G$ is a group and $a,b \in G,$ then there is a unique $x \in G$ such that $ax = b$ and a unique $y \in G$ such that $ya = b.$
\end{corollary}

\begin{proposition}
Suppose $G$ is a group and $g, h \in G.$ Then 
\begin{enumerate}
	
	\item If $ab=e,$ then $a=b^{-1}$ and $b=a^{-1.}$

	\item $(ab)^{-1}=b^{-1}a^{-1}.$

	\item $(a^{-1})^{-1}=a.$

\end{enumerate}
\end{proposition}

\subsection{Powers of group elements}

\begin{definition}
If $(G, *)$ is a group and $g \in G,$ then we define the $n^{\text{th}}$ power of $g$ for positive integers $n$ by 
$$g^n=\underbrace{g*g*\ldots*g}_{n \text{ times}}$$
and set $g^0=e$ and $g^n=\left(g^{-n}\right)^{-1}$ for $n<0.$ Such that $g^n$ is defined for all $n\in \bb{Z}.$
\end{definition}

\begin{proposition}
Suppose $G$ is a group, $g \in G$ and $m,n \in \bb{Z}.$ Then $g^mg^n = g^{m+n}$ and $(g^m)^n = g^{mn}.$
\end{proposition}

\begin{example}
ADD EXAPLES !!
\end{example}

\subsection{Orders of group elements}

\begin{definition}
Let $g$ be an element of a group $G.$ We say that $g$ has \textbf{finite order} (in $G$) if $g^n =e$ for some $n \in \bb{N}.$ In that case the least $n\in \bb{N}$ such that $g^n = e$ is called the \textbf{order} of $g.$ If no such positive integer $n$ exists, we say that $g$ has \textbf{infinite order}.
\end{definition}

\begin{remark}
The identity element of a group has order 1.
\end{remark}

\begin{theorem}
Suppose that $g$ is an element of a group $G.$
\begin{enumerate}
	
	\item If $g$ has infinite order, then $g^n =e \iff n=0.$

	\item If $g$ has finite order $d,$ then $g^n =e \iff d|n.$

\end{enumerate}
\end{theorem}

\begin{corollary}
Suppose that $g$ is an element of a group $G.$
\begin{enumerate}
	\item If $g$ has infinite order, then the powers of $g$ are distinct; i.e., $g^m = g^n \iff m=n.$

	\item If $g$ has finite order $d,$ then $g^m = g^n \iff m \equiv n \Mod{d}.$
\end{enumerate}
\end{corollary}

\begin{definition}
Suppose that $G$ is a group. If $G$ has infinitely many elements, we say $G$ has \textbf{infinite order}. Otherwise we say $G$ has \textbf{finite order}, and we define the order of $G$ to be the number of elements in $G.$
\end{definition}

\begin{corollary}
If a group $G$ has finite order, then so does every element of $G.$
\end{corollary}

\begin{proposition}
Suppose that $a_1, a_2,\ldots, a_k$ are distinct elements of the set $\{1,2,\ldots,n\},$ and let $\sigma = (a_1 \, a_2 \, \ldots \, a_k).$ Then
\begin{enumerate}
	
	\item $\sigma^{-1} = (a_k \, a_{k-1} \, \ldots \, a_1).$

	\item $\sigma$ has order $k.$

\end{enumerate}
\end{proposition}

\subsection{Subgroups}

\begin{definition}
Suppose that $(G,*)$ is a group. A subset $H \subseteq G$ is called a \textbf{subgroup} of $G$ if $H,$ with the operation $*,$ is a group.
\end{definition}

\begin{proposition}
Suppose that $(G, *)$ is a group and $H \subseteq G.$ Then $H$ is a subgroup of $G$ if and only if the following conditions are all satisfied:
\begin{enumerate}
	
	\item $h,h'\in H \imply h*h'\in H;$

	\item $e \in H;$

	\item $h\in H \imply h^{-1} \in H.$

\end{enumerate}
\end{proposition}

\subsection{Cyclic groups}

\begin{proposition}
Suppose that $G$ is a group and $g \in G.$ Then 
$$H=\{g^n : n\in \bb{Z}\}$$
is a subgroup fo $G.$
\end{proposition}

\begin{definition}
If $g$ is an element of a group $G,$ then $\{g^n :n \in \bb{Z}\}$ is denoted $\langle g \rangle$ and called the \textbf{subgroup of $G$ generated by $g.$}
\end{definition}

\begin{proposition}
Suppose that $g$ is an element of a group $G.$
\begin{enumerate}
	
	\item If $g$ has infinite order, then so does $\langle g \rangle.$

	\item If $g$ has order $d\in \bb{N},$ then so does $\langle g \rangle.$

\end{enumerate}
\end{proposition}

\begin{definition}
Suppose that $G$ is a group. We say that $G$ is a \textbf{cyclic group} if $G=\langle g \rangle$ for some $g\in G.$ If $G=\langle g \rangle,$ then we say that $g$ is a \textbf{generator} (of $G$).
\end{definition}

\begin{proposition}
Suppose that $G$ is a finite group of order $n.$
\begin{enumerate}
	
	\item If $g\in G,$ then $g$ has order at most $n.$

	\item $G$ is cyclic if and only if $G$ has an element of order $n.$

	\item If $G$ is cyclic and $g \in G,$ then $g$ is a generator of $G$ if and only if $g$ has order $n.$

\end{enumerate}
\end{proposition}

\begin{proposition}
If $G$ is a cyclic group, then $G$ is abelian.
\end{proposition}

\begin{theorem}
Suppose that $G$ is a group, $g \in G$ has order $d,$ and $a \in \bb{Z}.$ Then $g^a$ has order $d/ \gcd(a, d).$
\end{theorem}

\begin{corollary}
Suppose that $a \in \bb{Z}$ and $n \in \bb{N}.$ Then the element $[a]_n$ in $\bb{Z}_n$ has order $n/\gcd(a,n).$
\end{corollary}

\begin{corollary}
Suppose that $a \in \bb{Z}$ and $n \in \bb{N}.$ Then $[a]_n$ generates $\bb{Z}_n$ if and only if $a$ and $n$ are relatively prime.
\end{corollary}

\begin{theorem}
Every subgroup of a cyclic group is cyclic.
\end{theorem}

\begin{corollary}
If $H$ is a subgroup of $\bb{Z},$ then 
$$H=\langle m \rangle=\{km : k \in \bb{Z}\}$$
for some $m\in \bb{Z}.$
\end{corollary}

\subsection{Cosets}

\begin{definition}
Suppose that $(G,*)$ is a group, $H$ is a subgroup of $G,$ and $g$ is an element of $G.$ The subset $g*H \subseteq G$ defined by 
$$g*H=\{g*h : h\in H\}$$
is called a \textbf{left coset} of $H$ in $G.$
\end{definition}

\begin{proposition}
Suppose that $G$ is a group, $H$ is a subgroup of $G,$ and $g$ and $g'$ are elements of $G.$ Then the following are equivalent: 
\begin{enumerate}
	
	\item $g'H=gH;$

	\item $g' \in gH;$

	\item $g^{-1}g'\in H.$

\end{enumerate}
\end{proposition}

\begin{corollary}
Suppose that $G$ is a group, $H$ is a subgroup of $G$ and $g$ is an element of $G.$ Then $g$ is in exactly one left coset of $H$ in $G,$ namely $gH.$
\end{corollary}

\subsection{Lagrange's Theorem}

\begin{lemma}
Suppose that $H$ is subgroup of a group $G,$ and that $H$ has finite order $d.$ Then every left coset of $H$ in $G$ has $d$ elements.
\end{lemma}

\begin{theorem}\textbf{(Lagrange's Theorem).}
Suppose that $G$ is a group of finite order. If $H$ is a subgroup of $G,$ then the order of $G$ is divisible by the order of $H.$ i.e. 
$$\text{number of elements in } G = \text{number of left cosets} \times \text{size of each coset}.$$
\end{theorem}

\begin{definition}
If $H$ is a subgroup of a group $G,$ then the number of left cosets of $H$ in $G$ is called the \textbf{index} of $H$  in $G,$ and denoted $[G:H]$ i.e. we call the $$\frac{\text{order of }G}{\text{order of }H}=\text{number of distinct left cosets of } H \text{ in } G.$$
\end{definition}

\begin{corollary}
Suppose that $G$ is a group of finite order $n,$ and that
$g \in G.$ Then the order of $g$ is a divisor of $n.$
\end{corollary}

\begin{corollary}
Suppose that $G$ is a group of order $p,$ where $p$ is a prime number. Then $G$ is cyclic.
\end{corollary}

\begin{corollary}
Suppose that $G$ is a group of order $n,$ and $g$ is an element of $G.$ Then $g^n =e.$
\end{corollary}

\begin{theorem}\textbf{(Fermat's Little theorem).}
Suppose that $p$ is a prime number and $a$ is an integer.
\begin{enumerate}
	
	\item $a^p \equiv a \Mod{p};$

	\item if $a$ is not divisible by $p,$ then $a^{p-1} \equiv 1 \Mod{p}.$

\end{enumerate}
\end{theorem}

\begin{example}
Find the remainder of $50^{50}$ on division by $13.$ \\
\textbf{Solution:} $$50 \equiv -2 \Mod{13} \imply (50)^{50} \equiv (-2)^{50} \Mod{13}.$$ 
By Fermat's Little theorem: 
$$\begin{aligned}
13 \nmid -2 \imply (-2)^{13-1} \equiv 1 \Mod{13} \\
(-2)^{50} &\equiv (-2)^{12}(-2)^{12}(-2)^{12}(-2)^{12}(-2)^2 \\
&\equiv (1)(1)(1)(1)(4) \\
&\equiv 4 \Mod{13}
\end{aligned}$$
Therefore, the remainder of $50^{50}$ on division by $13$ is $4.$
\end{example}

\subsection{Product groups}

\begin{definition}
Suppose that $A$ and $B$ are sets. The product of $A$ and $B$ is defined to be the set 
$$A\times B =\{(a,b):a\in A, b\in B\}.$$ 
Thus an element of $A \times B$ is an ordered pair $(a, b),$ where a is an element of $A$ and $b$ is an element of $B.$
\end{definition}

\begin{proposition}
If $G$ and $H$ are groups, then $G \times H$ is a group under the binary operation defined as 
$$(g,h)*(g',h')=(g*_G g', h*_H h').$$
\end{proposition}

\subsection{Homomorphism}

\begin{definition}
Let $(G,*_G)$ and $(H,*_H)$ be groups. A function $\phi : G \to H$ is a \textbf{homomorphism} (of \textbf{groups}) if 
$$\phi(g*_G g')=\phi(g)*_H\phi(g') \quad \forall g,g'\in G.$$
\end{definition}

\begin{proposition}
If $\phi : G \to H$ is a homomorphism of groups, then
\begin{enumerate}
	
	\item $\phi(e_G)=e_H;$

	\item $\phi(g^{-1})=(\phi(g))^{-1}.$

\end{enumerate}
\end{proposition}

\begin{proposition}\label{prop:powers in h-morph}
Suppose $\phi : G \to H$ is a homomorphism of groups. If $g\in G$ and $n\in \bb{Z},$ then $\phi(g)^n =\phi(g^n).$
\end{proposition}

\begin{proposition}
Suppose that $G, H$ and $K$ are groups, and that $\phi : G\to H$ and $\psi:H\to K$ are homomorphisms. Then $\psi\circ \phi:G\to K$ is a homomorphism.
\end{proposition}

\begin{definition}
If $G$ and $H$ are groups, then a function $\phi : G \to H$ is an \textbf{isomorphism} if it is a bijective homomorphism.
\end{definition}

\begin{proposition}
Suppose that $G, H $and $K$ are groups, and that $\phi : G\to H$ and $\psi:H\to K$ are isomorphisms. Then $\psi\circ\phi:G\to K$ is an isomorphism.
\end{proposition}

\begin{proposition}
Suppose that $G$ and $H$ are groups and $\phi : G \to H$ is an isomorphism. Let $\psi = \phi^{-1} : H \to G$ be the inverse function of $\phi.$ Then $\psi$ is an isomorphism.
\end{proposition}

\begin{definition}
If $G$ and $H$ are groups, then we say $G$ is \textbf{isomorphic} to $H$ if there is an isomorphism $\phi:G\to H.$
\end{definition}

\begin{proposition}
If $\phi: G \to H$ is a homomorphism, and $g$ is an element of $G$ of finite order, then the order of $\phi(g)$ divides the order $g$. If $\phi$ is injective, then the order of $\phi(g)$ is equal to the order of $g.$
\end{proposition}

\begin{proposition}
Suppose that $G$ is a cyclic group.
\begin{enumerate}
	
	\item If $G$ has infinite order, then $G$ is isomorphic to $\bb{Z}.$

	\item If $G$ has order $n,$ then $G$ is isomorphic to $\bb{Z}_n.$

\end{enumerate}
\end{proposition}

\begin{corollary}
If $G$ is a group of prime order $p,$ then $G$ is isomorphic to $\bb{Z}_p.$
\end{corollary}

\begin{proposition}
Suppose that $G$ and $H$ are isomorphic groups. Then
\begin{enumerate}
	
	\item $G$ is abelian if and only if $H$ is abelian;

	\item $G$ is cyclic if and only if $H$ is cyclic.
\end{enumerate}
\end{proposition}

\begin{proposition}
Suppose that $G$ and $H$ are groups and $\phi : G \to H$ is an isomorphism. Then for each $g \in G,$ the order of $g$ in $G$ is the same as the order of $\phi(g)$ in $H.$
\end{proposition}

\begin{definition}
The \textbf{image} of $\phi$ is defined as 
$$\phi(G)=\{\phi(g) : g \in G\},$$ also denoted image($\phi$).
\end{definition}

\begin{proposition}
Suppose that $G$ and $H$ are groups and $\phi : G \to H$ is a homomorphism. Then $\phi(G)$ is a subgroup of $H.$
\end{proposition}

\begin{definition}
Suppose that $\phi : G \to H$ is a homomorphism of groups.
The \textbf{kernel} of $\phi$ is the following subset of $G:$
$$\ker(\phi)=\{g\in G : \phi(g) =e_H\}.$$
\end{definition}

\begin{proposition}
Suppose that $G$ and $H$ are groups and $\phi : G \to H$ is a homomorphism. Then $\ker(\phi)$ is a subgroup of $G.$
\end{proposition}

\pagebreak

\section{Rings}

\begin{definition}
A \textbf{ring} is a set $R$ with binary operations $+$ and $*$ satisfying:
\begin{enumerate}
	
	\item $(R,+)$ is an abelian group;

	\item the operation $*$ is associative and has an identity elements in $R;$

	\item $x*(y+z)=(x*y)+(x*z)$ and $(y+z)*x=(y*x)+(z*x)$ for all $x, y, z \in R.$

\end{enumerate}
\end{definition}

\begin{example} \hphantom{This is to make it look nice.}
\begin{itemize}

	\item $(\bb{Z},+,\cdot);$

	\item $(\bb{Q},+,\cdot);$

	\item $(\bb{R},+,\cdot);$

	\item $(\bb{C},+,\cdot);$

	\item $(\bb{Z}_n,+,\cdot);$

	\item $(M_2(\bb{R}),+,\cdot).$

\end{itemize}
\end{example}

\subsection{Basic properties of rings}

We will use $0_R$ to represent the \textbf{additive identity} and $1_R$ for \textbf{multiplicative identity.}

\begin{proposition}
Suppose that $(R, +, *)$ is a ring.
\begin{enumerate}

	\item $0_R * x = 0 = x *0_R$ for all $x\in R.$

	\item $(-x) * y = -xy = x*(-y)$ for all $x,y \in R.$

\end{enumerate}
\end{proposition}

\begin{proposition}
If $(R,+,\cdot)$ is a ring, then 
$$(n\cdot x)y= n\cdot (xy) = x(n \cdot y)$$ for all $x,y \in R$ and $n\in \bb{Z}.$
\end{proposition}

Let $m_R$ represent the $m^{\text{th}}$ multiple of $1_R$ i.e. $m_R=m\cdot 1_R.$

\begin{proposition}
Suppose that $R$ is a ring. Then
\begin{enumerate}
	
	\item $m_Rx=m\cdot x=xm_R$ for all $m \in \bb{Z}, x
	\in R;$

	\item $m_R+n_R =(m+n)_R$ and $m_Rn_R=(mn)_R$ for all $m,n \in \bb{Z};$

	\item $(m_R)^n=(m^n)_R$ for all $m\in \bb{Z}, n\in \bb{N}.$
\end{enumerate}
\end{proposition}

\subsection{Subring}

\begin{definition}
Suppose that $(R,+,*)$ is a ring and $S$ is a subset of $R.$ Then $S$ is a \textbf{subring} of $R$ if $S,$ with the operations $+$ and $*,$ is a ring and $1_S=1_R.$
\end{definition}

\begin{proposition}
Suppose that $(R, +, *)$ is a ring and $S \subset R.$ Then $S$ is a subring of $R$ if and only if all of the following hold: 
\begin{enumerate}
	\item $0_R \in S$ and $1_R \in S;$

	\item if $x,y \in S,$ then $x+y\in S$ and $x*y \in S;$

	\item if $x\in S,$ then $-x \in S.$	
\end{enumerate}
\end{proposition}

\begin{definition}
Suppose that $(R, +, *)$ is a ring. An element $x \in R$ is called a \textbf{unit} in $R$ if $x$ has a multiplicative inverse in $R,$ i.e., if there exists an element $y \in R$ such that 
$$x*y = 1_R =y*x.$$
\end{definition}

\begin{theorem}
Suppose that $(R, +, *)$ is a ring. Then $(R^{\times}, *)$ is a group.
\end{theorem}

\begin{remark}
For any ring $R$ we let $R^{\times}$ denote the set of units in $R.$
\end{remark}

\subsection{Types of rings}

\begin{definition}
We say that a ring $(R,+,*)$ is \textbf{commutative} if the operation $*$ on $R$ is commutative, i.e. 
$$x*y=y*x \quad \text{for all } x,y\in R.$$
\end{definition}

\begin{definition}
We say that a ring $R$ is an \textbf{integral domain} (or simply a \textbf{domain}) if $R$ is commutative, $0_R \neq 1_R$ and 
$$x,y \in R, xy=0_R \imply x=0_R \text{ or } y=0_R.$$
\end{definition}
	
\begin{proposition}
Let $n$ be a positive integer. Then $\bb{Z}_n$ is a domain if and only if $n$ is prime.
\end{proposition}

\begin{proposition}
Suppose that $R$ is an integral domain, $x,y,z \in R$ and $x=0_R.$ If $xy=xz,$ then $y=z.$
\end{proposition}

\begin{definition}
A ring $R$ is called a \textbf{field} if $R$ is an integral domain and every non-zero element of $R$ is a unit (where non-zero means different from $0_R).$
\end{definition}

\begin{proposition}
Suppose that $R$ is a commutative ring. Then $R$ is a field if and only if 
$$R^{\times}=R\backslash \{0_R\}=\{x\in R : x\neq 0_R\}.$$
\end{proposition}

\subsection{Matrix rings}

\begin{proposition}
If $R$ is a ring, then $M_n(R)$ is a ring under matrix addition and multiplication.
\end{proposition}

\subsection{Ring homomorphism}

\begin{definition}
Suppose that $(R,+_R,*_R)$ and $(S,+_S,*_S)$ are rings. A function $\phi : R \to S$ is a \textbf{homomorphism} (of \textbf{rings}) if all of the following hold:
\begin{enumerate}
	
	\item $\phi(x +_R y)=\phi(x) +_S \phi(y)$ for all $x,y \in R;$

	\item $\phi(x *_R y)=\phi(x) *_S \phi(y)$ for all $x,y \in R;$

	\item $\phi(1_R)=1_S.$

\end{enumerate}
\end{definition}
	
\begin{proposition}
Suppose that $\phi : R \to S$ is a homomorphism of rings.
Then $\phi(n_R)=n_S$ for all $n\in \bb{Z}.$
\end{proposition}

\begin{proposition}(This is from 2017 past paper.)
Proposition \ref{prop:powers in h-morph} can be extended to homomorphism of rings with additive property: if $\phi : R \to S$ is a homomorphism of rings $r \in R$ and $n\in \bb{Z}$ then $\phi(n\cdot r)=n\cdot \phi(r).$
\end{proposition}

\begin{proposition}
If $\phi : R \to S$ and $\psi : S \to T$ are homomorphisms of rings, then so is $\psi \circ \phi :R\to T.$
\end{proposition}

\begin{proposition}
Suppose that $\phi : R \to S$ is a homomorphism of rings. Then the restriction of $\phi$ defines a homomorphism of groups from $R^{\times}$ to $S^{\times}.$ (In particular, if $r \in R^{\times},$ then $\phi(r) \in S^{\times}.)$
\end{proposition}

\begin{example}
\begin{itemize}

	\item The function $f : \bb{Z} \to \bb{Z}_n,$ defined by $f(a) = [a]_n;$

	\item The complex conjugation $\bb{C} \to \bb{C}$ is a ring homomorphism;

	\item If $R$ and $S$ are rings, the zero function from $R$ to $S$ is a ring homomorphism if and only if $S$ is the zero ring.

\end{itemize}
\end{example}

\begin{remark}
There is \textbf{NO} ring homomorphism between $\bb{Z}_n\to \bb{Z}$ for $n\geq 1.$
\end{remark}

\begin{definition}
If $R$ and $S$ are rings, then a function $\phi : R \to S$ is called an \textbf{isomorphism} (of \textbf{rings}) if $\phi$ is a bijective homomorphism (of rings). In that case we say $R$ is \textbf{isomorphic} to $S.$
\end{definition}

\begin{example}
We have that $\bb{Z}_6$ is isomorphic to $\bb{Z}_2 \times \bb{Z}_3.$
\end{example}

\begin{proposition}
Suppose that $\phi : R \to S$ is an isomorphism of rings.
\begin{enumerate}

	\item The inverse function $\phi^{-1} : S \to R$ is also an isomorphism of rings.

	\item If $\psi : S \to T$ is also an isomorphism of rings, then so is the composite $\psi \circ \phi : R \to T.$

\end{enumerate}
\end{proposition}

\begin{corollary}
If $\phi : R \to S$ is an isomorphism of groups, then its restriction $R^{\times} \to S^{\times}$ is an isomorphism of groups.
\end{corollary}

\subsection{The Chinese Remainder Theorem}

\begin{theorem}
If $m$ and $n$ are relatively prime positive integers, then the function 
$$\psi : \bb{Z}_{mn} \to \bb{Z}_m \times \bb{Z}_n$$
defined by $\psi([a]_{mn})=([a]_m,[a]_n)$ is an isomorphism of rings.
\end{theorem}

\begin{theorem}\textbf{(The Chinese Remainder Theorem).}\label{theorem:CRT}
Suppose that $a,b,m,n \in \bb{Z}$ with $m,n > 0.$ If $m$ and $n$ are relatively prime, then there are integers $x \in Z$ which simultaneously satisfy the congruences 
$$x \equiv a \Mod{m} \quad \text{and} \quad x\equiv b \Mod{n}.$$
\end{theorem}

Theorem \ref{theorem:CRT} does not explicitly outline how to find a solution to $[x]_{mn}$ in practice. This is done as follows: we use the Euclidean algorithm to find $r,s \in \bb{Z}$ such that $mr+ns=1.$ We note that

$$\begin{aligned}
mr &\equiv 0 \Mod{m}, \quad mr \equiv 1 \Mod{n} \\
ns &\equiv 1 \Mod{m}, \quad ns \equiv 0 \Mod{n}.
\end{aligned}$$

Therefore letting $x_0 = b(mr)+a(ns)$ gives
$$\begin{aligned}
	x_0 &\equiv b\cdot 0+a\cdot 1 \equiv a \Mod{m} \\
\text{and }x_0 &\equiv b \cdot 1 + a\cdot  0 \equiv \Mod{n}.
\end{aligned}$$

Therefore the general solution is $x\equiv bmr+ans \Mod{mn}.$

\begin{remark}
To solve system of congruence with $3$ or more congruences, solve a pair with the Chinese remainder theorem first. Then use the the result to solve the next pair and vice versa.
\end{remark}

\begin{example}
Use the Chinese Remainder Theorem to find all integers $x$ such that 
$$x \equiv 11 \Mod{47} \quad \text{and} \quad x\equiv 3 \Mod{31}.$$
\textbf{Solution:} \\
First we check if $47$ and $31$ are relatively prime. They are since $\gcd(47,31)=1.$ We use the Euclidean Algorithm to solve find $r,s \in \bb{Z}$ such that $47r+31s=1.$ We begin as such:
$$\begin{aligned}
47&=1\cdot 31+16 \\
31&=1\cdot 16+15 \\
16&=1\cdot 15+1 \\
\end{aligned}$$ 

We can ``unwind'' the system of equations:

$$\begin{aligned}
1 	&= 16-15 \\
	&= 16-(31-16)\\
	&= 2 \cdot 16 -31 \\
	&= 2(47-31)-31\\
	&= 2 \cdot 47 - 3\cdot 31
\end{aligned}$$

Therefore we have that $r=2$ and $s=-3.$ The general solution of the system of congruences 
$$x \equiv a \Mod{m} \quad \text{and} \quad x \equiv b \Mod{n}$$ 
is given by 
$$x \equiv bmr+ans \Mod{mn}.$$

Hence, the solution to our system of congruences is: 
$$\begin{aligned}
x &\equiv (3)(47)(2)+(11)(31)(-3) \Mod{47 \times 31} \\
x &\equiv 282-1023 \Mod{47 \times 31} \\
x &\equiv -741 \Mod{47 \times 31} \\
x &\equiv 716 \Mod{47 \times 31} \\
\end{aligned}$$

This means that, 
$$\begin{aligned}
x-716&=1457k \\
x&=716 +1457k \quad \text{for } k\in \bb{Z}.
\end{aligned}$$
\end{example}

\begin{example}
Use the Chinese Remainder Theorem to find all integers $x$ such that 
$$4x \equiv 5 \Mod{9} \quad \text{and} \quad 2x\equiv 6 \Mod{20}.$$
\textbf{Solution:} \\
We begin by finding the a solution for each system of congruence independently i.e. find a solution of $x$ for $4x \equiv 5 \Mod{9}$ and $2x\equiv 6 \Mod{20}.$ By using the Euclidean Algorithm we find that
$$x \equiv 9 \Mod{9} \quad \text{and} \quad x\equiv 3 \Mod{10}.$$

Then we can use the Chinese Remainder Theorem as in the example above.
\end{example}

\subsection{Euler's \texorpdfstring{$\varphi$}{TEXT} function}

\begin{definition}
\textbf{Euler's $\varphi$ function} (or \textbf{Euler's totient function}) is the number of integers in $\{0,1,2,\ldots,n-1\}$ which are relatively prime to $n.$
\end{definition}

\begin{proposition}
If $n=p$ where $p$ is prime then $\varphi(p)=p-1.$
\end{proposition}

\begin{proposition}
If $n=p$ where $p$ is prime then $\varphi(p^r)=p^r-p^{r-1}=(p-1)p^{r-1}.$
\end{proposition}

\begin{corollary}
If $m$ and $n$ are relatively prime, then $\bb{Z}^{\times}_{mn}$ is isomorphic to $\bb{Z}^{\times}_m \times \bb{Z}^{\times}_n.$ In particular, if $\gcd(m, n) = 1,$ then $\varphi(mn) = \varphi(m)\varphi(n).$
\end{corollary}

\begin{example}
$\varphi(300)=\varphi(2^2 \times 3 \times 5^2)=\varphi(2^2)\varphi(3)\varphi(5^2).$
\end{example}

\begin{corollary}
Suppose that $n$ is a positive integer and $a$ is an integer relatively prime to $n.$ Then 
$$a^{\varphi(n)} \equiv 1 \Mod{n}.$$
\end{corollary}

\end{document}