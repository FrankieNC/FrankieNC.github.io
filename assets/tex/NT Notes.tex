\documentclass[12pt, a4paper]{article}
\usepackage{francesco}
\usepackage[colorlinks=true,
            urlcolor=RubineRed,
            linktoc=all,
            linkcolor=black,
            pdfauthor={Francesco N. Chotuck},
            pdftitle={Introduction to Number Theory}
            ]{hyperref}
\hypersetup{urlcolor=RubineRed,linktoc=all, linkcolor=black,hidelinks}

%\pgfplotsset{width=\textwidth}

\pagestyle{fancy}
\lhead{Francesco Chotuck}
\rhead{5CCM224A Introduction to Number Theory}
\setlength{\headheight}{15pt}

\title{Introduction to Number Theory Notes}
\date{}
\author{Francesco Chotuck}
\begin{document}

\maketitle

\begin{abstract}
    This is KCL undergraduate module 5CCM224A, instructed by Dr Stephen Lester. The formal name for this class is ``Introduction to Number Theory''.
\end{abstract}

\tableofcontents

\pagebreak

\section{Divisibility}

\subsection{GCD \& Euclidean algorithm}

\begin{definition}
    Let \(a\) and \(b\) be two integers. We say that \(b\) \textbf{divides} \(a\) if there exists an integer \(q\) such that \(a=qb\). If \(b\) divides \(a\), we write \(b \mid a\).
\end{definition}

\begin{mdthm}
    Some basic properties of divisibility, let \(a,b,c \in \ZZ\):
\begin{enumerate}
    \item If \(a\mid b\) and \(b\mid c\), then \(a\mid c\).
    \item If \(a\mid b\) and \(a\mid c\) then \(a\mid (bx+cy)\) for all \(x,y \in \ZZ\). 
    \item If \(a\mid 1\) then \(a =\pm 1\).
    \item If \(a\mid b\) and \(b\mid a\) then \(a=\pm b\).
    \item Suppose \(c\neq 0\) then, \(a\mid b\) if and only if \(ac\mid bc\).
\end{enumerate}
\end{mdthm}

\begin{example}
    Prove \(\gcd(a,b)=\gcd(a+b,b)\).\\
    \textbf{Solution:} Let \(d\) be a divisor of \(a\) and \((a+b)\) then,
    \[\begin{aligned}
        d\mid a \quad &\text{and} \quad d\mid (a+b) \\
        \then d \mid &\underbrace{(a+b-a)}_{b}
    \end{aligned}\]
\end{example}

\begin{theorem}[Division algorithm]
    Let \(a\in \ZZ\) and \(b\in N\). Then there exists unique integers \(q,r \in \ZZ\) such that 
    \[a = qb+r\]
    and \(0\leq r <b\).
\end{theorem}

\begin{definition}
Let \(a\) and \(b\) be integers. If \(d\) is another integer such that \(d\mid a\) and \(d\mid b\) then we call \(d\) a \textbf{common divisor} of \(a\) and \(b\).
\end{definition}

\begin{definition}
    If at least one of \(a\) and \(b\) are non-zero then we define the \textbf{greatest common divisor} of \(a\) and \(b\) to be the largest positive integer \(d\) which is a common divisor of \(a\) and \(b\). This is usually denoted as \(\gcd(a,b)\).
\end{definition}

\begin{lemma}[Euclidean algorithm]
    If \(a=qb+r\) then \(\gcd(a,b)=\gcd(b,r)\).
\end{lemma}

\begin{mdexample}\label{ex: Euclidean algorithm}
Let \(a=1492\) and \(b=1066\). Then applying the Euclidean algorithm:
\[\begin{aligned}
    1492 &=1 \cdot 1066 +426 \\
    1066 &= 2 \cdot 426 +214 \\
    426 &= 1 \cdot 214 +212 \\
    214 &=1 \cdot 212 +2 \\
    212 &= 106 \cdot 2 +0.
\end{aligned}\]
The last non-zero remainder is \(2\), so \(\gcd(1492,1066)=2\).
\end{mdexample}

\begin{mdremark}
    Why the Euclidean algorithm works:
    \begin{itemize}
        \item Algorithm always terminates since the remainder strictly decreases;
        \item Refer to Lemma 1.1 -- each iteration of the algorithm does not change the \(\gcd\) of the original pair;
        \item \(\gcd(0,r)= r\) for \(r \in \ZZ\).
    \end{itemize}
\end{mdremark}

\subsection{Bezout's lemma}

\begin{mdthm}[Bezout's lemma]
Let \(a\) and \(b\) be integers (not both \(0\)). Then there exists integers \(x\) and \(y\) such that 
\[\gcd(a,b)= ax+by.\]
\end{mdthm}

\begin{mdexample}
Using the information from Example \ref*{ex: Euclidean algorithm} we can 'reverse' the Euclidean algorithm to find the integers \(x,y\) such that \(\gcd(1492,1066) = 2 = 1492x+1066y\). So, we have:
\[\begin{aligned}
    \gcd(1492,1066) &= 2 \\
                    &= 214 - 1\cdot 212 \\
                    &= 214 - 1 \cdot (426-1\cdot 214) \\
                    &= -1\cdot 426 + 2\cdot 214 \\
                    &= -1 \cdot 426 + 2(1066 -2 \cdot 426) \\
                    &=2 \cdot 1066 - 5\cdot 426 \\
                    &= 2\cdot 1066 - 5(1492 -1\cdot 1066) \\
                    &= -5 \cdot 1492 +7 \cdot 1066.
\end{aligned}\]
Therefore, \((x,y)=(-5,7)\).
\end{mdexample}

\begin{proposition}\label{prop:Bezout prop}
    Let \(a,b\) be integers, not both zero, and consider the set 
    \[S=\{ax+by : x,y \in \ZZ\}.\]
    Let \(d>0\) be the smallest positive integer in \(S\). Then \(d=\gcd(a,b)\).
\end{proposition}

\begin{mdremark}
A consequence of Proposition \ref*{prop:Bezout prop}: \\
\(\gcd(a,b)=1\) if and only if there are integers \(x,y\) such that \[1=ax+by.\]
\end{mdremark}

\begin{corollary}
    Let \(a,b\) be integers, not both zero and consider the set 
    \[S =\{ax+by : x,y \in \ZZ\};\]
    we can also consider the set 
    \[S'=\{n\gcd(a,b) : n\in \ZZ\}.\]
    Then the two sets of integers \(S,S'\) are equal.
\end{corollary}

\begin{mdnote}
    Interpretation of Corollary 1.1: linear combinations (over \(\ZZ\)) of \(a,b\) are precisely the multiples of \(\gcd(a,b)\).
\end{mdnote}

\begin{corollary}
    Let \(a,b\) be integers, not both zero. Let \(c\) be an integer. Then \(c\) is a common divisor of \(a\) and \(b\) if and only if \(c\mid \gcd(a,b)\).
\end{corollary}

\begin{definition}
    Two integers \(a,b\) are said to be \textbf{coprime} or \textbf{relatively prime} if \[\gcd(a,b)=1.\]
\end{definition}

\begin{mdlemma}
    Suppose \(a,b\) are coprime:
    \begin{enumerate}
        \item If \(a\mid c\) and \(b\mid c\) then \((ab)\mid c\);
        \item if \(a\mid (bc)\) then \(a\mid c\);
        \item if \(a\) and \(c\) are also coprime, then \(a\) and \(bc\) are coprime.
    \end{enumerate}
\end{mdlemma}

\begin{proof}
    \begin{enumerate}
        \item We have \(ax+by =1\) for some integers \(x,y\). Since \(a \mid c\) and \(b \mid c\) then we can write \(c =aj\) and \(c=bk\). Multiplying the first equation by \(c\) we get 
        \[\begin{aligned}
            cax +cby &=c \\
            (bk)ax + (aj)by &= c \\
            ab(kx)+ab(jy) &= c\\
            ab(kx+by) &=c.
        \end{aligned}\]
        So, \((ab) \mid c\).
        \item We have \(c=cax+cby\). Since \(a \mid (bc)\) and \(a \mid a\) we get that \(a \mid [a(cx)+(bc)y] =c\).
        \item We have 
        \[1= au +bv \quad \text{and} \quad 1=ax+cy.\]
        Multiplying the equations together gives
        \[\begin{aligned}
            1&=(au+bc)(ax+cy)\\
            &= a(uax +ucy + bvx) +bc(vy).
        \end{aligned}\]
        It follows that \(\gcd(a,bc)=1\).
    \end{enumerate}
\end{proof}

\subsection{LCM \& Linear Diophantine Equations}

\begin{definition}
    If \(a,b\) are integers, then a \textbf{common multiple} of \(a\) and \(b\) is an integer \(c\) such that \(a\mid c\) and \(b\mid c\).
\end{definition}

\begin{definition}
    If \(a\) and \(b\) are both non-zero, the \textbf{least common multiple} of \(a\) and \(b\) is defined to be the \textbf{smallest} (positive) integer \(\lcm(a,b)\) which is a common multiple of \(a\) and \(b\).
\end{definition}

\begin{proposition}
    Let \(a,b\) be non-zero integers. Then \[\gcd(a,b)\lcm(a,b)=\abs{ab}\]
\end{proposition}

\begin{corollary}
    Let \(a,b \in \NN\). Suppose \(\gcd(a,b)=1\) then \(\lcm(a,b)=ab\)
\end{corollary}

\begin{mdremark}
    The \(\lcm(a,b) \leq ab\) for \(a,b>0\).
    \end{mdremark}

\begin{definition}
    \textbf{Linear Diophantine equations} where \(a,b,c \in \ZZ\) are equations of the form \[ax+by=c,\] has integer solutions for \((x,y)\).
\end{definition}

\begin{mdnote}
    In general, \textbf{Diophantine equations} are equations in one or more variables, for which we seek integer valued solutions.
\end{mdnote}

\begin{theorem}
    Let \(a,b,c\) be integers, with \(a\) and \(b\) not both \(0\) and let \(g=\gcd(a,b)\). The equation
    \[ax+by=c\] 
    has an integer solution \((x,y)\) if and only if \(\gcd(a,b) \mid c\).
\end{theorem}

\begin{mdthm}
Assume \(\gcd(a,b) \mid c\). Let \(x_0\) and \(y_0\) be solutions to \(ax_0+by_0=g\). Then the solutions to 
\[ax+by=c\]
are given by \((x_n,y_n)_{n\in \ZZ}\), where 
\[\begin{aligned}
    x_n &= \frac{c}{g}x_0+\frac{b}{g}n, \\
    y_n &= \frac{c}{g}y_0-\frac{a}{g}n.
\end{aligned}\]
\end{mdthm}

%%FROM IAA

% \begin{mdthm}
%     Suppose that \(a, b, c \in \ZZ\) and that \(g = \gcd(a, b)\). If \(x = x_0, y = y_0\) is an integer solution to \(ax + by = c\), then all the integer solutions of \(ax + by = c\) are given by 
%     \[\begin{aligned}
%     x&=x_0 +k\left(\frac{b}{g}\right), \\
%     y&=y_0-k\left(\frac{a}{g}\right) \quad \text{for } k\in \ZZ.
%     \end{aligned}\]
% \end{mdthm}

\section{Prime numbers \& modular arithmetic}

\subsection{Prime numbers}

\begin{definition}
    An integer \(p>1\) is called a \textbf{prime number} or a \textbf{prime} if it has no positive divisors other than \(1\) and \(p\). \\ An integer \(n>1\) is called \textbf{composite} if it is not prime.
\end{definition}

\begin{theorem}[Fundamental theorem of arithmetic]
    \hphantom{wahoo} \\
    Every integer \(n>1\) can be expressed uniquely (up to reordering) as a product of primes.
\end{theorem}

\begin{corollary}
    There exists primes \(p_1,p_2,\ldots, p_r\) and non-negative integers, \(a_1,a_2,\ldots,a_n\) with 
    \[n=p_1^{a_1}p_2^{a_2}\cdots p_r^{a_r}.\]
\end{corollary}

\begin{lemma}[Euclid's Lemma]
    \hphantom{wahoo}
    \begin{enumerate}
        \item Let \(p\) be a prime number and let \(a,b\) be integers. Suppose \(p\mid ab\), then \(p\mid a\) or \(p\mid b\).
        \item If we have integers \(a_1,a_2,\dots,a_n\) and \(p\mid (a_1,a_2,\dots,a_n)\) then \(p\mid a_i\) for some \(i\).
    \end{enumerate}
\end{lemma}

\begin{mdlemma}
    Let 
    \[n=p_1^{a_1}p_2^{a_2}\cdots p_r^{a_r}\]
    with the \(p_i\) distinct primes and the \(a_i\) positive integers. Then,
    \begin{enumerate}
        \item \(d>0\) is a divisor of \(n\) if and only if \[d=p_1^{b_1}p_2^{b_2}\cdots p_r^{b_r}\] with \(0 \leq b_i \leq a_i\) for each \(i\).
        \item The number of positive divisors of \(n\) is \(\prod_{i=1}^r (a_i+1)\).
    \end{enumerate}
\end{mdlemma}

\begin{example}
    How many divisors does \(200\) have? \\
    \textbf{Solution:} The prime factorisation of \(200=2^3 \cdot 5^2\) therefore, \(200\) has \((3+1)(2+1) = 12\) divisors.
\end{example}

\begin{mdexample}
    How many positive divisors of \(999 = 3^3 \cdot 37\) are multiples of \(9\)? \\
    \textbf{Solution:} We have that any divisor of \(999\) is of the form \(d =3^a \cdot 37^b\) for \(0\leq a \leq 3\) and \(0 \leq b \leq 1\). For \(d\) to be a multiple of \(9\) we need \(9\mid d \iff a\geq 2\). Hence, \(2 \leq a \leq 3 \then 2\) choices and \(0\leq b \leq 1 \then 2 \) choices; we then have \(2\cdot 2 =4\) choices in total, i.e. \(4\) such divisors.
\end{mdexample}

\begin{proposition}
    For \(n\in \NN\), then the \(\gcd(n,n+1)=1\).
\end{proposition}

\begin{proof}
    If \(d\mid n\) and \(d\mid (n+1)\) then \(d\mid (n+1-n) =d\mid 1\) (i.e. any linear combination of \(n\) and \(n+1\)) so, \(d=\pm 1\). Since \(d>0\) to be the \(\gcd\) we have that \(d=1\).
\end{proof}

\begin{lemma}
    Let \(m,n\) be two positive integers with 
    \[\begin{aligned}
        m&=p_1^{a_1}p_2^{a_2}\cdots p_r^{a_r} \\
        n&=p_1^{b_1}p_2^{b_2}\cdots p_r^{b_r}
    \end{aligned}\]
    where \(a_i,b_i \geq 0\) are integers. Then,
    \begin{enumerate}
        \item \(\gcd(m,n)=p_1^{e_1}p_2^{e_2}\cdots p_r^{e_r}\) where \(e_i = \min(a_i,b_i)\).
        \item \(\lcm(m,n)=p_1^{f_1}p_2^{f_2}\cdots p_r^{f_r}\) where \(f_i = \max(a_i,b_i)\).
    \end{enumerate}
\end{lemma}

\begin{theorem}[Euclid]
    There are infinitely many primes.
\end{theorem}

\subsection{Infinite primes}

\begin{proposition}
    There are infinitely many primes of the form \(4k+3\), with \(k \in \NN\).
\end{proposition}

\begin{proof}
    We will be using the following facts:
    \begin{enumerate}
        \item For \(n \in \NN\) we have that \(n=4k,4k+1,4k+2,\) or \(4k+3\) for some \(k\in \ZZ\). [This follows from the division algorithm applied to \(4\) and \(n\)].
        \item If \(a,b \in \ZZ\) with \(a=4k+1\) and \(b=4j+1\) for some \(k,j \in \ZZ\) then \(ab=(4k+1)(4j+1)=4 \underbrace{(4kj+k+j)}_{k'}+1 = 4k'+1\). I.e. numbers of this form are closed under multiplication.
    \end{enumerate}
    We will use a proof by contradiction. Suppose, \(p_1,p_2,\ldots, p_r\) are all primes of the form \(4k+3\). Consider 
    \[\begin{aligned}
        N&=4(p_1p_2\cdots p_r - 1) +3 \\
        N&=4p_1p_2\cdots p_r -1 
    \end{aligned}\]
    This number is of the form \(4k+3\), we suppose \(N\) is not prime, so there must exist a prime which divides \(N\). If \(p\mid N\) then \(p\) is odd since \(N\) is odd. Using Fact (1) we have that \(p\neq 4k, 4k +2\) for any \(k \in \ZZ\) since \(N\) is odd. Also since \(p\mid N\) and \(p\mid p_1p_2\cdots p_r\) we know that \(p \nmid p_1p_2\cdots p_r\) since 
    \[N-4p_1p_2\cdots p_r=1\]
    so \(p \neq p_j\) for any \(j=1,2,\ldots, r\) [from divisibility facts we know that \(p\) must divide any linear combination of \(N\) and \(p_1p_2\cdots p_r\) so, we choose our linear combination to be \(N-4p_1p_2\cdots p_r=1\)]. This tells us that \(p\neq 4k+3\) for any \(k \in \ZZ\). By Fact [1] \(p=4k+1\) for \(k\in \ZZ\), because there is no \(k\in \ZZ\) for which \(4k+3 =1\). By Fact [2] we have that \(N\) is also of the form \(4k+1\), i.e. \(N=4k'+1\) for some \(k'\in \ZZ\). 
    \[\begin{aligned}
        4k'+1 = N &= 4p_1p_2\cdots p_r -1 \\
            &=4(p_1p_2\cdots p_r -1-k')= 2\\
            &\then 4\mid 2.
    \end{aligned}\]
    We have arrived at a contradiction.
\end{proof}

\begin{mdthm}
    Let \(a \in \ZZ\) and \(q \in \NN\). Suppose that \(\gcd(a,q)=1\). Then there are infinitely many primes of the form \(qk+a\) with \(k\) a positive integer.
\end{mdthm}

\subsection{Congruence}

\begin{definition}
    Let \(m\) be a non-zero integer and let \(a,b \in \ZZ\). We say that \(a\) is \textbf{congruent} to \(b\) \textbf{modulo} \(m\) if \(m\mid (a-b).\) If \(a\) is congruent to \(b\) modulo \(m\), we write \[a\equiv b \Mod{m}\] 
\end{definition}

\begin{mdremark}
    The definition of congruence also implies that 
    \begin{enumerate}
        \item \(a = b+km\) for some \(k \in \ZZ\);
        \item \(a\) and \(b\) have the same remainder on division by \(m\).
    \end{enumerate}
\end{mdremark}

\begin{mdthm}
    Some properties of congruences:
    \begin{enumerate}
        \item \(a \equiv b \Mod{m} \iff b \equiv a \Mod{m} \iff a-b \equiv 0 \Mod{m}\).
        \item If \(a \equiv b \Mod{m}\) and \(b \equiv c \Mod{m}\), then \(a \equiv c \Mod{m}\).
        \item If \(a \equiv b \Mod{m}\) and \(c \equiv d \Mod{m}\), then \(ac \equiv bd \Mod{m}\) and \(ax+cy \equiv bx+dy \Mod{m}\) for all \(x,y \in \ZZ\).
        \item For \(n\geq 1\) we have \(a^n \equiv b^n \Mod{m}\).
        \item If \(a \equiv b \Mod{m}\) and \(d\mid m\), then \(a \equiv b \Mod{d}\).
        \item Suppose \(c\neq 0\), \(a \equiv b \Mod{m}\) if and only if \(ac \equiv bc \Mod{mc}.\)
    \end{enumerate}
\end{mdthm}

\begin{example}
    Example of Property 5: 
    \[8x \equiv 2 \Mod{10} \then 4x \equiv 1 \Mod{5}.\]
\end{example}

\begin{mdnote}
    Some of these properties are inherent from congruences being an \textbf{equivalence} relation.
\end{mdnote}

\begin{definition}
    Let \(m\) be a non-zero integer and \(a \in \ZZ\). The \textbf{residue class} or \textbf{congruence class} of \(a\) is the set 
    \[\begin{aligned}
        [a]_m &= \{b \in \ZZ : b \equiv a \Mod{m}\} \\
            &= \{a+mk : k \in \ZZ\}.
    \end{aligned}\]
\end{definition}

\begin{mdremark}
    Congruence classes modulo \(m\) can be thought of all the integers that have a common remainder when divided by \(m\).
\end{mdremark}

\begin{mdnote}
    If the modulo \(m\) is not specified then write \([a]\).
\end{mdnote}

\begin{lemma}
    \[[a]_m=[b]_m \iff a\equiv b\Mod{m}.\]
\end{lemma}

\begin{definition}
    For a positive integer \(m\), the set \(\ZZ_m\) denotes the set of congruence classes modulo \(m\). That is 
    \[\ZZ_m =\{[0]_m,[1]_m,\ldots,[m-1]_m\}.\]
\end{definition}

\begin{mdremark}
    If \(\{x_1,x_2,\dots,x_m\}\) is \textbf{any} complete residue system modulo \(m\) then the set \[\ZZ_m = \{[x_1]_m,[x_2]_m,\ldots, [x_m]_m\}.\]
\end{mdremark}

\begin{definition}
    For \(m\neq 0\) and \(a,b \in \ZZ\) the operations of addition and multiplication on \(\ZZ_m\) are defined by:
    \[\begin{aligned}
        [a]_m+[b]_m&=[a+b]_m \\
        [a]_m\cdot [b]_m &= [a\cdot b]_m.
    \end{aligned}\]
\end{definition}

\begin{definition}
    Let \(m\) be a positive integer. A set \(\{x_1,x_2,\ldots,x_r\}\) is called a \textbf{complete residue system} modulo \(m\) (CRS) if for every integer \(y\) there is exactly one \(x_i\) such that
    \[y \equiv x_i\Mod{m}.\]
\end{definition}

\begin{mdremark}
    In general, every complete residue system modulo \(m\) has size \(m\).
\end{mdremark}

\begin{mdnote}
    We can reformulate the definition of a CRS as: all the elements of the group \(\ZZ_m\).
\end{mdnote}

\begin{example}
    Let \(m\) be a positive integer. Then \(\{0,1,2,\ldots m-1\}\) is a complete residue system modulo \(m\).
\end{example}

\subsection{Solving equations in \texorpdfstring{\(\ZZ_m\)}{TEXT}}

% \begin{definition}
%     The \textbf{multiplicative inverse} of an integer \(a\Mod{m}\) is the integer \(a\inv\) such that 
%     \[a\cdot a\inv \equiv 1\Mod{m}.\]
% \end{definition}

% \begin{theorem}
%     The multiplicative inverse \(a\inv \Mod{m}\) exists if and only if \(\gcd(a,m)=1\).
% \end{theorem}

\begin{mdlemma}
    Let \(m\) be a positive integer and let \(a\in \ZZ\). If \(\gcd(a,m)=1\) then there exists \(b\in \ZZ\) such that \(ab \equiv 1 \Mod{m}\). \\
    We call such \(ab\) \textbf{inverse} of \(a\) modulo \(m\), where the residue class \([b]_m\) by \([a]_m\inv\).
\end{mdlemma}

\begin{mdremark}
    Reformulation: \([a]_m \in \ZZ_m\) has a multiplicative inverse if and only if \\ \(\gcd(a,m)=1\).
\end{mdremark}

\begin{proof}
    Proof of converse\((\Leftarrow)\): \\
    Suppose \(\gcd(a,m)=1\) then we want to show \(\exists u \in \ZZ\) with \(au=1 \Mod{m}\), i.e. \([u]_m=[a]_m\inv\). By Bezout's lemma \(\exists u \in \ZZ\) such that
    \[\begin{aligned}
        au+mv &=1 \\
        \then m \mid  1&-au \\
        \then au &\equiv 1 \Mod{m}.
    \end{aligned}\]
    Proof of (\(\then\)): \\
    Suppose \(\exists [b]_m \in \ZZ_m\) with \([a]_m \cdot [b]_m =[1]_m\) i.e.
    \[\begin{aligned}
        ab &\equiv 1 \Mod{m} \\
        \then m \mid ab&-1\\
        \then mv &= ab-1.
    \end{aligned}\]
    If \(d\mid m\) and \(d\mid a\) then \(d\mid (\underbrace{mv-ab}_{-1})\) therefore \(d\mid \pm 1 \then \gcd(a,m)=1\).
\end{proof}

\begin{mdprop}
    Let \(a,b,m \in \ZZ\) and \(m\neq 0\) then 
    \[ax \equiv b \Mod{m}\]
    has solutions in the integers if and only if \(\gcd(a,m) \mid b\).
\end{mdprop}

\begin{mdremark}
    Reformulation: \([ax]_m=[b]_m\) has integer solutions if and only if \(\gcd(a,m)\mid b\).
\end{mdremark}

\begin{example}
    The linear case. Solve \(48x+14 \equiv 0 \Mod{85}\) for \(x\in \ZZ\). Note that \(\gcd(48,85)=1\). \\
    \textbf{Solution:} By the Euclidean algorithm we have that 
    \[85(13)+48(-23)=1.\]
    Now we need to find \(u \in \ZZ\) with \(48u \equiv 1 \Mod{85}\). Notice that 
    \[\begin{aligned}
        85(13)& = 1 -48(-23) \\
        &\then 85  (1-48(-23)) \\ 
        &\then 48(-23) \equiv 1 \Mod{85}
    \end{aligned}\]
    Since \(-14 \equiv -14 \Mod{85}\) by the addition law of modular arithmetic we can rewrite the original congruence as
    \[\begin{aligned}
        48x+14 -14 &\equiv -14 \Mod{85} \\
        48x &\equiv -14 \Mod{85}.
    \end{aligned}\]
    So \(u=-23\) and if we multiply the original congruence by \(u\) we have that 
    \[\begin{aligned}
        (-23)(48)x &\equiv (-14)(-23) \Mod{85} \\
        1x &\equiv 67 \Mod{85} \\
        x &\equiv 67 \Mod{85}.
    \end{aligned}\]
\end{example}

\begin{mdlemma}
    Let \(f(x)= a_0 +a_1 x+\cdots a_nx^n\) be a polynomial with integer coefficients \(a_i \in \ZZ\). If \(a \equiv b \Mod{m}\) then \(f(a) \equiv f(b)\Mod{m}\).
\end{mdlemma}

\begin{corollary}
    Suppose \(x\equiv y \Mod{m}\) then \(f(x) \equiv 0 \Mod{m}\) if and only if \\\(f(y) \equiv 0 \Mod{m}\). 
\end{corollary}

\begin{mdremark}
    To solve \(f(x)\equiv 0 \Mod{m}\) it suffices to find all the solutions among a complete residue system modulo \(m\).
\end{mdremark}

\begin{mdnote}
    When the modulo, \(m\), is very large obviously this method is not recommended being used.
\end{mdnote}

\begin{mdexample}
    Find all the solutions to 
    \[x^8 +3 \equiv 0 \Mod{4}.\]
    \textbf{Solution:} By trial and error we can consider the complete residue system of modulo \(4\). Therefore, consider the CRS \(\{-1,0,1,2\}\):
    \begin{itemize}
        \item \(x=-1 \then (-1)^8+3 = 4 \equiv 0 \Mod{8}\);
        \item \(x=0 \then 0^8+3 = 3 \not\equiv 0 \Mod{8}\);
        \item \(x=-1 \then 1^8+3 = 4 \equiv 0 \Mod{8}\);
        \item \(x=-1 \then 2^8+3 = 259 \not\equiv 0 \Mod{8}\);
    \end{itemize}
    Therefore, \(x \equiv -1 \Mod{4}\) or \(x \equiv 1 \Mod{4}\).
\end{mdexample}

\section{Multiplicative group of integers modulo \texorpdfstring{\(m\)}{TEXT}}

\begin{definition}
    Given a commutative ring \(R\) with an identity element \(1_R\) we say that \(a\in R\) is a \textbf{unit} provided there exists \(b\in R\) such that \(a\cdot b =1_R\).
\end{definition}

\begin{mdnote}
    Being a unit means the same as having a multiplicative inverse in the ring \(R\).
\end{mdnote}

\begin{definition}
    We write \(\ZZ_m^{\times}\) for the \textbf{multiplicative group} of integers modulo \(m\) of the \textbf{group of units} modulo \(m\), which are defined by 
    \[\begin{aligned}
        \ZZ_m^{\times} &= \{[a]_m \in \ZZ_m :[a]_m \text{ is a unit}\} \\
                        &=\{[a]_m \in \ZZ_m : \gcd(a,m)=1\}.
    \end{aligned}\]
\end{definition}

\begin{example}
    Consider \(\ZZ_6\) which are the units?
    \begin{itemize}
        \item \([5]_6,\) we know that \(5\cdot 5 \equiv 1 \Mod{6}\)  therefore \([5]_6\inv = [5]_6\);
        \item \([2]_6\) is not a unit because there is no solution \(x\) to the congruence \(2x \equiv 1\Mod{6}\).
    \end{itemize}
\end{example}

\begin{definition}
    Let \(m\) be a non-zero integer. A set \(\{x_1,x_2,\ldots,x_r\}\) is called a \textbf{reduced residue system} modulo \(m\) if for every integer \(y\) with \(\gcd(y,m)=1\) there is exactly one \(x_i\) such that \[y\equiv x_i \Mod{m}.\]
\end{definition}

\begin{mdnote}
    We can think of a reduced residue system as all the elements of the group \((\ZZ_m^{\times}, \times )\).
\end{mdnote}

\section{The Chinese Remainder Theorem}

% \begin{theorem}
%     Let \(m,n\) be positive \textbf{coprime} integers and \(a,b \in \ZZ\). Then there exists a unique solution \(x \Mod{mn}\) to the systems of equations 
%     \[\begin{aligned}
%         x &\equiv a \Mod{m} \\
%         x&\equiv b \Mod{n}
%     \end{aligned}\]
% \end{theorem}

\begin{theorem}[Chinese Remainder Theorem]
    Let \(m_1,m_2,\ldots,m_r\) be positive integers with \(\gcd(m_i,m_j)=1\) for all \(i\neq j\). Set \(m=m_1m_2\cdots m_r\) then, the map 
    \[\ZZ_m \to \ZZ_{m_1} \times \ZZ_{m_2} \times \cdots \times \ZZ_{m_r}\] 
    given by 
    \[[a]_m \mapsto \left( [a]_{m_1}, [a]_{m_2}, \cdots, [a]_{m_r}  \right)\]
    is a bijection.
\end{theorem}

\begin{mdprop}
    Suppose \(\gcd(m,n)=1.\) The image of the map \(\ZZ_{mn}^{\times} \to \ZZ_m \times \ZZ_n\) given by
    \[([a]_{mn}) \mapsto ([a]_m,[a]_n)\] 
    equals \(\ZZ_m^{\times} \times \ZZ_n^{\times}\).
\end{mdprop}

\begin{mdnote}
    Suppose \(\gcd(m,n)=1\) there exists an isomorphism \(\psi : \ZZ_{mn}^{\times} \to \ZZ_m^{\times} \times \ZZ_n^{\times}\) given by the map 
    \[[a]_{mn} \mapsto \left( [a]_m,[a]_n \right).\]
\end{mdnote}

\begin{mdthm}[CRT Reformulation]
    Let \(m_1,m_2,\ldots,m_r\) be positive integers with \(\gcd(m_i,m_j)=1\) for all \(i\neq j\). Let \(a_1,a_2,\ldots,a_r\) be integers then, the solutions of the simultaneous congruence equations 
    \[\begin{aligned}
        x&\equiv a_1 \Mod{m_1}\\
        x&\equiv a_2 \Mod{m_2}\\
        \vdots \\
        x&\equiv a_r \Mod{m_r}\\
    \end{aligned}\]
    are given by the integers \(x\) lying in a single congruence class \(\Mod{m_1m_2\cdots m_r}\).
\end{mdthm}

\subsection{How to use the CRT}

%%FROM IAA

\subsubsection{Method I: Euclidean Algorithm}

The CRT does not explicitly outline how to find a solution to \(x\) in practice. Suppose we are solving the simultaneous congruence of 
\[x \equiv a \Mod{m} \quad \text{and} \quad x\equiv b \Mod{n}.\]

To solve such system we use the Euclidean algorithm to find \(r,s \in \ZZ\) such that \(mr+ns=1\). Then the general solution is 
\[x \equiv bmr+ans \Mod{mn}.\]

\begin{mdremark}
    To solve a system of congruence with \(3\) or more congruences, solve a pair first with the CRT. Then use the solution to form a pair so that the CRT can be invoked again and vice versa.
\end{mdremark}

\begin{example}
    Use the Chinese Remainder Theorem to find all integers $x$ such that 
    \[x \equiv 11 \Mod{47} \quad \text{and} \quad x\equiv 3 \Mod{31}.\]
    \textbf{Solution:} \\
    First we check if $47$ and $31$ are relatively prime. They are since $\gcd(47,31)=1.$ We use the Euclidean Algorithm to solve find $r,s \in \ZZ$ such that $47r+31s=1.$ We begin as such:
    \[\begin{aligned}
    47&=1\cdot 31+16 \\
    31&=1\cdot 16+15 \\
    16&=1\cdot 15+1. \\
    \end{aligned}\]
    
    Furthermore, we can ``unwind'' the system of equations:
    
    \[\begin{aligned}
    1 	&= 16-15 \\
        &= 16-(31-16)\\
        &= 2 \cdot 16 -31 \\
        &= 2(47-31)-31\\
        &= 2 \cdot 47 - 3\cdot 31.
    \end{aligned}\]
    
    Therefore, we have that \(r=2\) and \(s=-3\). The general solution of the system of congruences 
    $$x \equiv a \Mod{m} \quad \text{and} \quad x \equiv b \Mod{n}$$ 
    is given by 
    $$x \equiv bmr+ans \Mod{mn}.$$
    
    Hence, the solution to our system of congruences is: 
    $$\begin{aligned}
    x &\equiv (3)(47)(2)+(11)(31)(-3) \Mod{47 \times 31} \\
    x &\equiv 282-1023 \Mod{47 \times 31} \\
    x &\equiv -741 \Mod{47 \times 31} \\
    x &\equiv 716 \Mod{47 \times 31} \\
    \end{aligned}$$
    
    This means that, 
    $$\begin{aligned}
    x-716&=1457k \\
    x&=716 +1457k \quad \text{for } k\in \ZZ.
    \end{aligned}$$
\end{example}

% $$\begin{aligned}
% mr &\equiv 0 \Mod{m}, \quad mr \equiv 1 \Mod{n} \\
% ns &\equiv 1 \Mod{m}, \quad ns \equiv 0 \Mod{n}.
% \end{aligned}$$

% Therefore letting $x_0 = b(mr)+a(ns)$ gives
% $$\begin{aligned}
% 	x_0 &\equiv b\cdot 0+a\cdot 1 \equiv a \Mod{m} \\
% \text{and }x_0 &\equiv b \cdot 1 + a\cdot  0 \equiv \Mod{n}.
% \end{aligned}$$

\subsubsection{Method II: Multiplicative inverses}

In the general case, suppose we are solving the simultaneous congruence of 
\[
    x \equiv a \Mod{m} \quad \text{and} \quad x\equiv b \Mod{n}.
\]
We note that to solve such system \(\gcd(m,n)=1\) therefore, by Bezout's lemma \(\exists r,s \in \ZZ\) such that \(mr+ns =1\); by such a relation we notice that 
\[
    mr \equiv 1 \Mod{n} \quad \text{and} \quad ns \equiv 1 \Mod{m}
\]
therefore we have that \(r,s\) are the multiplicative inverses of the system of congruences respectively. As such we need to find these multiplicative inverses and then set the solution 
\[
    x\equiv bmr + ans \Mod{mn}.
\]

\begin{example}
    Let us reconsider the same example from before: \\
    find all integers \(x\) such that
    \[x \equiv 11 \Mod{47} \quad \text{and} \quad x\equiv 3 \Mod{31}.\]
    \textbf{Solution:} \\
    As checked previously, \(47\) and \(31\) are coprime, so we can apply the CRT. By Bezout's lemma we have that \(\exists r,s \in \ZZ\) such that \(47r+31s=1\) therefore,
    \[
        47r \equiv 1 \Mod{31} \quad \text{and} \quad 31s \equiv 1 \Mod{47}.
    \]
    The values \(r\) and \(s\) are the multiplicative inverse of the congruence respectively; we have that \(r = [47]\inv_{31} = [2]_{31}\) and \(s = [31]\inv_{47} = [44]_{47}\). Hence, the solution to the problem is 
    \[\begin{aligned}
        x &\equiv (3)(47)(2) + (11)(31)(44) \Mod{47 \times 31} \\
        x &\equiv 282 + 15004 \Mod{1457} \\
        x &\equiv 15286 \Mod{1457} \\
        x &\equiv 716 \Mod{1457}.
    \end{aligned}\]
\end{example}

\subsection{The CRT for polynomials in \texorpdfstring{\(\ZZ_m\)}{TEXT}}

\begin{example}
        Find all solutions in \(\ZZ_{15}\) to \(f(x) \equiv 0 \Mod{15}\) for \(f(x)=2x^3+5x+2\). \\
        \textbf{Key idea:} \(f(x) \equiv 0 \Mod{15} \iff 15 \mid f(x)\) that is 
        \[\begin{aligned}
            f(x) &\equiv 0 \Mod{3} \iff 3 \mid f(x) \\
            f(x) &\equiv 0 \Mod{5} \iff 5 \mid f(x).
        \end{aligned}\]
        \textbf{Solution:} now we solve two equations 
        \begin{enumerate}
            \item \(f(x) \equiv 0 \Mod{3}\);
            \item \(f(x) \equiv 0 \Mod{5}\).
        \end{enumerate}
        Now we can just use trial and error to find the solutions:
        \begin{itemize}
            \item \(x=0, f(0)= 2 \not\equiv 0 \Mod{3}\);
            \item \(x=1, f(1)= 2+5+2 \equiv 0 \Mod{3}\);
            \item  \(x=-1, f(-1)= -2-5+2 \not\equiv 0 \Mod{3}\).
        \end{itemize}
        So our only solution for this congruence is 
        \[x\equiv 1 \Mod{3}.\]
        By a similar process for the second congruence the solution is
        \[x \equiv 4 \Mod{5}.\]    
        Applying the CRT to the congruences:
        \[\begin{aligned}
            x&\equiv 1 \Mod{3} \\
            x &\equiv 4 \Mod{5}.
        \end{aligned}\]
        We have that 
        \[\begin{aligned}
            x &\equiv (1)(5)(-1)+(4)(3)(2) \Mod{15}\\
            &\equiv -5+24 \Mod{15}\\
            &\equiv 19 \Mod{15} \\
            &\equiv 4 \Mod{15}.
        \end{aligned}\] 
\end{example}

\begin{example}
    How many solutions does the congruence
    \[x^2 \equiv 4 \Mod{15}\]
    have in \(\ZZ_{15}\)? \\
    \textbf{Solution:} Consider 
    \[\begin{aligned}
        x^2 &\equiv 4 \Mod{3} \then \text{2 solutions} \\ 
        x^2 &\equiv 4 \Mod{5} \then \text{2 solutions}
    \end{aligned}\]
    Therefore there are \(2\times 2=4\) pairs of solutions.
\end{example}

\section{Hensel's Lemma}

\begin{theorem}
    Let \(p\) be a prime and let \(f(x)=a_0+a_1x+\cdots +a_nx^n\) be a polynomial degree \(\leq n\) with integer coefficients (we allow the possibility that \(a_n=0\)). We suppose that \(a_i \not\equiv 0 \Mod{p}\) for some \(i\). Then the congruence equation 
    \[f(x) \equiv 0 \Mod{p}\]
    has \textbf{at most} \(n\) solutions in \(\ZZ_p\).
\end{theorem}

% The motivation behind Hensel's lemma. Given \(c_n,c_{n-1},\ldots,c_1,c_0\) and \(f(x) = c_nx^n+c_{n-1}x^{n-1}+\ldots c_1x+c_0\), we want to solve 
% \[f(x) \equiv 0 \Mod{p^a}.\]
% \begin{itemize}
%     \item For \(a=1\), solve \(f(x) \equiv 0 \Mod{p}\), we can solve by trial and error;
%     \item For \(a=2\), we observe that if \(f(x) \equiv 0 \Mod{p^2}\) then \(f(x) \equiv 0 \Mod{p}\). We will use the \\
%     \textbf{Strategy:} for each \([x]_p \in \ZZ_p\) with \(f(x) \equiv 0 \Mod{p}\) find all \([y]_{p^2} \in \ZZ_{p^2}\) with
%     \begin{itemize}
%         \item \([y]_p=[x]_p\);
%         \item \(f(y) \equiv 0 \Mod{p^2}\).
%     \end{itemize}
% \end{itemize}

% \begin{example}
%     Find all solutions in \(\ZZ_9\) to 
%     \[f(x)=x^3+x^2+1 \equiv 0 \Mod{9}.\]
%     \begin{enumerate}
%         \item Solve \(f(x) \equiv 0 \Mod{3}\); by trial and error we have that \(x\equiv 1 \Mod{3}\).
%         \item By the (Some sort of map ) \([x]_9 \mapsto [x]_3\) we use trial and error to check
%         \begin{itemize}
%             \item \(x = 1 \then f(1) \not\equiv 0 \Mod{9}\);
%             \item \(x = 4 \then f(4) \equiv 0 \Mod{9}\);
%             \item \(x = 7 \then f(7) \not\equiv 0 \Mod{9}\).
%         \end{itemize}
%     \end{enumerate}
%     Therefore, the answer is \(x\equiv 4 \Mod{9}\).
% \end{example}

% \begin{theorem}
%     Let \(p\) be a prime and let \(f(x)=a_0+a_1x+\ldots+a_nx^n\) be a polynomial of degree \textbf{less than} \(n\) with integer coefficients. Suppose that \(a_i \not\equiv 0 \Mod{p}\) for some \(i\). Then the congruence relation
%     \[f(x) \equiv 0 \Mod{p}\]
%     has at most \(n\) solutions in \(\ZZ_p\)
% \end{theorem}

\begin{mdthm}[Hensel's Lemma]
    Let \(f(x)=a_0+a_1x+\ldots+a_nx^n\) be a polynomial with integer coefficients, let \(p\) be a prime and let \(r\) be a positive integer. We let \(f'(x)\) be the derivative of \(f(x)\) so, \(f'(x)=a_1+2a_2x+\ldots+na_n x^{n-1}\). Suppose \(x_r\) is an integer with
    \[f(x_r) \equiv 0 \Mod{p^r}\]
    and
    \[f'(x_r) \not\equiv 0 \Mod{p}.\]
    Then there exists \(x_{r+1}\in \ZZ\) satisfying
    \[f(x_{r+1}) \equiv 0 \Mod{p^{r+1}} \quad \text{and} \quad x_{r+1} \equiv x_r \Mod{p^r}\]
    Moreover, the \(x_{r+1}\) satisfying these properties is \textbf{unique} modulo \(p^{r+1}\), and we can take 
    \[x_{r+1} = x_r - uf(x_r)\]
    where \(u\) is an inverse of \(f'(x_r)\) modulo \(p\).
\end{mdthm}

\begin{example}
    How many solutions does
    \[f(x) = x^{10}+x^3+1 \equiv 0 \Mod{9}\]
    have? \\
    \textbf{Solution:}
    \begin{enumerate}
        \item Solve \(f(x) \equiv 0 \Mod{3} \then x\equiv 1 \Mod{3}\);
        \item Check if \(f'(1) \equiv 0 \Mod{3}\); we have that \(f'(1) =13 \not\equiv 0 \Mod{3}\). 
    \end{enumerate}
    Therefore, the conditions Hensel's lemma are met, so there is a solution which is \textbf{unique}. The congruence has only one solution.
\end{example}

\begin{mdexample}
    Let \(f(x)=x^2+x+5\). Find all solutions to \(f(x) \equiv 0 \Mod{11^2}\).\\
    \textbf{Solution:}
    \begin{enumerate}
        \item Solve \(f(x) \equiv 0 \Mod{11}\) by trial and error, so we have \(x=2,8 \Mod{11}\);
        \item for each solution \(x_1\) check if \(f'(x_1) \equiv 0 \Mod{11}\) i.e.
         \begin{itemize}
            \item \(x_1=2 \then f'(x_1) = 5 \not\equiv 0 \Mod{11}\);
            \item \(x_1=8 \then f'(x_1) = 17 \not\equiv 0 \Mod{11}\);
        \end{itemize}
        \item Find the multiplicative inverse, \(u\), to \(f'(x_1) \Mod{11}\) i.e. find \(u\) such that \(uf'(x_1) \equiv 1 \Mod{11}\):
            \begin{itemize}
                \item for \(x_1= 2\) we need to find \(u\) such that \(uf'(2) = 5u \equiv 1 \Mod{11}\) which implies \(u = -2\);
                \item for \(x_1 = 8\) we have \(u=2\)
            \end{itemize}
        \item Apply Hensel's lemma to \(x_1=2,8\) for which we have a formula: 
        \[\begin{aligned}
            x_2&=x_1-uf(x_1) \\
            \then x_1&=2 \then x_2 \equiv 24 \Mod{121}\\
            \then x_1 &=8 \then x_2 \equiv 96 \Mod{121}.
        \end{aligned}\]
    \end{enumerate}

    
\end{mdexample}

% \begin{example}
%     Solve \(f(x) \equiv 0 \Mod{p^{r+1}}\) in \(\ZZ_{p^{r+1}}\) where \(r\in \NN\) and \(p\) is prime. \\
%     \textbf{Strategy:}
%     \begin{itemize}
%         \item Solve \(f(x) \equiv 0 \Mod{p^r}\) in \(\ZZ_{p^r}\).
%         \item For each solution of \(x\) to the congruence above find all \([y]_{p^{r+1}} \in \ZZ_{p^{r+1}}\) with 
%         \begin{itemize}
%             \item \(y \equiv x \Mod{p^r}\);
%             \item \(f(y) \equiv 0 \Mod{p^{r+1}}\).
%         \end{itemize}
%     \end{itemize}
%     \textbf{Goal:} Find all \(t \Mod{p}\) with \(f(x_r+tp^n) \equiv 0 \Mod{p^{r+1}}\). \\
%     We have that \(y=x_r+tp^r\), where \(f(x_r) \equiv 0 \Mod{p^r}\).
% \end{example}

% \begin{mdnote}
%     Informal discussion: notice \(p^r  tp^r\) so in a sense \(x_r+tp^r\) and \(x^r\) are 'close' ('in divisibility by \(p\)') in \(\text{mod } p^{r+1}\). \\
%     Idea from calculus: if \(h\in \RR\) is small \(\abs{h}<\eps\) and  \(g : \RR \to \RR\) is a (nice) smooth function 
%     \[g(x+h)=g(x)+g'(x)h+\frac{g''(x)}{2!}h^2 +\cdots\]
%     since \(h\) is small we have that \(g(x+h) = g(x) + g'(x)h\)
% \end{mdnote}

\begin{lemma}
    For \(t\in \ZZ\) and a positive integer \(r\), we have
    \[f(x+tp^r) \equiv f(x)+tf'(x)p^r \Mod{p^{r+1}},\]
    where we view both sides as polynomials in \(x\), and we mean that all the coefficients of these two polynomial are congruent modulo \(p^{r+1}\).
\end{lemma}

\begin{mdthm}
    With regard to Hensel's lemma if \(f'(x_r) \equiv 0 \Mod{p}\) then each of the following holds:
    \begin{enumerate}
        \item if \(p^{r+1} \mid f(x_r)\) then \(f(x_r+tp^r) \equiv 0 \Mod{p^{r+1}}\) for each \(t \Mod{p}\) i.e. \(t \in \{1,2,\ldots, p\}\).
        \item If \(p^{r+1} \nmid f(x_r)\) then there are \textbf{no} solutions \(x_{r+1}\) to \(f(x) \equiv 0 \Mod{p^{r+1}}\) \textbf{with} \(x_{r+1} \equiv x_r \Mod{p^r} \then x_{r+1} = x_r +tp^r\).
    \end{enumerate}
\end{mdthm}

\begin{mdremark}
    In Case 1. If \textbf{ONE} of the \(t\in \{1,2,\ldots, p\}\) are roots of \(f(x) \equiv 0 \Mod{p}\) then \textbf{ALL} \(t \in \{1,2,\ldots, p\}\) are roots of the congruence.
\end{mdremark}

\pagebreak

\begin{mdnote}
    That is, suppose \(x_r\) is a solution to the congruence \(f(x)\equiv 0 \Mod{p^r}\) but, Hensel's lemma's condition are not satisfied i.e. \(f'(x_r) \equiv 0 \Mod{p}\). Then we need to compute \(f(x_r) \Mod{p^{r+1}}\):
    \begin{itemize}
        \item if \(f(x_r) \equiv 0 \Mod{p^{r+1}}\) then \(x_r+tp^r\) are solutions for all \(t \in \{1,2,\ldots p\}\);
        \item if \(f(x_r) \not\equiv 0 \Mod{p^{r+1}}\) then there are no solutions modulo \(p^{r+1}\).
    \end{itemize}
\end{mdnote}

\begin{example}
    Solve \(x^3+1 \equiv 0 \Mod{9}\).
    \begin{enumerate}
        \item Solve \(f(x)=x^3 +1 \equiv \Mod{3}\) by trial and error, which implies that \(x \equiv 2 \Mod{3}\);
        \item Check if \(3 \mid f'(2)\). We have that \(f'(2)=3\cdot 2^2 \equiv 0 \Mod{3}\). So, Hensel's lemma does not apply.
        \item Check if \(9 \mid f(2)\), we have that \(f(2) =9 \) so yes.
        \item We can conclude that this will have \(3\) solutions i.e. \(x_1=2 \then x_1+tp\) for \(t \in \{1,2,3\}\) which leads to \(x \equiv 2,5,8 \Mod{5}\).
    \end{enumerate}
\end{example}

\section{The structure of \texorpdfstring{\(\ZZ_m^{\times}\)}{TEXT}}

\subsection{Euler's \texorpdfstring{\(\phi\)}{TEXT} Function}

\begin{definition}
    Let \(m\) be a positive integer. We define \(\phi : \NN\to \NN\) given by \(\phi(m)\) to be the number of integers \(a\) such that \(1\leq a \leq m\) and \(\gcd(a,m)=1\) i.e. 
    \[\phi(m) = \abs{\left\{ 1 \leq a \leq m : \gcd(a,m)=1 \right\}}.\]
\end{definition}

\begin{mdnote}
    The \(\phi\) function tells us how many numbers are coprime to \(m\).
\end{mdnote}

\begin{theorem}
    Equivalently \(\phi(m)=\abs{\ZZ^{\times}_m}\), the cardinality of the multiplicative group \(\ZZ^{\times}_m\).
\end{theorem}

\begin{lemma}
    Let \(m,n\) be coprime positive integers then, \(\phi(mn)=\phi(m)\phi(n)\).
\end{lemma}

\begin{mdlemma}
    Let \(p\) be prime and \(n\) a positive integer. Then 
    \[
        \phi(p^n)=p^{n-1}(p-1) = p^n-p^{n-1}.
    \]
\end{mdlemma}

\begin{mdremark}
    Notice that \(\phi(p) = p-1\).
\end{mdremark}

\subsection{The Fermat-Euler theorem}

\begin{proposition}
    Let \(m\) be a positive integer then 
    \[
        \sum_{d \mid m} \phi(d)= F(m) =m
    \]
    for \(d>0\). Note that we are summing over positive divisors of \(m\).
\end{proposition}

\begin{mdnote}
    We can interpret \(F(m)\) as the sum of the \(\phi\) of all the positive divisors of \(m\).
\end{mdnote}

\begin{mdremark}
    If \(m,n\) are coprime then \(F(mn)=F(m)F(n)\)
\end{mdremark}

\begin{example}
    Find 
    \begin{itemize}
        \item \(F(p) = \sum_{d\mid p} = \phi(1)+\phi(p)=1+(p-1)=p\);
        \item \(F(p^2) = \sum_{d \mid p^2} =\phi(1)+\phi(p)+\phi(p^2) = 1+(p-1)+(p^2-p)=p^2\).
    \end{itemize}
\end{example}

\begin{mdthm}[Fermat-Euler theorem]
    Let \(a \in \ZZ\) and let \(m\) be a positive integer. Suppose \(\gcd(a,m) = 1\) then,
    \[
        a^{\phi(m)} \equiv 1 \Mod{m}.
    \]
\end{mdthm}

\begin{definition}
    Let \(m\in \NN\) and \(a\in \ZZ\) with \(\gcd(a,m)=1\). The \textbf{order} of \([a]_m\in \ZZ^{\times}_m\) is the smallest positive integer \(d\) with \([a]^d_m=[1]_m\).
\end{definition}

\begin{mdnote}
    Notation: \(o\left( [a]_m \right)\) means the order of \([a]_m\).
\end{mdnote}

\begin{corollary}
    By the Euler-Fermat theorem \(o([a]_m) \leq \phi(m)\) for \(\gcd(a,m)=1\). In particular the order of \([a]_m\) divides \(\phi(m)\) i.e. \(o([a]_m) \mid \phi(m)\).
\end{corollary}

\begin{example}
    Find the order of \(2 \Mod{9}\). \\
    \textbf{Solution:} We know the order of \([2]_9\) divides \(\phi(9)=6\). So, \(o([2]_9) \in \{1,2,3,6\}\) i.e. the divisors of \(6\). Note \([1]_9\) has order \(1\) so check 
    \begin{itemize}
        \item \(2^2 \equiv 4 \Mod{9}\);
        \item \(2^3 \equiv 8 \Mod{9}\).
    \end{itemize}
    Therefore, the order of \([2]_9\) is \(6\).
\end{example}

\begin{mdcor}[Fermat's Little theorem]
    Suppose that $p$ is a prime number and $a$ is an integer.
    \begin{enumerate}
        \item $a^p \equiv a \Mod{p};$
        \item if \(\gcd(a,p)=1\) then $a^{p-1} \equiv 1 \Mod{p}.$
    \end{enumerate}
\end{mdcor}

\begin{mdexample}
    What is \(10^{4035} \Mod{2017}\)? (\(2017\) is a prime number) \\
    \textbf{Solution:} Since \(2017\) is prime we can use Fermat's Little Theorem which implies \(10^{2016} \equiv 1 \Mod{2017}\). Since \(4035 = 2\cdot 2016 +3\) we have 
    \[\begin{aligned}
        10^{4035} &\equiv 10^{2\cdot 2016 +3} \Mod{2017} \\
                &\equiv \left( 10^{2016} \right)^2 \cdot 10^3 \Mod{2017}\\
                &\equiv 1\cdot 1000 \Mod{2017}.
    \end{aligned}\]
    Therefore, \(10^{4035} \Mod{2017}\) is \(1000 \Mod{2017}\).
\end{mdexample}

\subsection{Primitive roots}

The motivation behind this section is to find the positive integers \(m\) for which \(\ZZ_m^{\times}\) is a cyclic group.

\begin{definition}
    Let \(m\) be a positive integer. If \(g\) is an integer which is coprime to \(m\), such that the order of \(g\) modulo \(m\) is \(\phi(m)\). Then we say that \(g\) is a \textbf{primitive root} modulo \(m\).
\end{definition}

\begin{mdnote}
    We can reformulate this definition as: if \(\gcd(g,m)=1\) such that \(o([g]_m)=\phi(m)\) then we say \(g\) is a \textbf{primitive root} modulo \(m\).
\end{mdnote}

\begin{mdremark}
    By this definition if a primitive root exists within \(\ZZ_m^{\times}\) then it is a cyclic group because, the order of the primitive root is equal to the order of the group. Therefore, a primitive root is a generator of \(\ZZ_m^{\times}\).
\end{mdremark}

\begin{mdlemma}[Primitive Root Test]
    Let \(m\geq 3\) be a positive integer and let \(g\) be coprime to \(m\). Then \(g\) is a primitive root modulo \(m\) if and only if 
    \[g^{\frac{\phi(m)}{p}} \not\equiv 1 \Mod{m}\]
    for all prime divisors \(p\) of \(\phi(m)\) i.e. all the primes, \(p\), for which \(p\mid \phi(m)\).
\end{mdlemma}

\begin{mdnote}
    We are determining that the only possible choice for the order of \(g\) is \(\phi(m)\).
\end{mdnote}

\begin{example}
    Find a primitive root modulo \(7\). \\
    \textbf{Solution:} 
    \begin{enumerate}
        \item Compute \(\phi(7)=7-1=6\).
        \item Use trial and error, try \(g=2\), we have that the prime divisors of \(6\) are \(2\) and \(3\) so now with primitive root test:
        \begin{itemize}
            \item \(2^{\frac{6}{3}}= 2^2 \equiv 4 \Mod{7}\);
            \item \(2^{\frac{6}{2}} =2^3 \equiv 1 \Mod{7}\).
        \end{itemize}
        So, \(2\) is not a primitive root modulo \(7\).
        \item Try a different number, \(g=3\).
        \begin{itemize}
            \item \(3^{\frac{6}{3}}= 3^2 \equiv 2 \Mod{7}\);
            \item \(3^{\frac{6}{2}} =3^3 \equiv 6 \Mod{7}\).
        \end{itemize}
    \end{enumerate}
    So, the primitive root test implies that \(3\) \textbf{IS} a primitive root modulo \(7\).
\end{example}

\begin{lemma}
    Let \(p\) be a prime number. For \(d\mid (p-1)\) let 
    \[W_d =\{[a] \in \ZZ_p^{\times} : [a] \text{ has order } d\}\]
    and \(w_d = \abs{W_d}\). Then \(w_d \leq \phi(d)\), for each \(d \mid (p-1)\).
\end{lemma}

\begin{theorem}
    For each divisor of \(p-1\) i.e. \(d>0\) such that \(d\mid (p-1)\), there are \(\phi(d)\) elements of order \(d\) in \(\ZZ_p^{\times}\).
\end{theorem}

\begin{mdcor}
    There are \(\phi(p-1)\) primitive roots modulo \(p\).
\end{mdcor}

\begin{mdremark}
    Therefore, \(\ZZ_p^{\times}\) is cyclic as there are \(\phi(p-1)\) elements of \(p-1\) i.e. the primitive roots.
\end{mdremark}

\begin{mdcor}
    There always exists a primitive root modulo \(p\).
\end{mdcor}

\begin{example}
    How many primitive roots are there modulo \(23\)? \\
    \textbf{Solution:} There are \(\phi(23-1)=\phi(22)=\phi(2)\phi(11)=10\).
\end{example}

\begin{mdexample}
    Show there is no primitive root modulo \(15\). \\
    \textbf{Solution:} We have \(\phi(15)=\phi(3)\phi(5)=8\). Observe for any \(g\Mod{15}\), by the CRT, we have 
    \[\begin{aligned}
        g^d \equiv 1 \Mod{15} \iff g^d &\equiv 1 \Mod{3} \text{ and}\\
                            \iff g^d &\equiv 1 \Mod{5}.
    \end{aligned}\]
    By Fermat's Little Theorem we obtain \(g^4 \equiv1 \Mod{3}\) and \(g^4 \equiv 1 \Mod{5}\) so, \(g^4 \equiv 1 \Mod{15}\). Hence, the order of \(g\) is at most \(4\). We are done because for \(g\) to be a primitive root, it needs to have order \(8\), but it has only at most order \(4\).
\end{mdexample}

\begin{mdprop}
    If \(g\) is a primitive root modulo \(p\) then
    \[g^{\frac{(p-1)}{2}} \equiv -1 \Mod{p}.\]
\end{mdprop}

\begin{proof}
    Suppose \(g\) is a primitive root modulo \(p\). Let \(x = g^{\frac{p-1}{2}}\). Then 
    \[x^2 = g^{p-1} \equiv 1 \Mod{p}\]
    by Fermat's Little Theorem. Hence, \(x\equiv 1 \text{ or } -1 \Mod{p}\). Since, \(g\) is a primitive root, we know \(o([g]_p)=\phi(p)=p-1\) therefore 
    \[x \not\equiv 1 \Mod{p}.\]
    The only possibly choice is
    \[x = g^{\frac{p-1}{2}} \equiv -1 \Mod{p}.\]
\end{proof}

\subsection{Order of an element}

\begin{mdprop}
    For \(G\) a finite group \(g\in G\) and \(o(g)=d\) we have 
    \[o(g^k) = \frac{d}{\gcd(k,d)}.\]
\end{mdprop}

\begin{example}
    Show \(4\) is not a primitive root modulo \(m\) for \(m\geq 3\). \\ 
    \textbf{Solution:} Write \(d=o([2]_m)\) (assuming \(\gcd(m,2)=1\)).
    \begin{itemize}
        \item If \(d\) is even then \(4=2^2\) so 
        \[o([2^2]_m)=\frac{d}{\gcd(2,d)}=\frac{d}{2} \leq \frac{\phi(m)}{2} \leq \phi(m).\]
        \item If \(d\) is odd then
        \[o([4]_m)=o([2]_m)=d < \underbrace{\phi(m)}_{\text{even}}.\]
    \end{itemize}
\end{example}

\begin{mdprop}
    Suppose \(p\) and \(q\) are distinct prime numbers. Then the maximum order of an element in \(\ZZ_{pq}^{\times}\) is given by the \(\lcm(p-1,q-1)\).
\end{mdprop}

\subsection{Applications of primitive roots}

\begin{mdlemma}
    Let \(a \in \ZZ\) and \(m \in \NN\) with \(\gcd(a,m)=1\). Then \(a^n \equiv 1 \Mod{m}\) if and only if \(o([a]_m)\) divides \(n\).
\end{mdlemma}

\begin{mdremark}
    Reformulation of lemma from lecture notes: \\
    Let \(G\) be a finite group with identity element \(e\). Then for \(g\in G\) we have that \(g^n =e\) if and only if \(o(g)\) divides \(n\).
\end{mdremark}

\begin{mdnote}
    In practice this lemma will be used when \(G=\ZZ_m^{\times}\), in which case the lemma states: for \(a\in \ZZ\) with \(\gcd(a,m)=1\) we have that 
    \[([a]_m)^n =1 \iff o([a]_m) \mid n.\]
\end{mdnote}

\begin{mdexample}
    Find all solutions in \(\ZZ_{19}\) to 
    \[4x^5 \equiv 7 \Mod{19}.\]
    \textbf{Solution:}
    \begin{enumerate}
        \item Find a primitive root modulo \(19\). With trial and error in combination with primitive root test we have that \(2\) is a primitive root modulo \(19\).
        \item Since we know \(2\) is a primitive root we can write:
        \begin{itemize}
            \item \(x =2^i\) for some \(i\);
            \item \(4=2^2\);
            \item (by trial and error) \(7 \equiv 2^6 \Mod{19}\).
        \end{itemize}
        \item Now the original problem becomes 
            \[\begin{aligned}
                2^2 2^{5i} &\equiv 2^6 \Mod{19} \\
                2^{5i-4} &\equiv 1 \Mod{19}
            \end{aligned}\]
            Recall \(2\) is a primitive root modulo \(19\) so \(o([2]_{19}) =\phi(19)=18\) therefore, by the lemma above we have \(18 \mid 5i-4 \then 5i \equiv 4 \Mod{18}\).
        \item Solve \(5i \equiv 4 \Mod{18}\) so, \(i \equiv 8 \Mod{18}\) which implies \(i=8+18k\) for some \(k\in \ZZ\).
        \item Notice, \(2^{18k} \equiv 1 \Mod{19}\) so, \(2^{8+18k}  = 2^8 \cdot 2^{18k} \equiv 2^8 \cdot 1 \Mod{19}\).
        Therefore, by substituting \(i\) into \(x\) we have the solution 
        \[x =2^8 \equiv 9 \Mod{19}.\]
    \end{enumerate}
\end{mdexample}

\begin{example}
    Find all integer \(x\) with \(4^x \equiv 9 \Mod{19}\). \\
    \textbf{Solution:}
    \begin{enumerate}
        \item Find a primitive root modulo \(19\): we have \(2\) is a primitive root modulo \(19\).
        \item Write:
        \begin{itemize}
            \item \(4=2^2\);
            \item (by trial and error) \(9 \equiv 2^8 \Mod{19}\)
        \end{itemize}
        so, 
        \[2^{2x} \equiv 2^8 \Mod{19} \then 2^{2x-8} \equiv 1 \Mod{18}.\]
        \item Recall \(o([2]_{19}) = \phi(19)= 18\), by the lemma above we know, 
        \[\begin{aligned}
            &18 \mid (2x-8) \\
            &\then 2x \equiv 8 \Mod{18} \\
            &\then x \equiv 4 \Mod{9}
        \end{aligned}\]
        Our answer is therefore, \(x\equiv 4 \Mod{9}\).
    \end{enumerate}
\end{example}

\subsubsection{Primitive roots of prime powers}

\begin{mdprop}
    Let \(p\) be a prime. Suppose \(g\) is a primitive root modulo \(p\). Then \(g\) or \(g+p\) is a primitive root modulo \(p^2\).
\end{mdprop}

\begin{mdremark}
    The group \(\ZZ_{p^2}^{\times}\) is a cyclic group.
\end{mdremark}

\begin{mdnote}
    This proposition helps us 'lift' primitive roots to higher powers of \(p\).
\end{mdnote}

\begin{mdnote}
    To determine which of \(g\) or \(g+p\) is a primitive root modulo \(p^2\), we need to compute \(g^{p-1} \Mod{p^2}\). If \(g^{p-1} \equiv 1 \Mod{p^2}\) then \(g+p\) is a primitive root modulo \(p^2\), otherwise \(g\) is a primitive root modulo \(p^2\).\\
    \noindent Since \(g\) is a primitive root modulo \(p\) it has order \(\phi(p) = p-1\). Suppose \(g\) is a primitive root modulo \(p^2\), in this case \(g\) would have order \(\phi(p^2)\) therefore, if \(g^{p-1} \equiv 1 \Mod{p^2}\) then it \textbf{cannot} be a primitive root modulo \(p^2\) as this would imply \(g\) has order \(\phi(p) \neq \phi(p^2)\). Then it follows that \(g+p\) is the primitive root.
\end{mdnote}

\begin{mdexample}
    Find two primitive roots modulo \(25\). \\
    \textbf{Solution:} 
    \begin{enumerate}
        \item Find a primitive root modulo \(5\). By trial and error we have \(2\) is a primitive root modulo \(5\).
        \item Compute \(2^{5-1} \Mod{5^2}\).
        \begin{itemize}
            \item If \(2^{5-1} \equiv 1\Mod{5^2}\) then \(2+5 =7\) is a primitive root modulo \(25\).
            \item If \(2^{5-1} \not\equiv 1\Mod{5^2}\) then \(2\) is a primitive root modulo \(25\).
        \end{itemize}
        \item We have \(2^4 = 16 \not\equiv 1 \Mod{25}\). So, we conclude \(2\) is a primitive root modulo \(25\).
        \item Since \(2\) and \(7\) are primitive roots modulo \(5\) we can compute
        \[7^{5-1}=7^4 = 2401 \equiv 1 \Mod{25}.\]
        So we conclude \(7+5=12\) is a primitive root modulo \(25\).
    \end{enumerate}
\end{mdexample}

\begin{mdprop}
    Let \(p>2\) be a prime. Suppose \(g\) is a primitive root modulo \(p^2\) then \(g\) is a primitive root modulo \(p^n\) for all \(n\geq 2\).
\end{mdprop}

\begin{mdremark}
    The group \(\ZZ_{p^n}^{\times}\) is cyclic whenever \(p \neq 2\).
\end{mdremark}

\begin{mdprop}
    The group \(\ZZ_m^{\times}\) is cyclic if and only if \(m=1,2,4,p^n,2p^n\) for \(p>2\) and \(n\geq 1\).
\end{mdprop}

\begin{mdprop}
    Suppose \(m>0\) is a positive integer and suppose that \(\ZZ_m^{\times}\) has a primitive root. Then the number of primitive roots in \(\ZZ_m^{\times}\) is \(\phi(\phi(m))\).
\end{mdprop}

\subsection{Quadratic residues}

\begin{definition}
    Let \(p>2\) and \(b\in \ZZ\) with \(\gcd(b,p)=1\). We say that \(b\) is a \textbf{quadratic residue} (QR) modulo \(p\) if the equation 
    \[x^2 \equiv b \Mod{p}\]
    has a solution. Otherwise, we say that \(b\) is a \textbf{quadratic non-residue} (QNR) modulo \(p\).
\end{definition}

\begin{mdnote}
    We can think of quadratic residues as the 'square numbers' modulo \(p\).
\end{mdnote}

\begin{mdremark}
    If \(p\mid b\) then \(x\equiv 0 \Mod{p}\) is the only solution. Also, if \(p=2\) and \(a\) is odd then the only other possibility is \(b \equiv 1 \Mod{2}\). Therefore, from now on we assume that \(p\) is odd and \(b\) is coprime to \(p\).
\end{mdremark}

\begin{mdremark}
    In this course \(0\) is neither a quadratic residue nor a quadratic non-residue.
\end{mdremark}

% \begin{example}
%     We have
%     \begin{itemize}
%         \item \(-1\) is a QR modulo \(5,13,17\).
%         \item \(-1\) is a QNR modulo \(3,7,11\).
%     \end{itemize}
% \end{example}

\begin{corollary}
    If \(a \equiv b \Mod{p}\) then \(a\) is a QR modulo \(p\) if and only if \(b\) is a QR modulo \(p\).
\end{corollary}

\begin{mdexample}
    Find all QR modulo \(7\). \\
    \textbf{Solution:} We write an exhaustive table.
        \begin{table}[H]
            \begin{center}
                \begin{tabular}{c|c}
                    \(a \Mod{7}\) & \(a^2 \Mod{7}\) \\ \hline
                    \(1\)          &      \(1^2=1\)      \\
                    \(2\)          &      \(2^2=4\)      \\
                    \(3\)          &        \(3^2 \equiv 2\)    \\
                    \(4 \equiv -3\) &   \(2\)         \\
                    \(5 \equiv -2\) &      \(4\)      \\
                    \(6 \equiv -1\) &         \(1\)  
                    \end{tabular}
            \end{center}
        \end{table}
    \begin{mdremark}
        The QR are the numbers that we get on the RHS of the table.
    \end{mdremark}
    The QR modulo \(7\) are all the squares modulo \(7\) i.e. all the numbers that are equal to a square modulo \(7\). By looking at the right-hand column of the table we have all the numbers that satisfy such property. Therefore, 
    \begin{itemize}
        \item the QR are: \(1,2,4\) modulo \(7\);
        \item the QNR are \(3,5,6\) modulo \(7\).
    \end{itemize}
\end{mdexample}

% \begin{mdprop}
%     Let \(g\) be a primitive root modulo \(p\) and \(b \in \ZZ\) such that \(p \nmid b\). Suppose \(b\equiv g^i \Mod{p}\) is a QR if and only if \(i\) is even.
% \end{mdprop}

\begin{mdprop}
    Let \(g \in \ZZ\) be a primitive root modulo \(p\). Then \([g^k]_p\) is a quadratic residue if and only if \(k\) is even.
\end{mdprop}

\begin{corollary}
    There are \(\frac{(p-1)}{2}\) quadratic residues and \(\frac{(p-1)}{2}\) quadratic non-residues in \(\ZZ_p^{\times}\).
\end{corollary}

\begin{mdthm}
    We have \(-1\) is a quadratic residue modulo \(p\) if \(p \equiv 1 \Mod{4}\) and a quadratic non-residue if \(p \equiv 3 \Mod{4}\).
\end{mdthm}

\begin{proof}
    Let \(g\) be a primitive root modulo \(p\) and let \(x=g^{\frac{(p-1)}{2}}\). We have
    \[\begin{aligned}
        x^2 = g^{p-1} \equiv 1 \Mod{p}
    \end{aligned}\]
    and \(x \not\equiv 1 \Mod{p}\), since \(g\) is a primitive root. The equation \(x^2 \equiv 1 \Mod{p}\) has only two solutions, so we have \(x\equiv -1 \Mod{p}\). We deduce that \(-1\) is a quadratic residue if and only if \(\frac{p-1}{2}\) is even i.e. 
    \[\begin{aligned}
    \frac{p-1}{2} &\equiv 0 \Mod{2} \\
    \then p-1 &\equiv 0 \Mod{4} \\
    \then p &\equiv 1 \Mod{4}.
    \end{aligned}\]
\end{proof}

\section{Euler's criterion}

\begin{theorem}
    There are infinitely many primes, \(p\), with \(p \equiv 1 \Mod{4}\) i.e. primes of the form \(4k+1\).
\end{theorem}

\begin{proof}
    For sake of contradiction suppose \(p_1,p_2,\ldots,p_n\) are \textbf{all} the primes congruent to \(1\) modulo \(4\). Consider
    \[x=2p_1p_2\cdots p_n \quad \text{and} \quad N=x^2+1.\]
    Suppose \(p\mid N\), then \(x^2+1 \equiv 0 \Mod{p}\). Since \(x^2 \equiv -1 \Mod{p}\) then \(-1\) is a quadratic residue modulo \(p\) we have that \(p \equiv 1 \Mod{4}\) by Theorem \(6.2\). Hence, \(p \mid x\); by assumption, \(p\) must be one of the primes \(p_1,p_2,\ldots p_n\). This is a contradiction since \(q \nmid N-x^2 =1\).
\end{proof}

% \begin{mdthm}[Dirichlet's Theorem]
%     Let \(a,q\) be coprime positive integers. Then there exists infinitely many primes \(p\), with 
%     \[p \equiv a \Mod{q}.\]
% \end{mdthm}

\begin{mdthm}[Euler's Criterion]
    Let \(b \in \ZZ\) and \(p>2\) with \(\gcd(b,p)=1\). Then each of the following holds:
    \begin{enumerate}
        \item \(b\) is a \textbf{quadratic residue} if and only if
        \[b^{\frac{(p-1)}{2}}\ \equiv 1 \Mod{p}.\]
        \item \(b\) is a \textbf{quadratic non-residue} if and only if 
        \[b^{\frac{(p-1)}{2}}\ \equiv -1 \Mod{p}.\]
    \end{enumerate}
\end{mdthm}

\subsection{Application to solving \texorpdfstring{\(x^2 \equiv b \Mod{p}\)}{TEXT}}

\begin{mdprop}
    Suppose \(b\) is a quadratic residue modulo \(p\) and \(p \equiv 3 \Mod{4}\). Then 
    \[x_0 =b^{\frac{p+1}{4}}\]
    is a solution to \(x^2 \equiv b \Mod{p}\).
\end{mdprop}

\begin{mdexample}
    Given that \(5\) is a quadratic residue modulo \(139\). Find \textbf{all} solutions in \(\ZZ_{139}\) to
    \[x^2 \equiv 5 \Mod{139}.\]
    \textbf{Solution:} We have that \(139\) is prime and \(139 \equiv 3 \Mod{4}\) so take 
    \[x_0=5^{\frac{139+1}{4}} = 5^{35}.\]
    Now we compute \(5^{35} \Mod{139}\) (using the method of repeated squaring) and we have
    \[5^{35} \equiv 137 \Mod{139}.\]
    Notice \((x_0)^2 \equiv b \Mod{p} \iff (-x_0)^2 \equiv 0 \Mod{p}\). Therefore, our solutions are
    \[x\equiv 127 \Mod{139} \quad \text{and} \quad x\equiv -127 \equiv 12 \Mod{139}.\]
\end{mdexample}

\section{Legendre symbol}

\begin{definition}
    Let \(b\in \ZZ\) and \(p>2\). The \textbf{Legendre symbol}, \(\left( \frac{b}{p} \right)\) is given by 
    \[\begin{aligned}
        \left( \frac{b}{p} \right) =\begin{cases}
            1 &\text{if } b \text{ is a \textbf{quadratic residue} modulo } p \\
            0 &\text{if } b \mid p \\
            -1 &\text{if } b \text{ is a \textbf{quadratic non-residue} modulo } p.
        \end{cases}
    \end{aligned}\]
\end{definition}

\begin{mdremark}
    For each prime \(p>2\) we can think of the Legendre symbol as a function:
    \begin{itemize}
        \item \[\left( \frac{\cdot}{p} \right): \ZZ \to \{-1,0,1\} .\]
        \item \[\left( \frac{\cdot}{p} \right): \ZZ_p \to \{-1,0,1\} .\]
        \item \[\left( \frac{\cdot}{p} \right): \ZZ_p^{\times} \to \{-1,1\} .\]
    \end{itemize}
\end{mdremark}

\begin{proposition}
    Some properties of the Legendre symbol:
    \begin{itemize}
        \item \(\left( \frac{1}{p} \right) = 1\) \textbf{always} because \(1\) is a quadratic root modulo \(p\).
        \item \(\left( \frac{b^2}{p} \right) =1\) if \(p\nmid b\) because \(x^2 \equiv b^2 \Mod{p} \iff x \equiv \pm b \Mod{p}\).
        \item If \(a \equiv b \Mod{p}\) then
        \[\left( \frac{a}{p} \right) = \left( \frac{b}{p} \right),\]
        because \(a\) is a quadratic residue if and only if \(b\) is a quadratic residue.
    \end{itemize}
\end{proposition}

\begin{mdlemma}[Periodicity]
    The Legendre symbol is periodic i.e. 
    \[\left( \frac{a+dp}{p} \right) =\left( \frac{a}{p} \right).\]
\end{mdlemma}

\begin{mdexample}
    Compute \(\left( \frac{2022}{7} \right)\). \\
    \textbf{Solution:} We have \(2022 \equiv 6 \equiv -1 \Mod{7}\), which implies
    \[\left( \frac{2022}{7} \right) = \left( \frac{-1}{7} \right).\]
    Since \(7 \equiv 3 \Mod{4}\) then \(-1\) is a quadratic non-residue modulo \(7\). Therefore,
    \[\left( \frac{-1}{7} \right) =-1.\]
\end{mdexample}

\begin{mdlemma}
    Let \(b\in \ZZ\) and \(p>2\). The number of solutions in \(\ZZ_p\) to
    \[x^2 \equiv b \Mod{p}\]
    is equal to \(1+\left( \frac{b}{p} \right)\).
\end{mdlemma}

\begin{proof}
    We have three cases to consider.
    \begin{itemize}
        \item If \(p \mid b\) the only solution is \(x =[0]\), and \(1 = 1+0 = 1 + \left( \frac{b}{p} \right)\).
        \item If \(b\) is QNR then, by definition there are no solutions to the congruence hence,  we \(0=1-1=1+\left( \frac{b}{p} \right)\).
        \item If \(b\) is a QR, we have one solution \(x\) and another solution given by \((-x)\) since, it is an even polynomial thus, we have \(2 = 1 +1=1+\left( \frac{b}{p} \right)\) solutions.
    \end{itemize}
\end{proof}

\begin{mdexample}
    How many solutions does the equation 
    \[3x^2+6x+2 \equiv 0 \Mod{23}\]
    have in \(\ZZ_{23}\)?
    \begin{solution}
        We have \(3x^2+6x+2 = 3(x+1)^2-1\), so we have to solve 
        \[3(x+1)^2 \equiv 1 \Mod{23}.\]
        Note that, \([8]_{23}=[3]\inv_{23}\) thus, we are solving 
        \[(x+1)^2 \equiv 8 \Mod{23}.\]
        The question is now if, \(8\) is a QR modulo \(23\), which indeed it is. Hence, we have two solutions.
    \end{solution}
\end{mdexample}

\subsection{Properties of the Legendre symbol}

\begin{theorem}
    Reformulation of Euler's criterion with the Legendre symbol. Let \(b \in \ZZ\) and \(p>2\) with \(\gcd(b,p)=1\). Then we have
    \[b^{\frac{p-1}{2}} \equiv \left( \frac{b}{p} \right) \Mod{p}.\]
\end{theorem}

\begin{mdremark}
    This reformulation also holds true when \(b \mid p\) as both sides are \(0\) therefore, they are congruent modulo \(p\).
\end{mdremark}

\begin{lemma}[Multiplicative property]
    Let \(a,b\) be integers then
    \[\left( \frac{ab}{p} \right) = \left( \frac{a}{p} \right)\left( \frac{b}{p} \right).\]
\end{lemma}

\begin{mdremark}
    Reformulation of Euler's criterion in terms of the Legendre symbol. If \([a]\in \ZZ_p^{\times}\) then we have 
    \[a^{\frac{p-1}{2}} \equiv \left( \frac{a}{p} \right) \Mod{p}.\]
    If \(p \mid a\) we also have 
    \[a^{\frac{p-1}{2}} \equiv \left( \frac{a}{p} \right) \Mod{p},\]
    because in this case both sides are \(0 \Mod{p}\).
\end{mdremark}

\begin{lemma}[The rule for \(-1\)]
    Let \(p>2\). We have that
    \[\begin{aligned}
        \left( \frac{-1}{p} \right) = (-1)^{\frac{p-1}{2}} =\begin{cases}
            +1 &\text{if } p\equiv 1 \Mod{4} \\
            -1 &\text{if } p\equiv 3 \Mod{4}.
        \end{cases}
    \end{aligned}\]
\end{lemma}

\begin{proposition}[The rule of \(2\)]
    Let \(p>2\) then
    \[\begin{aligned}
        \left( \frac{2}{p} \right) =(-1)^{\frac{(p^2-1)}{8}} = \begin{cases}
            +1 &\text{if } p \equiv 1 \text{ or } 7 \Mod{8} \\
            -1 &\text{if } p \equiv 3 \text{ or } 5 \Mod{8}.
        \end{cases}
    \end{aligned}\]
\end{proposition}

\begin{mdexample}
    Compute \(\left( \frac{51}{53} \right)\). \\
    \textbf{Solution:} Observe \(51 \equiv -2 \Mod{53}\), \(53 \equiv 1 \Mod{4}\) and \(53 \equiv 3 \Mod{8}\) then by periodicity
    \[\begin{aligned}
        \left( \frac{51}{53} \right) &= \left( \frac{-2}{53} \right) \\
        &= \left( \frac{-1 \cdot 2}{53} \right) \\
        &= \left( \frac{-1}{53} \right)\left( \frac{2}{53} \right)\\
        &= 1 \cdot (-1) \\
        &= -1. 
    \end{aligned}\]
\end{mdexample}

\begin{theorem}
    Let \(p>2\) then
    \[\sum_{n=1}^{p-1} \left( \frac{n}{p} \right) =0.\]
\end{theorem}

\subsection{Quadratic reciprocity}

\begin{mdthm}[The Law of Quadratic reciprocity]
    Let \(p,q>2\) be two distinct primes. Then
    \[\begin{aligned}
        \left( \frac{p}{q} \right) = (-1)^{\frac{p-1}{2} \frac{q-1}{2}} \left( \frac{q}{p} \right) = 
        \begin{cases}
            \left( \frac{q}{p} \right) &\text{if } p \equiv 1 \Mod{4} \textbf{ or } q \equiv 1 \Mod{4} \\
            -\left( \frac{q}{p} \right) &\text{if } p \equiv q \equiv 3 \Mod{4}.
        \end{cases}
    \end{aligned}\]
\end{mdthm}

\begin{mdremark}
    Quadratic reciprocity is transformative, in the following way
    \[
        \underbrace{\left( \frac{p}{q} \right)}_{\text{Arithmetic in } \ZZ_q} = \underbrace{(-1)^{\frac{p-1}{2} \frac{q-1}{2}} \left( \frac{q}{p} \right)}_{\text{Arithmetic in } \ZZ_p}.
    \]
\end{mdremark}

\begin{example}
    Compute \(\left( \frac{5}{8171} \right)\) (\(8171\) is a prime). \\
    \textbf{Solution:} Using quadratic reciprocity:
    \[\begin{aligned}
        \left( \frac{5}{8171} \right) = \left( \frac{8171}{5} \right)
    \end{aligned}\]
    As \(5 \equiv 1 \Mod{4}\) we do not include the \(-1\). Then by periodicity
    \[\left( \frac{8171}{5} \right) = \left( \frac{1}{5} \right) =1.\]
    As \(8171 \equiv 1 \Mod{5}\).
\end{example}

\begin{mdexample}
    Show that \(\left( \frac{5}{p} \right) =1\) if and only if \(p \equiv 1 \text{ or } 4 \Mod{5}\). (That is, show that \(5\) is a quadratic residue modulo \(5\)). \\
    \textbf{Solution:} Notice, \(5 \equiv 1 \Mod{4}\). Observe by quadratic reciprocity
    \[\begin{aligned}
        1 &=\left( \frac{5}{p} \right) \\
        &= \left( \frac{p}{5} \right).
    \end{aligned}\]
    The statement holds if and only if \(p\) is a quadratic residue modulo \(5\). We then list the quadratic residues modulo \(5\).
    \begin{table}[H]
        \begin{center}
            \begin{tabular}{c|c}
                \(a \Mod{5}\) & \(a^2 \Mod{5}\) \\ \hline
                \(\pm 1\)          &      \(1\)      \\
                \(\pm 2\)          &      \(4\)      
                \end{tabular}
        \end{center}
    \end{table}
    Which means \(p \equiv 1 \Mod{5}\) or \(p\equiv 4 \Mod{5}\).
\end{mdexample}

\begin{example}
    Compute \(\left( \frac{21}{67} \right)\). \\
    \textbf{Solution:} Observe, \(67 \equiv 7 \equiv 3 \Mod{4}\), \(67 \equiv 4 \Mod{4}\) and \(67 \equiv 1 \Mod{3}\).
    \[\begin{aligned}
        \left( \frac{21}{67} \right) &= \left( \frac{3}{67} \right) \left( \frac{7}{67} \right) \\
        &= (-1)\left( \frac{67}{3} \right) (-1)\left( \frac{67}{7} \right) \\\
        &= (-1)\left( \frac{1}{3} \right) (-1)\left( \frac{4}{7} \right)\\
        &= (-1)(1)(-1)(1)\\
        &= 1.
    \end{aligned}\]
\end{example}

\begin{mdremark}
    General strategy for computing \(\left( \frac{a}{p} \right)\).
    \begin{enumerate}
        \item If \(\abs{a}>p\) use the periodicity rule.
        \item Factor \(a\) and then use the multiplicative rule.
        \item Apply quadratic reciprocity.
    \end{enumerate}
    Repeat the process if necessary.
\end{mdremark}

\begin{mdthm}
    If \(p,q>2\) and \(p\) and \(q\) are distinct primes then for \(b \in \NN\)
    \[\begin{aligned}
        \left( \frac{q^b}{p} \right) = \left( \frac{q}{p} \right)^b = \begin{cases}
            +1 &\text{if } b \text{ is even}\\
            \left( \frac{q}{p} \right) &\text{if } b \text{ is odd.}
        \end{cases}
    \end{aligned}\]
\end{mdthm}

\subsection{Rules for computing the Legendre symbol}

\begin{mdthm}
    Let \(p,q\) be distinct odd primes and \(a,b \in \ZZ\).
    \begin{enumerate}
        \item[R0.] Periodicity: \(\left( \frac{a}{p} \right) = \left( \frac{b}{p} \right)\) if \(a \equiv b \Mod{p}\).
        \item[R1.] Multiplicativity: \(\left( \frac{ab}{p} \right) =\left( \frac{a}{p} \right) \left( \frac{b}{p} \right)\).
        \item[R2.] Rule for \(2\):
        \[\begin{aligned}
            \left( \frac{2}{p} \right)  = \begin{cases}
                +1 &\text{ if } p\equiv 1 \text{ or } 7 \Mod{8} \\
                -1 &\text{ if } p\equiv 3 \text{ or } 5 \Mod{8}.
            \end{cases}
        \end{aligned}\]
        \item[R3.] Rule for \(-1\):
        \[\begin{aligned}
            \left( \frac{-1}{p} \right)  = 
            \begin{cases}
                1 & \text{if } p \equiv 1 \Mod{4} \\
                -1 & \text{if } p \equiv 3 \Mod{4}.
            \end{cases}
        \end{aligned}\]
        \item[R4.] Quadratic reciprocity:
        \[\begin{aligned}
            \left( \frac{p}{q} \right) = 
            \begin{cases}
                \left( \frac{q}{p} \right) &\text{if } p \equiv 1 \Mod{4} \textbf{ or } q \equiv 1 \Mod{4} \\
                -\left( \frac{q}{p} \right) &\text{if } p \equiv q \equiv 3 \Mod{4}.
            \end{cases}
        \end{aligned}\]
    \end{enumerate}
\end{mdthm}

\begin{mdthm}
    Let \(p\) be an odd prime. Given integers \(a,b,c\) with \(\gcd(1,p)=1\); the quadratic equation
    \[ax^2+bx+c \equiv 0 \Mod{p}\]
    has (in \(\ZZ_p\)):
    \begin{itemize}
        \item \(0\) solutions if \(b^2-4ac\) is a quadratic non-residue modulo \(p\).
        \item \(1\) solution if \(b^2-4ac \equiv 0 \Mod{p}\).
        \item \(2\) solutions \(b^2-4ac\) is a quadratic residue modulo \(p\).
    \end{itemize}
\end{mdthm}

\begin{mdexample}
    Determine the number of solutions to \(5x^2+2x+4 \equiv 0 \Mod{29}\). \\
    \textbf{Solution:} Consider the congruence equation \(ax^2 +bx +c \equiv 0 \Mod{p}\), if \(p \nmid a\) then the number of solutions is given by \(1+ \left( \frac{b^2-4ac}{p} \right)\). We are computing 
    \[1+\left( \frac{2^2 - 4(5)(4)}{2(5)} \right).\]
    We compute the Legendre symbol first
    \[\begin{aligned}
        \left( \frac{2^2 - 4(5)(4)}{2(5)} \right) &= \left( \frac{-76}{29} \right) \\
        &= \left( \frac{-1}{11} \right) \\
        &= -1.
    \end{aligned}\]
    Therefore, there are \(1-1\) solutions i.e. there are no solutions.
\end{mdexample}

\section{Gauss sums}

\begin{definition}
    An \(n^\text{th}\) \textbf{root of unity} is a complex number, \(z\), such that \(z^n =1\) for \(n \in \NN\).
\end{definition}

\begin{mdnote}
    Suppose \(z\in \CC\), the roots of unity are the solutions to \(z^n =1\). Now we write the number \(1\) in polar form
    \[\begin{aligned}
        z^n &=1 \\
        &= e^{2\pi k i} \\
        &=\cos(2\pi k)+i\sin(2\pi k). \\
    \end{aligned}\]
    Therefore, by De Moivre's theorem
    \[\begin{aligned}
        z &=1^{\frac{1}{n}} \\
        &= e^{\frac{2\pi k}{n} i} \\
        &= (\cos(2\pi k)+i\sin(2\pi k))^{\frac{1}{n}} \\
        &= \cos\left( \frac{2\pi k}{n} \right)+i\sin\left( \frac{2\pi k}{n} \right)
    \end{aligned}\]

\end{mdnote}

\begin{definition}
    We will define the notation.
    Given \(p>2\) and \(b \in \ZZ\) then
    \[e_p(b) := e^{\frac{2\pi b}{p}i} = \cos\left( \frac{2 \pi b}{p} \right)+i\sin\left( \frac{2\pi b}{p} \right).\]
\end{definition}

\begin{theorem}
    Properties of the roots of unity; let \(a,b \in \ZZ\).
    \begin{itemize}
        \item \(e_p(ab) = e_p(a)^b\).
        \item If \(a \equiv b \Mod{p}\) then \(e_p(a) =e_p(b)\).
        \item \(e_p(a)^p = e_p(ap) =e_p(0) = 1\). Therefore, \(e_p(a)\) is a \(p^{\text{th}}\) root of unity.
    \end{itemize}
\end{theorem}

\begin{definition}
    Let \(p>2\) be a prime and \(b \in \ZZ\). The \textbf{Gauss sum} associated to \(b\) modulo \(p\) is given by
    \[g_b = \sum_{n=1}^{p-1} \left( \frac{n}{p} \right) e_p(bn).\]
\end{definition}

\begin{example}
    If \(p=5\) and \(b=2\) then the Gauss sum of \(2\) modulo \(5\) is given by
    \[\begin{aligned}
        g_2 &= \sum_{n=1}^4 \left( \frac{n}{5} \right)e_5(2n) \\
            &= \left( \frac{1}{5} \right) e_5(2) +\left( \frac{2}{5} \right) e_5(4) +\left( \frac{3}{5} \right) e_5(6)+\left( \frac{4}{5} \right) e_5(8) \\
            &= (1)e_5(2)+ (-1) e_5(4)+ (-1) e_5(6)+ (1) e_5(8) \\
            &= e_5(2)- e_5(4)- e_5(1)+ e_5(3) \\
            &= -\sqrt{5}.
    \end{aligned}\]
\end{example}

\begin{mdprop}
    Let \(p>2\) and \(b \in \ZZ\) with \(\gcd(b,p)=1\). Then 
    \[g_b^2 = p (-1)^{\frac{p-1}{2}}.\]
\end{mdprop}

\begin{lemma}
    Let \(p>2\) and \(b \in \ZZ\). Then
    \[g_b =\left( \frac{b}{p} \right)g_1 .\]
\end{lemma}

\begin{lemma}
    Let \(m,m \in \ZZ\). Then 
    \[\begin{aligned}
        \sum_{b=0}^{p-1} e_p(b(m-n)) = \begin{cases}
        p &\text{ if } m\equiv n \Mod{p} \\
        0 &\text{ otherwise.}
        \end{cases}
    \end{aligned}\]
\end{lemma}

\subsection{Proof of quadratic reciprocity}

\subsubsection{Preliminaries}

\begin{definition}
    The set \(\ZZ[x]\) is the ring of polynomials with integer coefficients.
\end{definition}

\begin{definition}
    The set \(\ZZ[e_p]\) is defined as
    \[\begin{aligned}
        \ZZ[e_p]&=\{f(e_p) : f \in \ZZ[x]\} \\
                &=\{c_{p-1} e_p^{p-1} + c_{p-2}e_p^{p-2}+\cdots +c_1 e_p +c_0 : c_{p-1},\cdots, c_0 \in \ZZ\}.
    \end{aligned}\]
\end{definition}

\begin{mdremark}
    Let \(\alpha,\beta \in \ZZ[e_p]\) and \(q\) be a prime then,
    \[(\alpha+\beta)^q \equiv \alpha^q+\beta^q \Mod{q}.\]
\end{mdremark}

\begin{mdremark}
    From now on the notation \(e_p :=e_p(1)\).
\end{mdremark}

\begin{definition}
    Let \(\gamma, \alpha, \beta \in \ZZ[x]\). We say \(\alpha\) \textbf{divides} \(\beta\) if there exists \(\delta \in \ZZ[x]\) with \(\alpha = \delta \beta\).
\end{definition}

\begin{definition}
    We say \(\alpha\) is \textbf{congruent} to \(\beta\) modulo \(\gamma\) if \(\gamma \mid (\alpha-\beta)\).
\end{definition}

\begin{theorem}
    If \(p\) is prime and \(\alpha,\beta \in \ZZ[e_p]\) the 
    \[(\alpha+\beta)^p \equiv \alpha^p +\beta^p \Mod{p}.\]
\end{theorem}

% \subsubsection{Key lemmas}

% \begin{lemma}
%     Let \(p,q\) be distinct odd primes. Let \(P=(-1)^{\frac{q-1}{2}}\). Then 
%     \[g_1^q \equiv \left( \frac{P}{q} \right) g_1 \Mod{q}.\]
% \end{lemma}

% \begin{lemma}
%     Let \(p,q >2\) be distinct primes. Then
%     \[g_1^q \equiv \left( \frac{q}{p} \right) g_1 \Mod{q}.\]
% \end{lemma}
%%%%%%%%
%These key lemmas are proved in PART I and PART II respectively of the proof of quadratic reciprocity.

\subsubsection{The proof}

\begin{mdthm}[The Law of Quadratic reciprocity]
    Let \(p,q>2\) be two distinct primes. Then
    \[\begin{aligned}
        \left( \frac{p}{q} \right) = (-1)^{\frac{p-1}{2} \frac{q-1}{2}} \left( \frac{q}{p} \right) = 
        \begin{cases}
            \left( \frac{q}{p} \right) &\text{if } p \equiv 1 \Mod{4} \textbf{ or } q \equiv 1 \Mod{4} \\
            -\left( \frac{q}{p} \right) &\text{if } p \equiv q \equiv 3 \Mod{4}.
        \end{cases}
    \end{aligned}\]
\end{mdthm}

\begin{proof}
    Let \(g_1\) be the Gauss sum associated to \(1\) modulo \(p\) i.e. 
    \[g_1 = \sum_{n=1}^{p-1} \left( \frac{n}{p} \right)e_p(n).\]
    We will compute \(g_1^q\) in two different ways then combine the results.
    \begin{enumerate}
        \item[\textbf{PART I.}] Let \(P=p(-1)^{\frac{p-1}{2}}\), by Euler's criterion we have 
        \[P^{\frac{p-1}{2}} \equiv \left( \frac{P}{q} \right) \Mod{q}.\]
        By Proposition \(9.1\) and Lemma \(9.1\) we have
        \[\begin{aligned}
            g_1^{q-1} &= (g_1^2)^{\frac{q-1}{2}} \\
                    &= P^{\frac{q-1}{2}} \\
                    &\equiv \left( \frac{P}{q} \right) \Mod{q}
        \end{aligned}\]
        therefore,
        \[g_1^q \equiv g_1\left( \frac{P}{q} \right) \Mod{q}\]
        where the congruence is taken in \(\ZZ[e_p]\).
        \item[\textbf{PART II.}] Recall that if \(q\) is prime and \(\alpha, \beta \in \ZZ[e_p]\) then
        \[(\alpha+\beta)^q \equiv \alpha^q +\beta^q \Mod{q}.\]
        As such we have that
        \[\begin{aligned}
            g_1^q &= \left( \sum_{n=1}^{p-1} \left( \frac{n}{p} \right) e_p(n)\right)^q \\
            &= \sum_{n=1}^{p-1} \left( \frac{n}{p} \right)^q e_p(n)^q \Mod{q}.
        \end{aligned}\]
        Since \(q\) is odd then \(\left( \frac{n}{q} \right)^q =\left( \frac{n}{q} \right)\) we can write
        \[\begin{aligned}
            g_1^q &=\sum_{n=1}^{p-1} \left( \frac{n}{p} \right) e_p(qn) \Mod{q} \\
            &\equiv g_q \Mod{q}.
        \end{aligned}\]
        Recall \(g_b = \left( \frac{b}{p} \right) g_1\) so,
        \[g_q = \left( \frac{q}{p} \right)g_1\]
        which implies that
        \[g_1^q \equiv g_q \equiv \left( \frac{q}{p} \right)g_1 \Mod{q}.\]
        \item[\textbf{PART III.}] Now we combine the results from the previous parts,
        \[g_1^q \equiv  g_1\left( \frac{P}{q} \right) \equiv\left( \frac{q}{p} \right)g_1 \Mod{q}\]
        thus,
        \[g_1\left( \frac{P}{q} \right) \equiv\left( \frac{q}{p} \right)g_1 \Mod{q}.\]
        Multiplying by \(g_1\) on both sides we get 
        \[g_1^2\left( \frac{P}{q} \right) \equiv g_1^2\left( \frac{q}{p} \right) \Mod{q}.\]
        Since \(\gcd(q,P)=1\) we can cancel \(g_1^2=P\) from both sides of the congruence to get 
        \[\left( \frac{P}{q} \right) \equiv \left( \frac{q}{p} \right) \Mod{q}.\]
        Finally, since \(\left( \frac{P}{q} \right) , \left( \frac{q}{p} \right) \in \{-1,1\}\) we must have that \(\left( \frac{q}{p} \right) = \left( \frac{P}{q} \right)\). \\
        As \(P=p(-1)^{\frac{p-1}{2}}\) we conclude that 
        \[\begin{aligned}
            \left( \frac{q}{p} \right) &= \left( \frac{P}{q} \right) \\
            &= \left( \frac{p(-1)^{\frac{p-1}{2}}}{q} \right) \\
            &= \left(  \frac{(-1)^{\frac{p-1}{2}}}{q} \right) \left( \frac{p}{q} \right) \\
            &= \left( \frac{-1}{q} \right)^{\frac{p-1}{2}} \left( \frac{p}{q} \right) \\
            &= \left( (-1)^{\frac{q-1}{2}} \right)^{\frac{p-1}{2}} \left( \frac{p}{q} \right) \\
            &= (-1)^{\frac{p-1}{2} \frac{q-1}{2}} \left( \frac{p}{q} \right),
        \end{aligned}\]
        as desired.
    \end{enumerate}
\end{proof}

\section{Sum of two squares}

\begin{definition}
    An integer \(m \in \NN\) is a \textbf{sum of two squares} if \(m=a^2+b^2\) for some \(a,b\in \ZZ\).
\end{definition}

\begin{mdremark}
    In this definition we allow for \(a\) and \(b\) to be zero. Thus, perfect squares are also sums of two squares.
\end{mdremark}

\begin{definition}
    The \textbf{Gaussian integers} is the ring \(\ZZ[i] = \{a+bi : a,b \in \ZZ\}\).
\end{definition}

\begin{theorem}
    The \textbf{units} in \(\ZZ[i]\) are: \(\pm 1\) and \(\pm i\).
\end{theorem}

\begin{theorem}
    Let \(\alpha, \beta \in \ZZ[i]\), we say \(\alpha\) \textbf{divides} \(\beta\) and, we write \(\alpha \mid \beta\) if there exists a \(\gamma \in \ZZ[i]\) with \(\beta =\alpha\gamma\).
\end{theorem}

\begin{definition}
    A \textbf{Gaussian prime} is a Gaussian integer \(\mathfrak{p} \in \ZZ[i]\) such that \(\mathfrak{p} \neq 0,\pm 1, \pm i\) and if \(\mathfrak{p} \mid \alpha \beta\) for \(\alpha, \beta \in \ZZ[i]\) then \(\mathfrak{p} \mid \alpha\) or \(\mathfrak{p} \mid \beta\).
\end{definition}

\begin{mdremark}
    Since \(\ZZ \subset \ZZ[i]\) we can deduce whether primes in \(\ZZ\) are Gaussian primes. If a prime, \(p\), is a sum of two squares i.e. \(p^2 =a^2+b^2\) then we can factor \(p=(a+ib)(a-ib)\) in \(\ZZ[i]\). So, \(p\) will not be a Gaussian prime. Conversely, if \(p \in \ZZ\) is not a sum of two squares then \(p\) is a Gaussian prime.
\end{mdremark}

\begin{proposition}
    A positive integer, \(m\), is a square if and only if every exponent \(a_i\) in the prime factorisation \(m=p_1^{a_1}p_2^{a_2} \cdots p_r^{a_r}\) is even.
\end{proposition}

\begin{mdlemma}
    Suppose \(m\in \ZZ\) is a sum of two squares i.e. \(m=a^2+b^2\) with \(a,b\in \ZZ\). Then \(m \equiv 0,1, \text{ or } 2 \Mod{4}\).
\end{mdlemma}

\begin{proof}
    If \(x\in \ZZ\) then \(x^2\) is either \(0\) or \(1\) modulo \(4\).
\end{proof}

\begin{mdcor}
    If \(m \equiv 3 \Mod{4}\) then \(m\) is not a sum of two squares.
\end{mdcor}

\begin{lemma}
    Let \(m \in \ZZ\), then \(m\) is a sum of two squares if and only if \(m=\abs{\alpha}^2\) for some \(\alpha\in \ZZ[i]\).
\end{lemma}

\begin{proof}
    \hphantom{Space}
    \begin{itemize}
        \item Proof of \((\then)\).
        Suppose \(m = a^2+b^2\) then \(m=(a+ib)(a-ib) = \abs{a+ib}^2\).
        \item Proof of \((\Leftarrow)\).
        If \(n=\abs{\alpha}^2\) for \(\alpha =a+ib \in \ZZ[i]\) then \(n=a^2+b^2\).
    \end{itemize}
\end{proof}

\begin{mdthm}
    Let \(m,n \in \ZZ\). If \(m\) and \(n\) are sums of two squares so is \(mn\). We can write \(m = a^2 +b^2\) and \(n = c^2+d^2\) then \(mn=(ac-bd)^2+(ad+bc)^2\).
\end{mdthm}

\begin{proof}
    Write \(m=\abs{\alpha}^2\) and \(n=\abs{\beta}^2\) for \(\alpha, \beta \in \ZZ[i]\). Then \(mn =\abs{\alpha}^2 \abs{\beta}^2=\abs{\alpha \beta}^2 \in \ZZ[i]\). So, by the previous lemma \(mn\) is a sum of two squares. Furthermore, we can write \(m=a^2+b^2=(a+ib)(a-ib)\) and \(n=c^2+d^2=(c+id)(c-id)\) as such, 
    \[\begin{aligned}
        mn &= (a+ib)(a-ib)(c+id)(c-id) \\
            &= (a+ib)(c+id) (a-ib)(c-id) \\
            &= [(ac-bd) +i(ad+bc)][(ac-bd)-i(ad+bc)] \\
            &= (ac-bd)^2+(ad+bc)^2.
    \end{aligned}\]
\end{proof}

\begin{mdexample}
    Write \(1313 = 13 \cdot 101\) as a sum of two squares. \\
    \textbf{Solution:} Notice that
    \begin{itemize}
        \item \(13 =2^2+3^2\);
        \item \(101 = 10^2+1^2\).
    \end{itemize}
    Therefore, we can write
    \[\begin{aligned}
        1313 &= (2\cdot 10 -3\cdot 1)^2+(2\cdot 1 + 3 \cdot 10)^2 \\
        &= 17^2+32^2.
    \end{aligned}\]
\end{mdexample}

\subsection{The two squares theorem}

\begin{theorem}[Pigeon hole principle]
    If \(m\) objects are distributed into \(n\) containers and \(m>n\) then \textbf{at least} one container contains more than \(2\) objects.
\end{theorem}

\begin{example}
    Let \(n \in \NN\) and \(S \subseteq \ZZ\) with \(\abs{S}=m>n\). There exists \(a,b \in S\) with \(a \neq b\) and \(a \equiv b \Mod{n}\).
\end{example}

\begin{lemma}
    Let \(a,n \in \ZZ\) with \(n>1\) and \(n\) is not equal to a square number. Then there exists \((c_1,d_1), (c_2,d_2)\in \ZZ \times \ZZ\) with \(0 \leq c_i,d_i < \sqrt{n}\) for \(i=1,2\) such that:
    \begin{itemize}
        \item \(c_1+d_1a \neq c_2+ad_2\);
        \item \(c_1+ad_1 \equiv c_2+ad_2  \Mod{n}\).
    \end{itemize}
\end{lemma}

\begin{mdthm}
    Let \(p\) be a prime. Then \(p\) is a sum of two squares if and only if \(p=2\) or \(p\equiv 1 \Mod{4}\).
\end{mdthm}

\begin{mdthm}[The two squares theorem]
    An integer \(n \in \NN\) is a sum of two squares if and only if the exponent of every prime number which is congruent to \(3\) modulo \(4\) in the prime factorisation of \(n\) is even.
\end{mdthm}

\section{Irrational numbers}

\begin{definition}
    We use \(\RR\) to denote the real numbers, \(\CC\) the complex numbers and \(\QQ =\left\{\frac{a}{b} : a \in \ZZ, b\in \NN\right\}\).
\end{definition}

\begin{definition}
    An \textbf{irrational number} is a complex number \(z \in \CC\) such that \(z \not\in \QQ\).
\end{definition}

\begin{theorem}
    A real number \(x \in \RR\) is rational if and only if its decimal expansion either terminates or repeats.
\end{theorem}

\begin{mdprop}
    Let \(z \in \CC\) be a root of a polynomial \(x^m +c_{m-1}x^{m-1}+\cdots + c_1x+c_0\) with integer coefficients \(c_i \in \ZZ\). Then \(z\) is an integer or \(z\) is irrational.
\end{mdprop}

\begin{mdremark}
    We need the polynomial \(f(x)\) to be monic i.e. the leading coefficient is \(1\).
\end{mdremark}

\begin{theorem}
    The number \(e\) is irrational.
\end{theorem}

\subsection{Algebraic and transcendental numbers}

\begin{definition}
    A complex number \(z \in \CC\) is called \textbf{algebraic} if \(z\) is a root of a non-zero polynomial with rational coefficients.
\end{definition}

\begin{definition}
    A complex number \(z\) is called \textbf{transcendental} if it is not algebraic.
\end{definition}

\begin{example}
    \hphantom{SPACE}
    \begin{itemize}
        \item \(\frac{a}{b} \in \QQ\) is algebraic as it is root of \(x-\frac{a}{b}\).
        \item \(\sqrt{2}, \sqrt[3]{5}\) and \(\sqrt[d]{p}\) are all algebraic for \(d\geq 2\) and \(p\) is prime.
        \item \(\pi\) and \(e\) are transcendental.
    \end{itemize}
\end{example}

\begin{mdexample}
    Let \(\alpha = \sqrt{7}+\sqrt{5}\). Find integers \(c_0,c_1,c_2,c_3\) such that \(\alpha\) is a root of \(f(x)=x^4+c_3 x^3+c_2 x^2 +c_1 x+c_0\). \\
    \textbf{Solution:} We know \(\alpha =\sqrt{7} + \sqrt{5}\) which can be rewritten as \(\alpha -\sqrt{7} = \sqrt{5}\) so,
    \[\begin{aligned}
        (\alpha -\sqrt{7})^2 &= (\sqrt{5})^2 \\
        \alpha^2 -2\alpha\sqrt{7} +7 &= 5.
    \end{aligned}\]
    Which can be rewritten as \(\alpha^2+2 = 2\alpha\sqrt{7}\). By squaring both sides
    \[\begin{aligned}
        (\alpha^2+2)^2 &= (2\alpha\sqrt{7})^2 \\
        \alpha^4 - 4\alpha^2+4 &= 28\alpha^2 \\
        \alpha^4 -24\alpha^2+ 4 &=0.
    \end{aligned}\]
    Therefore, the coefficients are
    \[\begin{aligned}
        c_0 &= 4 \\ 
        c_1 &= c_3 = 0 \\
        c_2 &= -24.
    \end{aligned}\]
\end{mdexample}

\begin{mdthm}[Dirichlet's approximation theorem]
    Let \(\alpha \in \RR\) and \(n \geq 1\) be an integer. Then there exists \(\frac{a}{b} \in \QQ\) with \(a \in \ZZ\) and \(1 \leq b \leq n\) such that 
    \[\abs{\alpha - \frac{a}{b}}<\frac{1}{bn}.\]
\end{mdthm}

\begin{corollary}
    Suppose \(\alpha \in \RR\) is irrational. Then there exists infinitely many (distinct) rational numbers \(\frac{a}{b}\) such that
    \[\abs{\alpha - \frac{a}{b}} < \frac{1}{b^2}.\]
\end{corollary}

\begin{definition}
    \textbf{Notation:} For \(\alpha \in \RR\) let 
    \[\alpha =N(\alpha)+F(\alpha)\]
    where \(N(\alpha)\) is an integer \(0 \leq F(\alpha)<1\), where \(F(\alpha)\) is called the \textbf{fractional part} of \(\alpha\) and \(N(\alpha)\) the \textbf{integer part} of \(\alpha\).
\end{definition}

\begin{example}
    \(\pi =N(\pi)+F(\pi)\) with \(N(\pi)=3\) and \(F(\pi)=0.14159 \ldots\)
\end{example}

\section{Liouville's Theorem}

\begin{mdthm}[Liouville's Theorem]
    Let \(\alpha \in \RR\) be an irrational number which is a root of a polynomial
    \[f(x) = c_m x^m+ c_{m-1}x^{m-1} + \cdots + c_1 x+c_0\]
    with \(c_i \in \QQ\) and \(c_m \neq 0\). Then there exists a real number \(C >0\) such that for all \(a \in \ZZ\) and \(b\in \NN\) we have
    \[\abs{\alpha -\frac{a}{b}} >\frac{C}{b^m}.\]
\end{mdthm}

\begin{mdnote}
    The degree of \(b\) in the inequality is the degree of the polynomial \(f\).
\end{mdnote}

\begin{mdthm}
    The number \(\sqrt{2}\) is irrational. Especially, we have 
    \[\abs{\sqrt{2}-\frac{a}{b}} >\frac{C}{b^2}.\]
\end{mdthm}

\begin{proof}
    The proof of Liouville's theorem for the case \(\alpha=\sqrt{2}\).\\
    We will prove that for all \(a \in \ZZ\) and \(b \in \NN\) we have
    \[\abs{\sqrt{2}-\frac{a}{b}} > \frac{1}{b^2} \underbrace{\frac{1}{1+2\sqrt{2}}}_{C}.\]
    \begin{itemize}
        \item \textbf{Case 1.} If \(\abs{\sqrt{2}-\frac{a}{b}}<1\) then we consider the polynomial \(f(x)=x^2-2=(x-\sqrt{2})(x+\sqrt{2})\) to find bounds for \(\abs{f(\frac{a}{b})}=\abs{\left( \frac{a}{b} - \sqrt{2} \right) \left( \frac{a}{b} + \sqrt{2}\right)}\). \\
        (We do this because we want to bound \(\abs{\frac{a}{b} - \sqrt{2}}=\abs{\sqrt{2}-\frac{a}{b}}\)).

        \begin{itemize}
            \item \textbf{Upper bound} of \(\abs{f\left( \frac{a}{b} \right)}\). Note that by the triangle inequality we have
            \[\begin{aligned}
                \abs{\frac{a}{b}+\sqrt{2}} &=\abs{\frac{a}{b}-\sqrt{2}+\sqrt{2}+\sqrt{2}}  \\
                &\leq \abs{\frac{a}{b}-\sqrt{2}}+\abs{\sqrt{2}+\sqrt{2}} \\
                &\leq 1+2\sqrt{2}.
            \end{aligned}\]
            Therefore,
            \[\abs{f\left( \frac{a}{b} \right)} \leq \abs{\frac{a}{b}-\sqrt{2}} \left( 1+2\sqrt{2} \right).\]
            \item \textbf{Lower bound} of \(\abs{f\left( \frac{a}{b} \right)}\). We have 
            \[\begin{aligned}
                \abs{f\left( \frac{a}{b} \right)} &=\abs{\left( \frac{a}{b} \right)^2 -2} \\
                &= \abs{\frac{a^2-2b^2}{b^2}}.
            \end{aligned}\]
            Since, \(a^2-2b^2 \in \ZZ\) and \(a^2-2b^2 \neq 0\) we have that \(\abs{a^2-2b^2}\geq 1\) hence,
            \[\abs{f\left( \frac{a}{b} \right)} \geq 1.\]
        \end{itemize}
        By combining the bounds we have
        \[\frac{1}{b^2} \leq \abs{f\left( \frac{a}{b} \right)} \leq \abs{\frac{a}{b}-\sqrt{2}}\left( 1+2\sqrt{2} \right),\]
        which implies
        \[\abs{\sqrt{2}-\frac{a}{b}} > \frac{1}{\left( 1+2\sqrt{2} \right)b^2}.\]

        \item \textbf{Case 2.} If \(\abs{\sqrt{2}-\frac{a}{b}} \geq 1\) then we clearly have 
        \[\abs{\sqrt{2}-\frac{a}{b}} > \frac{1}{\left( 1+2\sqrt{2} \right)b^2}\]
        as well (since \(b \geq 1\)).
    \end{itemize}
    Therefore, we have 
    \[\abs{\sqrt{2}-\frac{a}{b}} > \frac{1}{\left( 1+2\sqrt{2} \right)b^2}\]
    in all cases. In this case we take \(C=\frac{1}{1+2\sqrt{2}}\) in the statement of Liouville's theorem.
\end{proof}

\begin{mdnote}
    By varying \(C\) the inequality can switch from \(>\) to \(\geq\) and vice versa.
\end{mdnote}

\begin{mdcor}
    Let \(\alpha \in \RR\) be an irrational number as in Liouville's theorem. Suppose we have a real number \(\eps>0\), then the inequality
    \[\abs{\alpha-\frac{a}{b}} < \frac{1}{b^{m+\eps}}\]
    holds for only finitely many \(a \in \ZZ\) and \(b \in \NN\).
\end{mdcor}

\begin{mdexample}
    The above corollary  shows that thre exist pnly finite many \(a,b\) such that \(\abs{\sqrt{2}-\frac{a}{b}} \leq \frac{1}{b^3}\). We illustrate how to find them. \\
    We have 
    \[\abs{\sqrt{2}-\frac{a}{b}} > \frac{1}{(1+2\sqrt{2})b^2}\]
    so if, \(\abs{\sqrt{2}-\frac{a}{b}} \leq \frac{1}{b^3}\) then, we have 
    \[\frac{1}{b^3} > \frac{1}{(1+2\sqrt{2})b^2}\]
    which implies \(b<1+2\sqrt{2}\). Since, \(b \in \NN\), we deduce that \(b=1,2,3\).
    \begin{itemize}
        \item If \(b=3\), the inequality is \(\abs{\frac{a}{3}-\sqrt{2}} \leq \frac{1}{27}\) which implies that \(\abs{3\sqrt{2}-a} \leq \frac{1}{9}\). Since, \(3\sqrt{2} \approx 4.24\), there are no integers within the range \(\frac{1}{9}\) so, there are no \(a\)'s satisfying the inequality.
        \item If \(b=2\) the inequality is \(\abs{\frac{a}{2}-\sqrt{2}}\leq \frac{1}{8}\) which implies that \(\abs{2\sqrt{2}-a}\leq \frac{1}{4}\). We get one solution, \(a=3\).
        \item If \(b=1\) we get \(a=1,2\).
    \end{itemize}
\end{mdexample}

\begin{mdprop}
    The number \(\alpha=\sum_{n=1}^{\infty} \frac{1}{10^{n!}}\) is transcendental.
\end{mdprop}

\begin{proof}
    Suppose for the sake of contradiction that \(\alpha\) is a root of a polynomial of degree \(m\) with rational coefficients, i.e. \(\alpha\) is algebraic. By Liouville's theorem we know there exists a real number \(C>0\) such that 
    \[\abs{\alpha-\frac{a}{b}} > \frac{C}{b^m}\]
    for all \(a\in \ZZ\) and \(b\in \NN\). 
    To approximate \(\alpha\) by rational number, consider the finite sum
    \[\alpha_k = \sum_{n=1}^k \frac{1}{10^{n!}},\]
    which has denominator of \(10^{k!}\). Therefore, we have 
    \[\abs{\alpha-\alpha_k} = \sum_{n=k+1}^{\infty} \frac{1}{10^{n!}}.\]
    By considering the decimal expansion of 
    \[\begin{aligned}
        \sum_{n=k+1}^{\infty} \frac{1}{10^{n!}} &= \frac{1}{10^2}+\frac{1}{10^6}+\frac{1}{10^{24}} +\ldots\\
        &= 0.01+0.000001+0.\underbrace{0\ldots01}_{24} +\ldots \\
        &= 0.0\underset{2^{\text{th}}}{1}000\underset{6^{\text{th}}}{1}0\ldots0\underset{24^{\text{th}}}{1}\ldots \\
    \end{aligned}\]
    Generally, the decimal expansion of \(\sum_{n=k+1}^{\infty} \frac{1}{10^{n!}}\) takes the form of 
    \[\begin{aligned}
        \sum_{n=k+1}^{\infty} \frac{1}{10^{n!}} &= 0.0\ldots0 \underset{(k+1)!^{\text{th}}}{1} 0 \ldots 0\underset{(k+2)!^{\text{th}}}{1} \ldots \\  
        &< 0.0\ldots0 \underset{(k+1)!^{\text{th}}}{2} \\
        &= \frac{2}{10^{(k+1)!}}.
    \end{aligned}\] 
    Therefore, 
    \[\abs{\alpha-\alpha_k} = \sum_{n=k+1}^{\infty} \frac{1}{10^{n!}}<\frac{2}{10^{(k+1)!}}.\]
    By taking \(k\) large enough, we can make \(\frac{2}{10^{(k+1)!}}= \frac{2}{\left( 10^{k!} \right)^{k+1}}\) less than \(\frac{C}{\left( 10^{k!} \right)^m}\), which contradicts Liouville's theorem. Hence, \(\alpha\) is transcendental.
\end{proof}

\section{Pythagorean triples}

\begin{definition}
    We say \((x,y,z) \in \NN\) is a \textbf{Pythagorean triple} if \(x^2+y^2=z^2\).
\end{definition}

\begin{definition}
    A Pythagorean triple,\((x,y,z)\), is called \textbf{primitive} if \(\gcd(x,y,z) =1\).
\end{definition}

\begin{lemma}
    Suppose \((x,y,z)\) is a primitive Pythagorean triple. Then any two of three integers \((x,y,z)\) are coprime.
\end{lemma}

\begin{lemma}
    If \((x,y,z)\) is a primitive Pythagorean triple the one of \(x,y\) is even and the other is odd.
\end{lemma}

\begin{theorem}
    Let \(n \in \NN\) then \(n\) is a square (i.e. \(n=c^2\) for \(c \in \NN\)) if and only if in its prime factorisation each prime appears to an even power.
\end{theorem}

\begin{lemma}
    Suppose \(\gcd(a,b)=1\) for \(a,b \in \NN\) and \(ab=c^2\) for some \(c\in \NN\). Then \(a\) and \(b\) are both squares.
\end{lemma}

\begin{mdthm}[Pythagorean triples theorem]
    All primitive Pythagorean triples, \((x,y,z)\), with \(x\) even, are given by the formulas:
    \[\begin{aligned}
        x&=2st \\
        y&=s^2-t^2 \\
        z&=s^2+t^2
    \end{aligned}\]
    for integers
    \begin{enumerate}
        \item[(i)] \(s>t>0\);
        \item[(ii)] \(\gcd(s,t)=1\);
        \item[(iii)] \(s \not\equiv t \Mod{2}\).
    \end{enumerate}
    To get all Pythagorean triples (up to swapping \(x\) and \(y\)) we take integers \(s\) and \(t\) as above and \(d\) another positive integer and consider 
    \[\begin{aligned}
        x&=2dst \\
        y&=d(s^2-t^2) \\
        z&= d(s^2+t^2).
    \end{aligned}\]
\end{mdthm}

\begin{mdremark}
    The theorem implies that there is a bijection between primitive \\ Pythagorean triples, \((x,y,z)\) and \((s,t) \in \NN\) which satisfy (i), (ii) and (iii).
\end{mdremark}

\begin{example}
    Find all primitive Pythagorean triples, \((x,y,z)\) with \(z=x+3\). \\
    \textbf{Solution:} Write \(x=2st\) and \(z=s^2+t^2\). So,
    \[\begin{aligned}
        s^2+t^2 =2st+3 \iff &s^2-2st+t^3 = 3 \\
                        \then &(s-t)^2 =3. 
    \end{aligned}\]
    Which has no solutions as \(3\) is not a perfect square. Therefore, there are no primitive Pythagorean triples with \(z=x+3\) and \(x\) being even.
\end{example}

\begin{example}
    Find all primitive Pythagorean triples, \((x,y,z)\) with \(x\) being even and \(z=y+2\). \\
    \textbf{Solution:} Write \(x=2st, y=s^2-t^2\) and \(z=s^2+t^2\).
    \[\begin{aligned}
        s^2+t^2=s^2-t^2+2 \then 2t^2&=2 \\
                        \then t&=1.
    \end{aligned}\]
    Since \(s\not\equiv t \Mod{2}, s>t,\gcd(s,t)=1\) and \(t=1\), we have that \(s\) can be any positive even number. \\
    Write, \(s =2k\) for \(k\in \NN\) and \(t=1\).
    \[\begin{aligned}
        (x,y,z) &= (4k, (2k)^2-1, (2k^2)+1) \\
                &= (4k, 4k^2-1,4k^2+1),
    \end{aligned}\]
    with \(b\in \NN\) which satisfy \(z=y+2\).
\end{example}

\begin{mdexample}[Exam 2022]
    Find all primitive Pythagorean triples with \(x = 88\). \\
    \textbf{Solution:} By the Pythagorean triples theorem we can write
    \[\begin{aligned}
        x= 88 &=2st \\
        \then st &=44
    \end{aligned}\]
    for \(s,t \in \NN\). Since \(s>t\) by property (i) we have
    \begin{itemize}
        \item \(s=44 ,t=1\);
        \item \(s=22, t=2\);
        \item \(s=11, t=4\).
    \end{itemize}
    Now we need to check the remaining properties:
    \begin{table}[H]
        \begin{center}
            \begin{tabular}{c|c|c}
                \((s,t)\) & \(\gcd(s,t)\) & \(s\not\equiv t \Mod{2}\) \\ \hline
                \((44,1)\)          &      \(1\)    & \checkmark  \\
                \((22,2)\)          &      \(2\)      & \(\times\) \\
                \((11,4)\)          &        \(1\)    & \checkmark\\
                \end{tabular}
        \end{center}
    \end{table}
    \noindent Therefore, for \((s,t) = (44,1)\) we have
    \[x=88, y =1935, z=1937,\]
    and for \((s,t)=(11,4)\) we have
    \[x=88,y=105,z=137.\]
\end{mdexample}

\section{Fermat's Last Theorem}

\begin{definition}
    Given \(n\in \NN\), the \(n^{\text{th}}\) \textbf{Fermat equation} is given 
    \[x^n+y^n=z^n.\]
\end{definition}

\begin{mdthm}
    If \(n\geq 3\) there are no positive integer solutions \((x,y,z)\) to the equation
    \[x^n + y^n =z^n.\]
\end{mdthm}

\begin{theorem}
    There are no positive integer solution, \((x,y,z)\), to the equation
    \[x^4+y^4=z^2.\]
\end{theorem}

\begin{mdremark}
    If \((x_0,y_0,z_0)\) satisfy \(x_0^4 + y_0^4 =z_0^2=(z_0^2)^2\). Then \((x_0,y_0,z_0^2)\) is a solution to the \(4^{\text{th}}\) Fermat equation. \\
    Similarly, if \(n=4k\) for \(k \in \NN\) then \(4k^{\text{th}}\) Fermat equation has no solution by the theorem,
    \[x^{4k}+y^{4k}=z^{4k} \iff (x^k)^4+(y^k)^4=(z^{2k})^2.\]
\end{mdremark}

\begin{mdnote}
    We will use Fermat's method of ``descent'': given a solution \((x,y,z)\) we produce another solution \((x',y',z')\) with \(z'<z\). This will be a contradiction if we start the solution by minimising \(z\).
\end{mdnote}

\begin{proof}
    Let \((x,y,z) \in \NN\) be a solution with minimum possible \(z\). 
    \begin{itemize}
        \item If \(\gcd(x,y) >1\) then \(p \mid x\) and \(p \mid y\) for some prime \(p\). Then \(p^4 \mid (x^4+y^4)\) that is \(p^4 \mid z^2\). Hence, \(p^2 \mid z\). Then \((x',y',z')=\left( \frac{x}{p} ,\frac{y}{p},\frac{z}{p^2} \right)\) is a solution in \(\NN\) with \(z'<z\). This is a contradiction.
        \item If \(\gcd(x,y)=1\) then \(\gcd(x^2,y^2)=1\) and so \((x^2,y^2,z)\) is a primitive Pythagorean triple. Without loss of generality, assume that \(x^2\) is even and \(y^2\) is odd, that is \(x\) is even and \(y\) is odd. Hence, there exists \(s,t \in \NN\) with \(\gcd(s,t)=1, s>t>0\) and \(s \not\equiv t \Mod{2}\) such that 
        \[\begin{aligned}
            x&=2st \\
            y&=s^2-t^2 \\
            z&=s^2+t^2.
        \end{aligned}\]
        We can write
        \[t^2+y^2=s^2\]
        therefore, \((t,y,s)\) is a primitive Pythagorean triple with \(t\) even since \(y\) is odd. Applying the Pythagorean Triple theorem again we can write 
        \[\begin{aligned}
            t &= 2uv \\
            y&=u^2-v^2 \\
            s&=u^2+v^2 
        \end{aligned}\]
        with \(\gcd(u,v)=1, u>v>0\) and \(u \not\equiv v \Mod{2}\). Observe that 
        \begin{itemize}
            \item \(\gcd(u,u^2+v^2)=\gcd(u,v^2)=1\);
            \item \(\gcd(v,u^2+v^2)=\gcd(v,u^2)=1\).
        \end{itemize}
        Recall
        \[\begin{aligned}
            x^2 &=2st \\
                &=4uv(u^2+v^2) \\
            \left( \frac{x}{2} \right)^2  &= uv(u^2+v^2).
        \end{aligned}\]
        Hence, \(uv(u^2+v^2)\) is a square which implies \(u,v, u^2+v^2\) are also squares. Since \(\gcd(u,v)=\gcd(u,u^2+v^2)=\gcd(v,u^2+v^2)=1\) then there exists \(x',y',z' \in \NN\) with 
        \[\begin{aligned}
            u=(x')^2, \quad v=(y')^2 \quad \text{and} \quad u^2+v^2 =(z')^2
        \end{aligned}\]
        so,
        \[\begin{aligned}
            u^2+v^2 &= (x')^4+(y')^4 \\
                    &= (z')^2.
        \end{aligned}\]
        This implies \((x',y',z')\) is a solution to Fermat's \(4^{\text{th}}\) equation. Recall,
        \[z=s^2+t^2 \quad \text{and} \quad s=u^2+v^2=(z')^2\]
        hence, \(z>s^2>z'\) which is a contradiction to minimality.
    \end{itemize}
\end{proof}

% \textbf{Strategy:} We will argue by contradiction and use a ``descent'' argument. \textbf{Descent:} given a solution \((a,b,c)\) to \(x^4+y^4=z^2\) produce another solution \((a',b',c') \in \NN^3\) to the equation with \(c'<c\). Repeatedly iterating this argument yields a contradiction. \\
% \textbf{Modification:} Take \((a_*,b_*,c_*) \in \NN^3\) be a \textbf{minimal} solution to the equation; in the sense that if \((a,b,c)\in \NN^3\) is a solution to the equation then \(c\geq c_*\). We will show there exists a solution \((a',b',c')\in \NN^3\) which is a solution to the equation with \(c'<c_*\), which leads to the contradiction.

% \begin{proof}
%     Let \((a_*,b_*,c_*) \in \NN^3\) be a solution to the equation. Assume \(\gcd({a_*}^2,{b_*}^2)=1\) \\ (proof of this assumption: for sake of contradiction there exists a prime with \(p \mid {a_*}^2\) and \(p \mid {b_*}^2\) which implies \(p \mid a_*\) and \(p \mid b_*\). We know \((a_*)^4+(b_*)_*=(c_*)^2\) which implies that \(p^4 \mid (c_*)^2\) so \(p^2 \mid c_*\)). \\
%     Consider \(\left( \frac{a_*}{p},\frac{b_*}{p},\frac{c_*}{p^2} \right)\), this is a solution to the equation, with the condition \(\frac{c_*}{p} <c_*\). Which is a contradiction. Hence, \((a_*^2,b_*^2,c_*)\) is a primitive Pythagorean triple as \(\gcd(a_*,b_*)=1\) and \((a_*^2+b_*^2)\).
%     Apply primitive Pythagorean triple theorem (without loss of generality assume \(a_*\) is even). Write 
%     \[a_*^2=2st, \; b_*^2=s^2-t^2, \; c_*=s^2+t^2\]
%     with \(\gcd(2,t)=1, s>t>0\) and \(s\not\equiv t \Mod{2}\) which implies
%     \[t^2+b_*^2 =s^2\]
%     so, \((t,b_*,s)\) is a primitive Pythagorean triple with \(t\) being even since \(b_*\) is odd. Apply the primitive Pythagorean triple theorem and write,
%     \[t=2uv, \; b_*=u^2-v^2 \; s=u^2+v^2\]
%     with \(\gcd(u,v)=1 ,u>v>0\) and \(u \not\equiv v \Mod{2}\). Observe, \begin{itemize}
%         \item \(\gcd(u,u^2+v^2)=\gcd(u,v^2)=1\);
%         \item \(\gcd(v,u^2+v^2)=\gcd(v,u^2)=1\).
%     \end{itemize}
% \end{proof}

\section{General Diophantine equation}

\begin{definition}
    Given integers \(c_1,c_2,\ldots,c_n \in \ZZ\) a \textbf{Diophantine equation} is an equation of the form
\end{definition}

\begin{proposition}
    Let \(f(x)=c_n x^n + c_{n-1} x^{n-1}+ \cdots c_1 x + c_0\) where \(c_0, \ldots, c_n \in \ZZ\) with \(c_n \neq 0\). If \(a \in \ZZ\) is a root of \(f(x)\) then
    \[f(x)=(x-a)g(x),\]
    where \(g(x)=b_{n-1} x^{n-1}+\cdots +b_1 x +b_0\) where \(b_0,\ldots, b_{n-1} \in \ZZ\).
\end{proposition}

\begin{proposition}
    For each \(k \in \NN\) we have 
    \[a^k - b^k = (a-b)(a^{k-1}+a^{k-2} b + \cdots + a b^{k-2} +b^{k-1}).\]
\end{proposition}

\subsection{Solving Diophantine equations}

There is no general (known) method to solve Diophantine equations, but there are some special cases where there is a method.

\begin{mdprop}
    Suppose \(f(x)=c_n x^n + c_{n-1} x^{n-1}+ \cdots c_1 x + c_0\) with \(c_i\in \ZZ\), if \(f(a)=0\) with \(a \in \ZZ\) then \(a \mid c_0\).
\end{mdprop}

\begin{mdnote}
    \textbf{Strategy:} To solve \(f(x)=0\) with \(x \in \ZZ\) check \(f(d)\) for each \(d \mid c_0\).
\end{mdnote}

\begin{example}
    Find all integer solutions to \(f(x)=0\) for \(f(x)=2x^4-14x^3+3x^2+20x-7\).
    \begin{solution}
        We have \(c_0=-7\) therefore, we must check if \(f(d)=0\) for \(d \mid -7\) i.e. \(d=\pm 1\) or \(d= \pm 7\). Only \(f(7)=0\) hence, \(x=7\) is the only solution to \(f(x)\) in \(\ZZ\).
    \end{solution}
\end{example}

\begin{example}
    Find all integer solutions to 
    \[x^4+4x^2-12xy+9y^2-2=0.\]
    \textbf{Solution:} Notice that \(4x^2-12xy+9y^2 =(2x-3y)^2\). So, we have 
    \[x^4 +(2x-3y)^2 =2.\]
    Since, \(x,y \in \ZZ\) we have that \(x=\pm 1\) as the RHS is \(2\).
    \begin{itemize}
        \item If \(x=1\) then \((2-3y)^2=1 \then y=1\).
        \item If \(x=-1\) then \((-2-3^2)=1 \then y=-1\).
    \end{itemize}
    The solutions \((x,y)\) are \((1,1)\) or \((-1,-1)\).
\end{example}

\begin{example}
    Find all integer solutions to
    \[x^2-3y^4=0.\]
    \textbf{Solutions:} We can rewrite
    \[\begin{aligned}
        x^2&=3y^4 \\
        \left( \frac{x}{y^2} \right)^2 &=3 \text{ for } y\neq 0.
    \end{aligned}\]
    Therefore, the only solution is \((x,y)=(0,0)\).
\end{example}

\subsection{Diophantine and congruence equations}

Consider a Diophantine equation \(x^7 +7y^5 =610\). For each \(m\in \NN\) we get a corresponding congruence equation modulo \(m\). Consider 
\[x^7 +7y^5 \equiv 610 \Mod{m}.\]
In the Diophantine equation we seek solutions in \(\ZZ\) and in the congruence equation we seek solutions in \(\ZZ_m\).

\begin{proposition}
    If a Diophantine equation has a solution in \(\ZZ\) then the corresponding congruence equation has a solution for each \(m\geq 1\) in \(\ZZ_m\).
\end{proposition}

\begin{mdnote}
    Therefore, if the congruence equation has no solution in \(\ZZ_m\) for some \(m\geq 1\) then its associated Diophantine equation has no solution in \(\ZZ\).
\end{mdnote}

\begin{example}
    Solve
    \[x^7 +7y^5 \equiv 610 \Mod{2}.\]
    \begin{solution}
        We note that \(610 \equiv 0 \Mod{2}\) and for all \(a \in \ZZ\) and \(n\in \NN\) we have the following relation in modulo \(2\).
    \[a^n \equiv a \Mod{2}.\]
    Therefore, \(x^7 \equiv x \Mod{2}\) and \(7y^5 \equiv 7y \equiv y \Mod{2}\), since \(7 \equiv 1 \Mod{2}\). So, we are left to solve
    \[x+y \equiv 0 \Mod{2}.\]
    The solutions are \((x,y)=([0]_2,[0]_2)\) or \(([1]_2,[1]_2)\).
    \end{solution}
\end{example}


\begin{example}
    Solve the Diophantine equation
    \[x^{12}+13y^5 =z^{12}+2.\]
    \textbf{Strategy for choosing \(m\):}\\
    Want \(x^{12},13y^5\) and \(z^{12}\) to take on few values modulo \(m\). \\
    \textbf{Recall:} By the Euler-Fermat theorem \(a^{p-1} \equiv 1 \Mod{p}\) if \(p \nmid a\). \\
    \textbf{Solution:} With this in mind consider
    \[x^{12}+13y^5 \equiv z^{12}+2 \Mod{13}.\]
    So, \(x^{12} \equiv 1\) or \(0\) modulo \(13\) if \(13 \nmid x\) and \(13 \mid x\) respectively. The next term \(13y^5 \equiv 0 \Mod{13}\). Thus, we are left with
    \[x^{12} \equiv z^{12} +2 \Mod{13}.\]
    LHS \(\equiv 0\) or \(1\) modulo \(13\). \\
    RHS \(\equiv 2\) or \(3\) modulo \(13\).
    Hence, LHS \(\neq\) RHS. This, congruence equation has no solution in \(\ZZ_{13}\) which implies that the associated Diophantine equation has no solutions
\end{example}

\begin{mdremark}
    This method is not always possible i.e. some equation could have solutions for all \(m\geq 1\) in modulo \(m\) but no integer solutions.
\end{mdremark}

\begin{example}
    Find all integer solutions to 
    \[x^4+y^4=z^4+w^6+3.\]
    \begin{solution}
        We note that 
        \[\begin{aligned}
            x^4 &\equiv 0,1 \Mod{8} \\
            w^6 &\equiv 0,1 \Mod{8}.
        \end{aligned}\]
        Therefore,
        \[\begin{aligned}
            \text{LHS} &\equiv  0,1,2 \Mod{8} \\
            \text{RHS} &\equiv 3,4,5 \Mod{8}
        \end{aligned}\]
        Hence, \(\text{LHS}\not\equiv \text{RHS} \Mod{8}\). This congruence equation has no solutions in \(\ZZ_8\) thus, it will not have solutions in \(\ZZ\).
    \end{solution}
\end{example}

\subsection{Week 12 lectures}

\includepdf[pages=-]{./Resources/nt-lecture-notes-dec13.pdf}

\includepdf[pages=-]{./Resources/nt-notes-dec15.pdf}





\pagebreak

\appendix

\addcontentsline{toc}{section}{Appendix}
\section*{Appendix}

\section{Equivalence relations}

\begin{definition}
    A binary operation on a set \(X\) is said to be an \textbf{equivalence relation}, if and only if it is reflexive, symmetric and transitive. That is for all \(a,b,c \in X\):
    \begin{itemize}
        \item Reflexivity: \(a \sim a\);
        \item Symmetry: \(a \sim b\) if and only if \(b\sim a\);
        \item Transitivity: if \(a\sim b\) and \(b\sim c\) then \(a\sim c\).
    \end{itemize}
\end{definition}

\subsection{Equivalence classes}

\begin{theorem}
    If \(\sim\) is an equivalence relation on a set \(X\) and \(x,y\in X\) then, these statements are equivalent:
    \begin{itemize}
        \item \(x\sim y\);
        \item \([x]=[y]\);
        \item \([x]\cap [y] =\emptyset\)
    \end{itemize}
\end{theorem}

\section{Solving linear congruences}

\begin{proposition}
    Let \(a,b \in \ZZ\) and let \(m\) be a positive integer. Set \(g=\gcd(a,m)\). The congruence relation 
    \[ax \equiv b \Mod{m}\]
    has integer solutions for \(x\) if and only if \(g \mid b\).
    If \(d\mid b\), the solutions are given by the integers \(x\) such that 
    \[
        [x]_{\frac{m}{g}} = \left[\frac{a}{g}\right]\inv_{\frac{m}{d}} \left[ \frac{b}{d} \right]_{\frac{m}{d}.}
    \]
\end{proposition}

\begin{proof}
    If \(ax \equiv b \Mod{m}\) then \(b=ax+km\) for some \(k\in \ZZ\). So \(\gcd(a,m)\) (which divided \(a\) and \(m\)) must divide \(b\). Conversely, if \(d\mid b\) then 
    \[\frac{a}{g}x \equiv \frac{b}{g} \Mod{\frac{m}{g}}\]
    if and only if \(ax\equiv b \Mod{m}\). Multiplying by an inverse of \(\frac{a}{d}\) modulo \(\frac{m}{d}\) we get that 
    \[\frac{a}{g}x \equiv \frac{b}{g} \Mod{\frac{m}{g}}\]
    if and only if 
    \[
        [x]_{\frac{m}{g}} = \left[\frac{a}{g}\right]\inv_{\frac{m}{d}} \left[ \frac{b}{d} \right]_{\frac{m}{d}.}
    \]
\end{proof}

\end{document}