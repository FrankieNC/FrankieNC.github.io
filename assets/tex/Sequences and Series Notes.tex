\documentclass[12pt, a4paper]{article}   	
\usepackage{geometry}
\usepackage{amssymb}
\usepackage{mathtools}
\usepackage{amsmath}
\usepackage{amsthm}
\usepackage[utf8]{inputenc}
\usepackage{color}   
\usepackage{tikz}
\usepackage{tcolorbox}
\usepackage{multicol}
\usepackage[thinc]{esdiff}
\usepackage{physics}
\usepackage{bm}
\usepackage{pdfpages}
\usepackage{pdflscape}
\usepackage{listings}
\usepackage{float}

\usepackage{hyperref}

\hypersetup{colorlinks=true, linktoc=all, linkcolor=black,}

\newcommand{\bb}[1]{\mathbb{#1}}
\newcommand{\f}[2]{\frac{#1}{#2}}
\newcommand{\imply}{\Rightarrow}
\newcommand{\Cal}[1]{\mathcal{#1}}
\newcommand{\mb}[1]{\mathbf{#1}}

\newtheorem*{remark}{Remark}
\newtheorem*{note}{Note}

\theoremstyle{definition}
\newtheorem{definition}{Definition}[section]
\newtheorem{theorem}{Theorem}[section]
\newtheorem*{example}{Example}
\newtheorem{proposition}{Proposition}

\theoremstyle{plain}
\newtheorem{corollary}{Corollary}[theorem]
\newtheorem{lemma}[theorem]{Lemma}

\title{Sequences and Series Notes}
\date{}
\author{Francesco Chotuck}
\begin{document} 
\maketitle 

\tableofcontents

\pagebreak

\section*{Notation}

The notation $\#(A)$ for a set $A$ denotes the cardinality i.e. the size of the set.

\section{Set Theory}

\subsection{Operation with sets}

\begin{definition}
The \textbf{union} of two sets $A$ and $B,$ denotes $A\cup B,$ is the set of all elements which are either in A \textbf{or} B (or both).
\end{definition}

\begin{definition}
The \textbf{intersection,} $A\cap B,$ is the set of those elements that are both in $A$ and in $B.$  
\end{definition}

\begin{definition}
The notation $A\backslash B$ stands for the \textbf{complement} of $B$ in $A,$ i.e. the set of all elements of $A$ which are not in $B.$ 
\end{definition}

\begin{example}
Within the natural numbers $\bb{N}$ suppose $A=\{1,2,3,4\}$ and $B=\{1,2,5\}.$ Then $$A\backslash B=\{3,4\}.$$
\end{example}

\subsection{Intervals}
	
\begin{definition}
The following subsets of $\bb{R}$ are called \textbf{intervals:}
\begin{enumerate}
	
	\item[(i)] $[a,b]=\{x\in \bb{R} : a\leq x \leq b\}.$

	\item[(ii)] $(a,b)= \{x\in \bb{R} : a< x <b\}.$

	\item[(iii)] $(a,b]=\{x\in \bb{R} : a< x \leq b\}.$

	\item[(iv)] $(-\infty,b]=\{x\in \bb{R} : x \leq b\}.$

	\item[(v)] $(a,\infty)= \{x\in \bb{R} : a < x\}.$

	\item[(vi)] $(-\infty, \infty)=\bb{R}.$

\end{enumerate}
\end{definition}

\begin{remark}
To make operations with intervals easier draw the intervals on the number line to visualise the operation.
\end{remark}

\begin{tcolorbox}
Note that the symbol $\infty$ is only used as part of the notation.
$\mb{\infty}$ \textbf{is not a real number! There is no such real number as} $\mb{\infty}.$
\end{tcolorbox}

\subsection{Multiple and infinite unions and intersections}




\section{Functions}

\includepdf[pages=-]{./Resources/Functions.pdf}

\section{Logic}

\subsection{Propositions}

\begin{definition}
For our purposes it will be sufficient to define a \textbf{proposition} to be ‘a statement which is either \textbf{true or false}’. The word statement is often used as a synonym to proposition.
\end{definition}

If $A$ and $B$ are propositions, one can form new propositions as follows:

\begin{itemize}

	\item $A$ AND $B$ - is true if and only if both $A$ and $B$ are true;

	\item $A$ OR $B$ - is true if and only if either ($A$ is true) or ($B$ is true) or (both $A$ and $B$ are true);

	\item NOT $A$ - is true if and only if $A$ is false.

\end{itemize}

\subsection{Quantifiers}

From propositions with variables one can form new propositions by using the quantifiers $\exists$ ‘there exists’ and $\forall$ ‘for all’.

\begin{tcolorbox}
\textbf{If a proposition contains a variable, then without a quantifier one cannot establish whether it is true or false!}
\end{tcolorbox}

\subsection{Negating propositions}

The process to negate propositions is as follows: 

\begin{enumerate}
	
	\item Negate the quantifiers, i.e. the negation of $\forall$ is $\exists$ and the negation of $\exists$ is $\forall.$

	\item Negate the statement. For example:
	$$"\forall x \in \bb{R}, \text{ one has } x>0"$$ the negation is $$"\exists x \in \bb{R}, \text{ one has } x\leq0".$$

\end{enumerate}

\begin{remark}
When negating propositions containing operations AND, OR, one should replace AND by OR and vice versa.
\end{remark}

\begin{example}
Proposition: $\forall a>0$ one has $(a\geq 1)$ OR $(1/a\geq 1).$
Negation: $\exists a>0$ such that $(a<1)$ AND $(1/a <1).$
\end{example}

\subsection{Converse and contrapositive}

\begin{definition}
Let $A$ and $B$ be propositions depending on a variable. For a statement $A \imply B,$ the \textbf{contrapositive} is the statement $(\text{NOT } B) \imply (\text{NOT } A).$ A contrapositive is true if and only if the original statement is true.
\end{definition}

\begin{definition}
For a statement $A \imply B,$ the \textbf{converse} is the statement $B \imply A.$ A converse may or may not be true regardless of whether the original statement is true or not.
\end{definition}

\section{Proofs}

\subsection{Proving two sets are equal}

\begin{theorem}
Let $A$ and $B$ be two sets. Then $A=B$ if and only if both $A \subset B$ and $B \subset A.$
\end{theorem}

\begin{theorem}
For any sets $A,B$ and $C,$ we have: 
\begin{enumerate}

	\item[(i)] $A\cup(B\cap C)= (A\cup B)\cap (A\cup C);$

	\item[(ii)]$A\cap(B\cup C)= (A\cap B)\cup (A\cap C);$

\end{enumerate}
\end{theorem}

\subsection{Proof by contradictions}

In this type of proof instead of proving $A\imply B$ we prove the contrapositive (which may be easier to prove) $\text{NOT }A \imply \text{NOT }B.$

\subsection{Proof by counterexample}

It is enough to find an example in which the proposition proposed does not hold true.

\section{Boundedness; supremum and infimum}

\subsection{Boundedness}

\begin{definition}
A subset $S$ of $\bb{R}$ is said to be \textbf{bounded above} if there is a real number $M$ such that $x \leq M$ for all $x \in S.$

Alternatively: $S$ is bounded above if $S \subset (-\infty,M]$ for some $M \in \bb{R}.$ Such a number $M$ is called an \textbf{upper bound} for $S.$
\end{definition}

\begin{definition}
A subset $S$ of $\bb{R}$ is said to be \textbf{bounded below} if there is a real number m such that $x \geq m$ for all $x \in S.$

Alternatively: $S$ is bounded below if $S \subset [m,\infty)$ for some $m \in \bb{R}$. Such a number $m$ is called a \textbf{lower bound} for $S.$
\end{definition}

\begin{definition}
A subset $S$ of $\bb{R}$ is said to be \textbf{bounded} if it is both bounded above and bounded below. 

Alternatively: $S$ is bounded if $S \subset [m,M]$ for some real numbers $m, M.$

If $S$ is not bounded, it is called \textbf{unbounded}.
\end{definition}

\subsection{Maximum and minimum}

\begin{definition}
Let $S$ be bounded above and suppose that there exists an upper bound $M$ of $S$ such that $M \in S.$ Then $M$ is called the \textbf{maximum} of $S$ (or the maximal element of $S$): $M = \max(S).$
\end{definition}

\begin{definition}
Let $S$ be bounded below and suppose that there exists a lower bound $m$ of $S$ such that $m \in S.$ Then $m$ is called the minimum of $S$ (or the minimal element of $S$): $m = \min(S).$
\end{definition}

\begin{remark}
The set $S=(0,1)$ is bounded. However, neither maximum nor minimum exist for this set.
\end{remark}

\begin{theorem} \hphantom{This is to make space for formatting}
\begin{enumerate}
	
	\item[(i)] Let $S$ be bounded above. If $\max(S)$ exists, then it is unique.

	\item[(ii)] Let $S$ be bounded below. If $\min(S)$exists, then it is unique.

\end{enumerate}
\end{theorem}

\subsection{Supremum and infimum}

\begin{definition}
Let $S$ be bounded below. Suppose that there exists the largest number $m$ such that $S \subset [m, \infty).$ Then $m$ is called the greatest lower bound for $S,$ or the \textbf{infimum} of $S,$ denoted $\inf(S).$
\end{definition}

\begin{theorem}
Let $S$ be bounded from below. If $\min(S)$ exists, then $\inf(S)$ also exists and coincides with $\min(S).$
\end{theorem}

\begin{theorem}
Let $S\subset \bb{R}$ be a set bounded below and let $m=\inf{S}.$ Then for any $\varepsilon>0$ there exists $x\in S$ such that $x<m+\varepsilon.$
\end{theorem}

\begin{definition}
Let $S$ be bounded above. Suppose that there exists the smallest number $M$ such that $S \subset (-\infty,M]$. Then $M$ is called the least upper bound for $S,$ or the \textbf{supremum} of $S,$ denoted $\sup(S).$
\end{definition}

\begin{theorem}
Let $S$ be bounded from above. If $\max(S)$ exists, then $\sup(S)$ also exists and coincides with $\max(S).$
\end{theorem}

\begin{theorem}
Let $S \subset R$ be a set bounded above and let $M = \sup(S).$ Then for any $\varepsilon > 0$ there exists $x \in S$ such that $M -\varepsilon < x.$
\end{theorem}

\begin{note}
What the theorem above is saying is that number formed by taking a small number, $\varepsilon,$ away from the supremum of a set is no longer an upper bound of the set; therefore it cannot be the supremum of the set.
\end{note}

\begin{remark}
The supremum and infimum of a set by definition do not need to be in the set itself.
\end{remark}

\subsection{Completeness}

\begin{definition} \textbf{(Axiom of completeness).}
Every non-empty set of real numbers which is bounded above has a supremum. 

Every non-empty set of real numbers which is bounded from below has an infimum.
\end{definition}

\section{Sequences: Convergence}

\subsection{Sequences}

We say that a list of numbers $s_1,s_2,s_3,\ldots$ generated by a general formula such as, $$s_n=\f{1}{n^2} \quad n\in\bb{N},$$ is a sequence.

\begin{remark}
Some types of notations for sequences are $\{s_n\}_{n=n_0}^{\infty},$ alternatively $(s_n)_{n\geq n_0}.$
\end{remark}

\subsection{Convergence}

\begin{definition}
The sequence $s_n$ is said to \textbf{converge} to the limit $L$ if for all $\varepsilon > 0$ there exists a natural number $n_0$ such that for all $n\geq n_0$ we have $$|s_n-L|<\varepsilon.$$
\end{definition}

\begin{tcolorbox}
In symbols:
$$\forall \varepsilon >0 \: \exists n_0 \in \bb{N} \text{ such that } \forall n\geq n_0 \text{ we have } |s_n-L|<\varepsilon.$$
\end{tcolorbox}

\begin{remark}
The defintion does not change if $n>n_0$ or if $\varepsilon = f(\varepsilon).$
\end{remark}

\begin{theorem}
Every convergent sequence has one and only one limit.
\end{theorem}

\subsection{Divergence}

\begin{definition}
If the sequence $s_n$ does not converge to any limit it is said to \textbf{diverge}.
\end{definition}

\begin{definition}
The sequence $s_n$ is said to diverge to $+\infty,$ for which we write $s_n\to +\infty,$ if, for every positive real number $H,$ there exists $n_0$ such that for all $n\geq n_0$ we have $$s_n>H.$$ i.e. $$\forall H>0 \; \exists n_0 \in \bb{N} \; \forall n\geq n_0 \; s_n>H.$$
\end{definition}

\begin{definition}
The sequence $s_n$ is said to diverge to $-\infty,$ for which we write $s_n \to -\infty$ if, for any negative real number $H,$ there exists $n_0$ such that for all $n \geq n_0$ we have $$s_N<H.$$ i.e. 
$$\forall H<0 \; \exists n_0 \in \bb{N} \; \forall n\geq n_0 \; s_n<H.$$
\end{definition}

\begin{example}
TO DO !!
\end{example}

\begin{example}
TO DO !!
\end{example}

\subsection{Boundedness}

\begin{definition}
The sequence $s_n$ is said to be \textbf{bounded} if its terms form a bounded set. i.e., if there exists a real number $M$ such that $|s_n| \leq M$ for all $n \in \bb{N}.$
\end{definition}

\begin{theorem}
Every convergent sequence is bounded.
\end{theorem}

\section{The algebra of limits}

\begin{theorem}
Let $s_n \to L$ and $t_n \to M$ as $n\to \infty.$ Then 
\begin{enumerate}
 	
 	\item for any number $\alpha \in \bb{R},$ one has $\alpha s_n\to \alpha L$ as $n\to\infty;$

 	\item $s_n+t_n\to L+M$ as $n\to\infty;$

 	\item $s_nt_n\to LM$ as $n\to\infty;$

 	\item $s_n/t_n \to L/M$ as $n\to\infty,$ provided $m\neq 0$ and $t_n\neq 0$ for all $n.$
 
\end{enumerate} 
\end{theorem}

\begin{theorem}
Let $s_n \to \infty$ or $s_n \to -\infty$ as $n \to \infty$ and $s_n \neq  0$ for all $n$. Then $1/{s_n} \to 0$ as $n \to \infty.$
\end{theorem}

\begin{theorem}
Let $s_n \to L>0$ and let $t_n \to \pm\infty$ as $n\to\infty.$ Then $s_nt_n \to \pm\infty$ as $n\to \infty.$
\end{theorem}

\subsection{Sandwich theorem}

\begin{theorem}
Let $r_n \to L$ and $t_n \to L$ as $n \to \infty$ and suppose that $r_n \leq s_n \leq t_n$ for all $n\in\bb{N}.$ Then $s_n \to L$ as $n\to \infty.$
\end{theorem}	

\begin{example}
TO DO !!
\end{example}

\subsection{Limits with inequalities}

\begin{theorem}
Let $s_n, t_n$ be convergent sequences such that $s_n \leq t_n$ for all $n \in \bb{N}.$ Then $\lim_{n\to \infty}s_n \leq \lim_{n\to \infty} t_n.$
\end{theorem}

\begin{corollary}
Let $s_n, t_n$ be convergent sequences such that $s_n < t_n$ for all $n \in \bb{N}.$ Then $\lim_{n\to \infty}s_n \leq \lim_{n\to \infty} t_n.$
\end{corollary}

\begin{remark}
This is because $s_n<t_n \imply s_n\leq t_n.$ The use of $<$ in the inequality for limits may be FALSE sometimes. 
\end{remark}

\section{Standard sequences}

The following are `standard' sequences that converge to zero:

\begin{itemize}

	\item Exponentials: $a^n \to 0$ as $n \to \infty,$ $\forall a\in (-1,1);$

	\item Powers: $n^{-\gamma}\to 0$ as $n \to \infty,$ $\forall \gamma>0;$

	\item Logarithms: $\f{1}{\log(n)} \to 0$ as $n \to \infty.$

\end{itemize}

Alternately we also have the following sequences which diverge to $\infty:$

\begin{itemize}
	\item Exponentials: $A^n \to \infty$ as $n \to \infty,$ $\forall A>1;$

	\item Powers: $n^{\gamma}\to \infty$ as $n \to \infty,$ $\forall \gamma >0;$

	\item Logarithms: $\log(n) \to \infty$ as $n \to \infty.$
\end{itemize}

\subsection{Rates of convergence}

\begin{figure}[H]
\centering
\includegraphics[width=1\textwidth]{./Resources/Rates of convergence.png}
\end{figure}

\subsubsection{Bernoulli inequality}

\begin{lemma}\textbf{(Bernoulli inequality).} For every $k\in \bb{N}$ and $x\geq -1$ one has $$(1+k)^k\geq 1+kx.$$

\end{lemma}

\subsubsection{The sequence \texorpdfstring{$a^{1/n}$}{TEXT}}

\begin{theorem}
For any $a>0,$ we have $$a^{1/n} \to 1 \text{ as } n\to \infty.$$ 
\end{theorem}

\section{Monotone sequences}

\begin{definition}
A sequence $s_n$ is said to be:
\begin{itemize}

	\item \textbf{increasing}, if $s_{n+1} > s_n$ for all $n \in \bb{N}$;

	\item \textbf{decreasing}, if $s_{n+1} < s_n$ for all $n \in \bb{N}$;

	\item \textbf{non-decreasing}, if $s_{n+1} \geq s_n$ for all $n \in \bb{N}$;

	\item \textbf{non-increasing}, if $s_{n+1} \leq s_n$ for all $n \in \bb{N}$;

	\item \textbf{monotone} if it is either non-decreasing or non-increasing.

\end{itemize}
\end{definition}

\begin{theorem} \hphantom{This is to make space for formatting}
\begin{enumerate}
	
	\item Every non-decreasing sequence which is bounded above is convergent. Moreover, the limit of the sequence is the supremum of its terms.

	\item Every non-increasing sequence which is bounded below is convergent. Moreover, the limit of the sequence is the infimum of its terms.

\end{enumerate}
\end{theorem}

\section{Series}

A series is just a special type of sequence.

\begin{definition}
Let $\{a_k\}_{k=1}^{\infty}$ be a sequence of real numbers. We say that the \textbf{series} $\sum_{k=1}^{\infty} a_k$ converges if the limit $$\lim_{n\to \infty} \sum_{k=1}^{n} a_k$$ exists. The limit is called the sum of the series $\sum_{k=1}^{\infty} a_k.$ So by definition, $$\sum_{k=1}^{\infty} a_k =\lim_{n\to\infty} \sum_{k=1}^{n} a_k$$ if the limit exists.
\end{definition}

\begin{remark}
Essentially if a sequence is composed of $a_1,a_2,a_3,\ldots$ then its corresponding series is $S_n=a_1+a_2+a_3+\ldots$
\end{remark}

\begin{theorem}
If the series $\sum_{k=1}^{\infty} a_k$ is convergent then $\lim_{k=\infty} a_k =0.$
\end{theorem}

\subsection{Geometric series}

Let $|a|<1.$ Then the series $$\sum_{k=0}^{\infty} a_k = \f{1-a^{n+1}}{1-a}.$$ Furthermore $$\sum_{k=0}^{\infty} a_k =\lim_{n\to\infty} \sum_{k=0}^{\infty} a_k =\f{1}{1-a}.$$ Clearly if $|a|>1$ the series diverges.

\subsection{Series with positive terms}

\begin{corollary}
Let $a_k \geq 0$ be a sequence of non-negative real numbers. Suppose that there exists a constant $C>0$ such that for all $n\in \bb{N}$ $$\sum_{k=1}^{\infty} a_k\leq C.$$ Then the series $\sum_{k=1}^{\infty} a_k$ converges and the sum, $s,$ of this series satisfies $s\leq C.$
\end{corollary}

\subsection{Special series}

\includepdf[pages=-]{./Resources/Special series.pdf}

\section{Subsequences and limit points}

\subsection{Subsequences}

\begin{definition}
Let $\{s_n\}_{n=1}^{\infty}$ be a sequence of real numbers and let $n_k$ be a strictly increasing sequence of natural numbers (i.e. $1 \leq n_1 < n_2 < n_3 < \ldots$). Then $\{s_{n_k}\}_{k=1}^{\infty}$ is called a \textbf{subsequence} of $\{s_n\}_{n=1}^{\infty}.$
\end{definition}

\begin{example}
Let $s_n=\f{1}{n}$ and $n_k=k^2.$ Then $s_{n_k}=s_{k^2}=\f{1}{k^2}.$
\end{example}

\begin{theorem}
Let $s_n$ be a convergent sequence with $s_n \to L$ as $n \to \infty.$ Then every subsequence of $s_n$ converges to $L.$
\end{theorem}

\begin{theorem}\textbf{(The Bolzano-Weierstrass Theorem).}
Every bounded sequence has a convergent subsequence. 

More precisely, if $s_n$ is a sequence of real numbers such that $a \leq s_n \leq b$ for all $n,$ then there exists a subsequence of $s_n$ which converges to a limit $L \in [a, b].$
\end{theorem}

\subsection{Limit points}

\begin{definition}
A real number a is called a \textbf{limit point} of a sequence $s_n, n \in \bb{N},$ if there exists a subsequence $s_{n_k}$ such that $a = \lim_{k\to\infty} s_{n_k}.$
\end{definition}

\begin{theorem}
A convergent sequence has one and only one limit point; this point is the limit of the sequence.
\end{theorem}

\begin{theorem}
Let $s_n, n \in \bb{N},$ be a bounded sequence which has only one limit point $L.$ Then $s_n \to L$ as $n\to\infty.$
\end{theorem}

\subsection{Cauchy sequences}

\begin{definition}
The sequence $s_n$ is called a \textbf{Cauchy Sequence} if for all $\varepsilon>0$ there exists a natural number $n_0$ such tha for all $m,n \geq n_0$ we have $|s_m-s_n|<\varepsilon.$
\end{definition}

\begin{theorem}
Every convergent sequence is a Cauchy sequence.
\end{theorem}

\begin{theorem}
Every Cauchy sequence is bounded.
\end{theorem}

\begin{theorem}\textbf{(Cauchy's convergent Criterion)}
Every Cauchy sequence is a convergent sequence.
\end{theorem}

\begin{theorem}
A sequence of real numbers is a Cauchy sequence if and only if it is convergent.
\end{theorem}

\section{Absolute and conditional convergence of series}

\begin{definition}
The series $\sum_{k=1}^{\infty} a_k$ is said to \textbf{converge absolutely} if the series $\sum_{k=1}^{\infty} |a_k|$ is convergent.
\end{definition}

\begin{theorem}
Every absolutely convergent series is convergent. 
I.e. if $\sum_{k=1}^{\infty} |a_k|$ converges then $\sum_{k=1}^{\infty} a_k$ converges as well.
\end{theorem}

\subsection{Conditional convergence}

\begin{definition}
If the series $\sum_{k=1}^{\infty} a_k$ converges, but does not converge absolutely, it is said to \textbf{converge conditionally.}
\end{definition}

\subsection{Tests for convergence}

\begin{theorem} \textbf{(The Comparison Test).}
Let $\sum_{k=1}^{\infty} b_k$ be a convergent series of non-negative numbers and suppose that for some constant $M > 0$ we have $|a_k| \leq Mb_k$ for all $k.$ Then the series $\sum_{k=1}^{\infty} a_k$ is absolutely convergent.
\end{theorem}

\begin{theorem}\textbf{(The Alternating Series Test).}
Let $a_k$ be a non-increasing sequence of positive numbers such that $a_k \to 0$ as $k \to \infty$. Then the series $$\sum_{k=1}^{\infty} (-1)^{k+1}a_k =a_1-a_2+a_3-a_4+\ldots$$ converges. Moreover, its sum lies between $a_1-a_2$ and $a_1.$
\end{theorem}

\pagebreak

\section{Appendix}

\subsection{The \texorpdfstring{$O$}{TEXT} symbol}

\begin{definition}
Let $s_n$ and $t_n$ be sequences. Suppose that there exist $C > 0$ such that for all $n \in N,$ one has $|s_n| \leq C|t_n|.$ Then one writes $$s_n=O(t_n) \text{ as } n\to \infty.$$
\end{definition}

\begin{definition}
Let $s_n$ and $t_n$ be two sequences, such that $t_n \neq 0$for all $n.$ Suppose that $\lim_{n\to \infty} \f{s_n}{t_n}=0$  Then we write $s_n =o(t_n),n\to\infty.$
\end{definition}

\subsection{Modulus}

\begin{definition}
For $x\in \bb{R}$ the \textbf{modulus} of $x,$ $|x|,$ is defined as $$\begin{aligned}
|x|=\begin{cases}
x \quad &\text{if $x\geq0,$} \\
-x \quad &\text{if $x<0$.}
\end{cases}
\end{aligned}$$
\end{definition}

\begin{theorem}\textbf{(Properties of the modulus)}
\begin{enumerate}
	
	\item $\forall x,y \in \bb{R}:$ $|xy|=|x||y|;$ in particular, $|ax|=a|x|$ if $a>0;$

	\item The triangle inequality: $\forall x,y \in \bb{R}:$ $|x+y|\leq |x|+|y|;$ 

	\item $\forall x,y \in \bb{R}:$ $|x-y|\leq |x|+|y|.$ 

\end{enumerate}
\end{theorem}

\subsection{Floor and Ceiling functions}

Property: $$x-1< \lfloor x\rfloor \leq x \leq \lceil x\rceil< x+1.$$

\end{document}